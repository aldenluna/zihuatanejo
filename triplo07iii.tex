\documentclass[]{article}
\usepackage[bf,sf,center]{titlesec} % Required for modifying section titles - bold, sans-serif, centered
\usepackage{fontspec,xltxtra,xunicode}
\defaultfontfeatures{Mapping=tex-text}
%%%%%%%%%%%%%%%%%%%%%%%%%%%% %%%%%%%%%%%%%%
\newcommand{\entry}[4]{\noindent{\ene{#1}}\markboth{#1}{#1}\ \textbf{(#2)}\ {#3}\ $\bullet$\ {#4}\par} %Defines the command to print each word on the page,
\usepackage[marginpar=1cm]{geometry}
 \geometry{a4paper, total={170mm,257mm},left=20mm, top=20mm, }
 %%%%%%%%%%%%%%%%%%%%%%%%%%%% %%%%%%%%%%%%%%
%%%% multicomluna
\usepackage{multicol}
\setlength{\columnsep}{1cm}
%%%
\usepackage[inline]{enumitem} % Required for list customisation
%%%%%%%%%%%%%%%%%%%%%%%% paquete geometrico %%%%%%%%%%%% 
\usepackage{polyglossia}
%\usepackage{csquotes} % Causes some error messages?
% Load font from current working directory. 
\setmainfont[
  BoldFont          = LinLibertine_RZ.otf,
  ItalicFont        = LinLibertine_RI.otf,
  BoldItalicFont    = LinLibertine_RZI.otf,
  Ligatures=TeX]{LinLibertine_R.otf}
\setsansfont[
  BoldFont          = LinBiolinum_RB.otf,
  ItalicFont        = LinBiolinum_RI.otf,
  BoldItalicFont    = LinBiolinum_RI.otf, % FIXME?!
  Ligatures=TeX]{LinBiolinum_R.otf}
\setmainlanguage{spanish}
%\setotherlanguage{russian}
%----------------------------------------------------------
\usepackage{tikz}
%---------------------------------------------------------------
\newfontfamily{\ena}{Noto Sans CJK TC}
\newfontfamily{\ene}{Noto Serif CJK TC}
%---------------------------------------------------------------
%\usepackage{propio1}
\usepackage{comand0}

%\usepackage{propio3} %%% aqui esta abreviacion
%---------------------------------------------------------------
\usepackage{array}
\usepackage{paracol}
\usepackage{textcomp}
\usepackage{hyperref}% no funciona con \enese 罪
\hypersetup{
    colorlinks=true,
    linkcolor=black,
    filecolor=magenta,      
    urlcolor=cyan,
    pdftitle={guangyan},
   % bookmarks=true,
    pdfpagemode=FullScreen,
}
\titleformat{\section}[display]
  {\normalfont\bfseries\filcenter}{\LARGE}{1ex}
  {\titlerule[2pt]\vspace{2ex}}[\vspace{1ex}{\titlerule[2pt]}]
  %{\LARGE}
\titleformat{\subsection}[display]
  {\normalfont\bfseries\filcenter}{\thesubsection}{6pt}{\Large} %%\thesubsection pone el subsection
  \titleformat{\subsubsection}[runin]
  {\normalfont\bfseries}{\thesubsubsection}{6pt}{\Large}
%\documentclass{paladio_24}
\usepackage{etoolbox}
%\patchcmd{\chapter}{\thispagestyle{plain}}{\thispagestyle{fancy}}{}{}
\renewcommand{\thesubsection}{\arabic{subsection}} %descomentar para el sub
\renewcommand{\thesubsubsection}{\arabic{subsubsection}}
%\begin{document}
%\usepackage{tocbibind}
  %\newfontfamily\discreto[Ligatures={Discretionary}]{Junicode}
\usepackage{letltxmacro}
\usepackage[xindy]{imakeidx}
%\usepackage{blindtext}
\makeatletter
% Global redefinition of indexentry to section instead of page%
\renewcommand{\imki@wrindexentrysplit}[3]{%
 \expandafter\protected@write\csname#1@idxfile\endcsname{}%
    {\string\indexentry{#2}{\thesubsubsection}}%
}%
\makeatother
%\usepackage{blindtext}
\makeindex[name=fon, columns=5 ,title=Índice Fónetico, intoc, 
options= -M subindice.xdy]
\makeindex[name=esquinas, columns=5, title={Índice Cuadrangular}, intoc, options= -M subindice.xdy]

%\newcommand{\BH}[1]{\large\textbf{\hyperpage{#1}}\normalsize}
\newcommand{\IN}[1]{\index{#1}}
%https://tex.stackexchange.com/questions/22012/hyperref-and-imakeidx-do-not-work-together予
%\usepackage[style=ieee,backend=biber]{biblatex} 
%\addbibresource{referencias.bib} 
\author{Alden Omar Luna Zúñiga}
\title{Proeycto Zihuatanejo}
\begin{document} \newcounter{subsss}
%\maketitle
\begin{multicols}{4}
  \tableofcontents
\end{multicols}
\pagebreak
\section{1 Trazo}\label{1trazo} 
\begin{multicols}{2}
\subsection{\ene 一 }\label{unos} %\input{1trazo/uno}
\subsubsection{\ene 一}\label{一}
\index[esquinas]{$1000$!{\ene 一}} \index[fon]{yi!{\ene 一}}
 {\Large 1 [1,0] \textbf{ \begin{tabular}{|c | } \hline %
y\={\i}  \\ \hline
\end{tabular} } }%%
\begin{enumerate} [noitemsep,label=\Roman{enumi}.,ref=\Roman{enumi}, leftmargin=*]
\item \textit{numeral/adjetivo/adverbio}  {\begin{enumerate*}[label=\textbf{\arabic*})]
\item  uno, 1; \item primero; \item único, unitario.
\end{enumerate*}}
\item \textit{conjunción} {\begin{enumerate*}[label=\textbf{\arabic*})]
\item  ocasión; cada vez; \item como cuando.
\end{enumerate*}}
\end{enumerate}
\entry{一切}{y\={\i}qiè'}{}{todo, entero, global}

\subsubsection{\ene 丄}\label{丄}
\index[esquinas]{$2010_{0}$!{\ene 丄}} \index[fon]{shang!{\ene 丄}}
{\Large 2 [1,1]} \** % \begin{tabular}{|c | c| } \%пень & нем \\ \hline\end{tabular}
\enlugarde\ {\ene 上} \vease\ \textnumero\ \ref{上}% {\ene\ 上 例} omnn N. 


\subsubsection{\ene 丅} \label{丅} 
\index[esquinas]{$1020_{0}$!{\ene 丅}}\index[fon]{xie!{\ene 丅}}
{\Large 2 [1,1]} \enlugarde\ {\ene 下}
\vease\ \textnumero\ \ref{下}% {\ene\ 上 例} omnn N. 

\subsubsection{\ene 丁}\label{丁} 
\index[esquinas]{$1020_{0}$!{\ene 丁}}
\index[fon]{ding!{\ene 丁}}\index[fon]{zheng!{\ene 丁}}
{\Large 2 [1,1] \textbf{ \begin{tabular}{|c | } \hline %\large
 d\v{\i}ng\\ \hline\end{tabular} }}
\textit{en repetición} {\Large \textbf{ \begin{tabular}{|c | } \hline
 zh\={e}ng\\ \hline\end{tabular} } }%\par
trabajador, mayor de edad. \par
Exponer, empujar. \par


% \subsubsection{\ene  ㄘ} %%%% yyyxt  ㄘ
 %{\Large 2 [1, 1]}  \enlugarde\ {\ene 七  } \vease\ \textnumero\ ref.; \textit{ no confundir con el símbolo del alfabeto zhuyin}
 
\subsubsection{\ene 七}\label{七} %%% ju 七
\index[esquinas]{$4070_{1}$!{\ene 七}}\index[fon]{qi!{\ene 七}}
{\Large 2 [1,1] \textbf{ \begin{tabular}{|c| } \hline
q\={\i} \\ \hline \end{tabular} } } \textit{también}\ 
{\Large \begin{tabular}{|c| } \hline
qí \\ \hline\end{tabular}} \par
Siete; 7.\\
\entry{七一}{q\={\i}y\={\i}}{} {Primer número del séptimo mes lunar; 
1 de Julio (Día de la fundación del Partido Comunista Chino, 1921)}	

\subsubsection{\ene 三} \label{三}
\index[esquinasl]{$1010_{1}$!{\ene 三}}\index[fon]{san!{\ene 三}}
{\Large 3 [1, 2] \textbf{\begin{tabular}{|c| c | c | } \hline %\large
s\={a}n & s\`{a}n & s\={a}  \\ \hline
\end{tabular} } } \\ \par% \marginalnote{xxx}

\textbf{s\={a}n} tres; 3; {\ene 三人合作} tres personas trabajaron en colaboración;
{\ene 三十三} 33.\par 

\subsubsection{\ene 上}\label{上}
\index[esquinas]{$1010_{9}$!{\ene 上}}\index[fon]{shang!{\ene 上}}
{\Large 3 [1, 2] \textbf{\begin{tabular}{|c | c | c | } \hline 
sh\`{a}ng & sh\v{a}ng & -sh\r{a}ng \\ \hline
\end{tabular}} }\\ \par
I. \textbf{sh\`{a}ng} Superior, arriba, encima.
\entry{上晝}{shàng-zhòu}{}{al mediodía.}
\entry{上上}{shàng'shàng}{}{óptimo, el mejor; la primera (más favorable) de diez posibilidades; lo más ventajoso.}
%\entry{上工}{shàng g\ouno ng}{}{ir al trabajo; comenzar el trabajo.}

\subsubsection{\ene 下} \label{下}
\index[esquinas]{$1023_{0}$!{\ene 下}}\index[fon]{xia!{\ene 下}}
{\Large 3 [1, 2] \textbf{ \begin{tabular}{|c | c| } \hline %\large
xi\`{a}& -xi\r{a}\\ \hline \end{tabular} } }\\ \par
I \textbf{xi\`{a}} \emph{adj./conv.} \textbf{1}) Inferior, abajo; {\ene 門牙} incisivos inferiores; {\ene 下視} mirar [de arriba] abajo.
\entry{下晝}{xiàzhòu}{}{segunda mitad del día, después del mediodía, después de comer.}
\entry{下江}{xiàji\={a}ng'}{}{bajo Yang-Tse.}
\entry{下雪}{xiàxiè}{}{nevada}
\entry{下世}{xià-shì}{}{ ir al otro mundo, morir; dos\ el siguiente siglo, vida futura; tres\ }
\entry{下日}{xià'rì}{}{uno\ [en el] siguiente día; por la mañana; dos\ en breve, }

\subsubsection{\ene 兀}%\label{兀} 
\index[esquinas]{$1022_{0}$!{\ene 兀}}\index[fon]{ji!{\ene 兀}}
{\Large 3 [1, 2] \textbf{ \begin{tabular}{| c | } \hline
j\={\i}  \\ \hline
\end{tabular}}} \\ \par
Soporte.\par

\subsubsection{\ene 丈}\label{丈}
\index[esquinas]{$5000_{0}$!{\ene 丈}}\index[fon]{zhang!{\ene 丈}}
{\Large  3 [1, 2] \textbf{ \begin{tabular}{| c | } \hline %
zhàng \\ \hline \end{tabular} }}\\ \par

\subsubsection{\ene 车}\label{车} %%% 车 jq
\index[esquinas]{$4050_{0}$!{\ene 车}}
\index[fon]{che!{\ene 车}}\index[fon]{ju!{\ene 车}}
 {\Large 4 [1, 3], [159, 0]} \abreviacion\ \enlugarde\ {\ene 車} \vease\ \textnumero\ ref. 

\subsubsection{\ene 开}%\label{开}
\index[esquinas]{$1044_{0}$!{\ene 开}}\index[fon]{kai!{\ene 开}}
4 [1, 3], [55, 1] \abreviacion\ \enlugarde\ {\ene 開} \vease\  \textnumero\ ref.

\subsubsection{\ene 冇}\label{冇}%xkb
\index[esquinas]{$4022_{7}$!{\ene 冇}}\index[fon]{mou!{\ene 冇}}
{\Large 4 [1, 3]  \textbf{\begin{tabular}{|c | } \hline 
m\v{o} u  \\ \hline\end{tabular} } } \\ \par
negación

%\subsubsection{  \ene 韦}\label{韦} 
%\index[esquinas]{$5002_{7}$!{\ene 韦}}\index[fon]{wei!{\ene 韦}} 
%4 [1, 3] \abreviacion\ \enlugarde\ {\ene 韋} \vease\ \textnumero\ ref. %\vease\ 

\subsubsection{\ene 乌}\label{乌}% pvsm
\index[esquinas]{$2712_{7}$!{\ene 乌}}\index[fon]{wu!{\ene 乌}}
4 [1, 3] \abreviacion\ \enlugarde\ {\ene 鳥} %hayf
 \vease\ \textnumero\ ref.  

\subsubsection{\ene 不}\label{不}
\index[esquinas]{$1090_{0}$!{\ene 不}}\index[fon]{bu!{\ene 不}} 
{\Large 4 [1, 3] \textbf{ \begin{tabular}{| c | } \hline %\large
bù \\ \hline \end{tabular}} }
\begin{tabular}{c}
\textit{se lee en fuentes antiguas con}\\
\textit{frecuencia reemplazado por:}\\
\end{tabular}    \par {\Large
\textbf{ \begin{tabular}{| c c | c | c | c |} \hline
f\otres u & f\ouno u & f\etres i & p\euno i & b\iuno \\ \hline
\end{tabular}} } \\ \par
\begin{tabular}{c}
\textit{en nombres propios}\\
\textit{ se lee también:}\\
\end{tabular} {\Large \textbf{ \begin{tabular}{| c | } \hline
biáo \\ \hline
\end{tabular} } }\\ \par
I. \textit{negación universal}

\subsubsection{\ene 丕}\label{丕}  
\index[esquinas]{$1010_{9}$!{\ene 丕}}\index[fon]{pi!{\ene 丕}}
{\Large 5 [1, 4], 6 [7, 4] \textbf{ \begin{tabular}{|c | } \hline
p\iuno  \\ \hline\end{tabular}} }\\ \par
inmenso, grande, gigante; mucho, fuerte; muy, demasiado; del todo; excesivo; desmesurado.\par
II tomar, recibir educación superior (completar); {丕命} recibir órdenes; aceptar mandato (Cielo).\par 
 peseta.\par 
\entry{丕塞地}{p\={\i} sài't\={a} }{}{peseta.}

\subsubsection{\ene 丘} \label{丘} 
\index[esquinas]{$7210_{2}$!{\ene 丘}}\index[fon]{qi!{\ene 丘}} 
{\Large 5 [1, 4] \textbf{ \begin{tabular}{| c | } \hline
qi\={u} \\ \hline \end{tabular} } }\\ \par


\subsubsection{\ene 业 } \label{业} 
\index[esquinas]{$3210_{0}$!{\ene 业}}\index[fon]{ye!{\ene 业}} 
{\Large 5 [1, 4]} \abreviacion\ \enlugarde\ {\ene 業 } \vease\ \textnumero\ ref.

\subsubsection{\ene 且} \label{且}
\index[esquinas]{$7710_{2}$!{\ene 且}}
\index[fon]{qi!{\ene 且}} \index[fon]{ju!{\ene 且}} 
{\Large 5 [1, 4] \textbf{ \begin{tabular}{| c | c | c | c | } \hline
qi\etres & j\uuno & qù & z\utres \\ \hline
\end{tabular}} }\\ \par
I. \textbf{qi\v{e}} 

\subsubsection{\ene 世} \label{世}% pt 世
\index[esquinas]{$4471_{7}$!{\ene 世}}\index[fon]{shi!{\ene 世}} 
 {\Large 5 [1, 4] \textbf{ \begin{tabular}{| c | } \hline
shì \\ \hline\end{tabular}} }\\ \par
generación, género; generaciones (\textit{pasada, presente y futura});  de generación en generación; tradición.
\entry{世紀}{shì'jì}{}{siglo, centenario.}
\subsubsection{\ene 丽}\label{丽}%%%% mmbb 丽
\index[esquinas]{$1022_{7}$!{\ene 丽}} \index[fon]{liang!{\ene 丽}}  
{\Large 7 [1, 6]} %\abreviacion\ \enlugarde\ {\ene 麗} \vease\ \textnumero\ ref.  %%%%麗 mmbbp
 
\subsubsection{\ene 两}\label{两} %%% moob
\index[esquinas]{$1022_{7}$!{\ene 两}} \index[fon]{liang!{\ene 两}} 
{\Large 7 [1, 6]} \abreviacion\ \enlugarde\ {\ene 兩 } \vease\ \textnumero\ ref. 

\subsubsection{\ene 严}\label{严} %%% mtch 严
\index[esquinas]{$1020_{1}$!{\ene 严}} \index[fon]{yan!{\ene 严}} 
{\Large 7 [1, 6]} \abreviacion\ \enlugarde\ {\ene 嚴 } \vease\ \textnumero\ ref. 
\textnumero\
\subsubsection{\ene 並} \label{並} %%% ttc
 \index[esquinas]{$8010_{2}$!{\ene 並}}\index[fon]{pang!{\ene 並}} 
\index[fon]{bing!{\ene 並}} 
 {\Large 8 [1, 7] \textbf{ \begin{tabular}{| c | c | } \hline %\large
bìng & pàng \\ \hline
\end{tabular}} }\\ \par
%I. \textbf{bìng} \verbo\ \uno\ estar en fila.
\setcounter{subsss}{\value{subsubsection}} 

\subsection{\ene 丨}\label{doss} 
\setcounter{subsubsection}{\arabic{subsss}} %\input{1trazo/dos} 
%\subsubsection{a}sdf
%\guillemotleft % \guillemotright
\subsubsection { \ene 丨}\label{丨} 
\index[esquinas]{$2000_{0}$!{\ene 丨}}\index[fon]{gun!{\ene 丨}}
{\Large 1 [2,0]\textbf{ \begin{tabular}{|c | } \hline
g\v{u}n  \\ \hline
\end{tabular} } }\\ \par
I. sustantivo\ uno\ palo, vara, bastón, varilla (\textit{trazo vertical axial en los ideogramas}).
II. CONVENCION: \textit{símbolo, que representa el jeroglífico inicial en las entradas de los artículos de los diccionarios chinos. } \vease\ ref. Paladio.

\subsubsection { \ene 丩}\label{丩} %%xxvl 
\index[esquinas]{$2270_{0}$!{\ene 丩}}\index[fon]{jiu!{\ene 丩}}
 {\Large 2 [2,1] \textbf{ \begin{tabular}{|c | } \hline
ji\={u}  \\ \hline
\end{tabular} } }\\ \par
\textit{observación: no confundir con el símbolo del alfabeto zhuyin} \textbf{j(i)} \par
verbo\*\  enredar, enrollar, envolver, encuadernar, entrelazar, trenzar, tejer.

\subsubsection { \ene 丫}\label{丫} %%%%%% cl
\index[esquinas]{$8020_{7}$!{\ene 丫}}\index[fon]{ya!{\ene 丫}}
 {\Large 3 [2,2]\textbf{ \begin{tabular}{|c | } \hline
y\auno  \\ \hline\end{tabular} } }\\ \par
\textit{observación: no confundir con el 25\textdegree\ símbolo del alfabeto zhuyin, correspondiente a la vocal} \textbf{\guillemotleft A\guillemotright} \textit{en la transcripción alfabética china y latina. }\par
sustantivo\ uno\ cuernito; ramificación, horcadura (POREJEMPLO\ \textit{árbol}); {\ene 丫桂} mesa con superficie bifurcada; {\ene 丫戾} sobresalir por un lado (cuernito); %%%%% 戾 hsik
dos\ manojo, haz, trenzas (\textit{peinado de niños, niñas}).

\subsubsection{\ene  个}\label{个}
\index[esquinas]{$8020_{0}$!{\ene 个}}
\index[fon]{ge!{\ene 个}}
{\Large 3 [2,2]\textbf{ \begin{tabular}{|c c | } \hline
gè & g\={e}  \\ \hline
\end{tabular} } }\\ \par
sustantivo\ uno\ \abreviacion\ \enlugarde\ {\ene 個} \vease\ \textnumero\ ref{個}.; dos\ \*\ ala de reposo, pabellón; {\ene 青陽右个} pabellón derecho de la sala Qinian.
\entry{个字}{gè'zì}{}{pintura\ símbolo ge (\textit{trazo para representación de hojas, POREJEMPLO\ de bambú}).}
 

\subsubsection{\ene 㐄}\label{㐄}% 㐄 jv2
\index[esquinas]{$5070_{0}$!{\ene 㐄}}\index[fon]{kua!{\ene 㐄}}
{\Large 3 [2, 2] \textbf{ \begin{tabular}{|c c | } \hline
ku\atres & kuà  \\ \hline\end{tabular} } }\\ \par
verbo\ dar pasos cortos; pasos cortos.

\subsubsection{\ene 丯}\label{丯} %%%% 丯 qj2
\index[esquinas]{$5000_{0}$!{\ene 丯}}\index[fon]{jie!{\ene 丯}}
{\Large 4 [2, 3] \textbf{ \begin{tabular}{|c | } \hline
jiè  \\ \hline\end{tabular} } }\\ \par
adjetivo\ embrollado, enmarañado, desordenado (\textit{sobre el pasto crecido}).


\subsubsection{\ene 丰}\label{丰} 
\index[esquinas]{$5000_{0}$!{\ene 丰}}\index[fon]{feng!{\ene 丰}}
{\Large 4 [2, 3] \begin{tabular}{|c | } \hline
f\euno ng  \\ \hline\end{tabular} }\\ \par
I. adjetivo\ uno\ bello, impecable, bien visto, atractivo (\textit{de exteriores}).

\subsubsection{\ene 中}\label{中} %4 [2, 3] 
\index[esquinas]{$5000_{6}$!{\ene 中}}
\index[fon]{zhong!{\ene 中}}
{\Large 4 [2, 3] \textbf{ \begin{tabular}{|c c | c |} \hline
zh\={o}ng & zhòng  & -zh\r{o}ng \\ \hline
\end{tabular} } }\\ \par
I. \textbf{zh\={o}ng} sustantivo\ uno\ centro, mitad;

%\subsubsection{\ene 韦} 
\subsubsection{  \ene 韦}\label{韦} 
\index[esquinas]{$5002_{7}$!{\ene 韦}}\index[fon]{wei!{\ene 韦}} 
4 [2, 3] \abreviacion\ \enlugarde\ {\ene 韋} \vease\ \textnumero\ ref.
%\vease\ 

\subsubsection{\ene 书}\label{书} %%% 书 ids
\index[esquinas]{$5302_{7}$!{\ene 书}}\index[fon]{shu!{\ene 书}} 
 {\Large 4 [2, 3]} \abreviacion\ \enlugarde\ {\ene 書} %% lga
 \vease\ \textnumero\ ref.

\subsubsection{\ene 丱}\label{丱}%%%%% vlllm 丱
\index[esquinas]{$2277_{0}$!{\ene 丱}}
\index[fon]{guan!{\ene 丱}} \index[fon]{kuang!{\ene 丱}} 
 {\Large 5 [2, 4] \textbf{ \begin{tabular}{|c | } \hline
guàn  \\ \hline
\end{tabular} } } {\Large \textbf{ \begin{tabular}{|c | } \hline
kuàng,  ku\v{a}ng  \\ \hline\end{tabular} } }\\ \par
sustantivo\ dos moños (en forma de cuerno) en la cabeza de los infantes; infantil; 
{\ene 丱齒} desarrollo infantil, infancia; %%%% ymuoo
\entry{丱童}{guàn-tóng}{}{niño.}%%%%% ytwg
\entry{丱女}{guàn-n\v{ü}}{}{niña.}

\subsubsection{\ene 串}\label{串} 
\index[esquinas]{$5000_{6}$!{\ene 串}}
\index[fon]{chuan!{\ene 串}} \index[fon]{guan!{\ene 串}} 
 {\Large 7 [2, 6] \textbf{ \begin{tabular}{| c | c |} \hline
 chuàn  & guàn \\ \hline\end{tabular} } }\\ \par
I. \textbf{chuàn} verbo\ uno enhebrar, ensartar; atravesar (pasar) de lado a lado;
{\ene 串錢} coser monedas en el estambre; {\ene 貫串} a) pasar a través; b) figurado\ superar, aprender de cabo a rabo; dos\ visitar, hacer una visita (a alguien).\\
\entry{串星}{liánx\={\i}ng'}{}{astronomia\ estrella doble.}
\entry{串聯}{chuànlián'}{}{uno\  electronica\ circuito en serie; dos\ acuerdo inmediato.} %%%% wjbuc 貫

\subsubsection{\ene 丳}\label{丳} %%% llww 丳
\index[esquinas]{$5500_{6}$!{\ene 丳}}\index[fon]{chan!{\ene 丳}} 
{\Large 8 [2, 7] \textbf{ \begin{tabular}{|c | } \hline
chàn  \\ \hline\end{tabular} } }\\ \par
sustantivo\ asador; {\ene 以⺼貫丳} poner la carne en el asador (brocheta).

\subsubsection{\ene 临}\label{临}%%%% lloa 临
\index[esquinas]{$2806_{3}$!{\ene 临}}\index[fon]{lin!{\ene 临}} 
  {\Large 9 [2, 8]} \abreviacion\ \enlugarde\ {\ene 臨} %% slorr 臨
 \vease\ \textnumero\ ref.

\subsubsection{ \ene 丵}\label{丵} %%%%%% tctj 丵
\index[esquinas]{$3240_{1}$!{\ene 丵}}\index[fon]{zhuo!{\ene 丵}}  
 {\Large 3 [2,2] \textbf{ \begin{tabular}{|c | } \hline
zhuó  \\ \hline\end{tabular} } }\\ \par
 sustantivo\ pasto espeso; maleza.
%\subsubsection{ff}sdf
\setcounter{subsss}{\value{subsubsection}} 

\subsection{\ene 丶}\label{tress}

\setcounter{subsubsection}{\arabic{subsss}} %\input{1trazo/dos} 
\subsubsection{\ene   丶} \label{丶} %% i2
\index[esquinas]{$3000_{0}$!{\ene 丶}} \index[fon]{zhu!{\ene 丶}} 
{\Large 1 [3, 0] \textbf{ \begin{tabular}{|c | } \hline
zh\v{u}  \\ \hline \end{tabular} } }\\ \par 
%\sustantivo\ \uno\ \*\ 
punto; %\dos\ \*\ llama luminosa.

\subsubsection{\ene   丸}\label{丸} %%% 丸 kni
\index[esquinas]{$5001_{7}$!{\ene 丸}} \index[fon]{wan!{\ene 丸}} 
{\Large 3 [3, 2] \textbf{ \begin{tabular}{|c | } \hline
wán  \\ \hline \end{tabular} } }\\ \par 
%I. \sustantivo/\CLASIFICADOR\ \uno\ bola; pelota; bala, núcleo;

\subsubsection{\ene    丹} \label{丹}%%% 丹 by
\index[esquinas]{$7744_{0}$!{\ene 丹}} \index[fon]{dan!{\ene 丹}} 
 {\Large 4 [3, 3] \textbf{ \begin{tabular}{|c | } \hline
d\={a}n  \\ \hline \end{tabular} } }\\ \par 
I.% \sustantivo\ \uno\
  cinabrio; bermellón; tinte rojo; rojo {\ene 丹色} color rojo; %\dos\ píldora; panacea; \daoismo\ piedra filosofal, píldora de la inmortalidad.

\subsubsection{\ene   为}\label{为} %%%% iksi 为 bhnf 爲 iknf 為
\index[esquinas]{$3402_{7}$!{\ene 为}}
\index[fon]{wei!{\ene 为}} 
 {\Large 4 [3, 3]} \abreviacion\ \enlugarde\ {\ene 爲} \vease\ \textnumero\ ref. 

\subsubsection{\ene   主} \label{主}% 主 yg
\index[esquinas]{$0010_{4}$!{\ene 主}} \index[fon]{zhu!{\ene 主}} 
{\Large 5 [3, 4] \textbf{ \begin{tabular}{|c | } \hline
zh\v{u}  \\ \hline \end{tabular} } }\\ \par
I.% \sustantivo\ \uno\
 dueño; patrón; propietario;

\subsubsection{\ene     乓}  \label{乓} %%% omi 乓
\index[esquinas]{$7203_{1}$!{\ene 乓}} \index[fon]{pang!{\ene 乓}} 
{\Large 6 [3, 4] \textbf{ \begin{tabular}{|c | } \hline
p\={a}ng  \\ \hline \end{tabular} } }\\ \par
\textit{solamente en compuestos;} \vease\ {\ene 乒乓} %%% omh

\subsubsection{\ene    举}{\label{举}}%fcq2 举 9 [3, 8] %%%% 舉 hcq
\index[esquinas]{$9050_{8}$!{\ene 挙}} \index[fon]{ju!{\ene 挙}} 
 {\Large 9 [3, 8]} \abreviacion\ \enlugarde\ {\ene 舉} \vease\ \textnumero\ ref. 
\setcounter{subsss}{\value{subsubsection}} 

\subsection{\ene 丿 乁  乀} \label{cuatros} 
\setcounter{subsubsection}{\arabic{subsss}} %\input{1trazo/dos} 

\subsubsection{\ene 丿}\label{丿}%  1 [4, 0] %%%% qfbk 撇
\index[esquinas]{$2000_{0}$!{\ene 丿}}\index[fon]{pie!{\ene 丿}} 
 {\Large 4 [3, 3]\textbf{ \begin{tabular}{|c | } \hline
pi\v{e}  \\ \hline\end{tabular} } }\\ \par 
\enlugarde\ {\ene 撇} \textit{trazo caligráfico plegado de derecha a izquierda}

\subsubsection{\ene ㇏}\label{㇏} %%%%1 [4, 0]
\index[esquinas]{$3000_{0}$!{\ene ㇏}}\index[fon]{yi!{\ene ㇏}} 
 {\Large 1 [4, 0] \textbf{ \begin{tabular}{|c | } \hline
yí \\ \hline\end{tabular} } }\\ \par 
\textit{observación: no confundir con el símbolo} {\ene 乀} \textit{del alfabeto zhuyin.}\\
verbo\ \*\ correr. 

%\subsubsection{\esetc  𠂆} % xh 𠂆
%\index[esquinas]{$7220_{0}$!{\esetc 𠂆}}\index[fon]{yi!{\esetc 𠂆}} 
%{\Large 2 [4, 1] \textbf{ \begin{tabular}{|c | } \hline yì \\ \hline\end{tabular} } }\\ \par 
%I\ verbo\ tirar; arrastrar.\par
%II\ adjetivo\ diáfano.\par
%III\ sustantivo\ \textit{trazo} {\esetc 𠂆} (\guillemotleft yì\guillemotright en caligrafía)

\subsubsection{\ene  𠂇}\label{𠂇} %% 𠂇 xk3
\index[esquinas]{$4000_{0}$!{\ene 𠂇}}\index[fon]{zuo!{\ene 𠂇}} 
 {\Large 2 [4, 1]} \enlugarde\ {\ene 左} \vease\ \textnumero\ ref.
 
\subsubsection{\ene 乃}\label{乃} %%% 乃 2[4, 1] nhs 乃
\index[esquinas]{$1722_{7}$!{\ene 乃}}
\index[fon]{nai!{\ene 乃}} \index[fon]{ai!{\ene 乃}} 
 {\Large 2 [4, 1] \textbf{ \begin{tabular}{|c | } \hline
n\v{a}i \\ \hline\end{tabular} } }tablaencompuestostambien\ {\Large \textbf{\begin{tabular}{|c | } \hline
\atres i \\ \hline\end{tabular} } }\\ \par 
 \textit{cópula en oraciones con predicado nominal, frecuentemente con énfasis (insistencia):} precisamente, exactamente;
dos\ \textit{conjunción adverbial frente al predicado en oraciones subordinadas en el lenguaje literario, a menudo con énfasis y significado principal} entonces. \textit{el siguiente contexto merece especial atención:}
a) \textit{después de una oración temporal suplementaria:} entonces, y entonces, sólo cuando, siempre y cuando. b) \textit{después de oraciones suplementarias condicionales:} entonces, en tal caso. c) \
\entry{乃組}{n\v{a} i-z'\v{u}}{}{uno\ tu tío; tu padre; dos\ tu ancestro, tu clan (género). }
\entry{乃是}{n\v{a}i' shì}{}{precisamente; siendo; sea.}
\entry{乃父}{n\v{a}i-fù}{}{tu padre; yo (el padre de uno).}
\entry{乃公}{n\v{a}i-g\={o}ng}{}{uno\ tu padre; dos\ yo, mío.}


\subsubsection{\ene 㐅}\label{㐅} %%%% k5 㐅
\index[esquinas]{$4000_{0}$!{\ene 㐅}}\index[fon]{wu!{\ene 㐅}} 
{\Large 2 [4, 1]}  \enlugarde\ {\ene 五} \vease\ \textnumero\ ref.

\subsubsection{\ene  乂}\label{乂} %%%% 2 [4, 1]
\index[esquinas]{$4000_{0}$!{\ene 乂}}\index[fon]{yi!{\ene 乂}}
\index[fon]{si!{\ene 乂}} \index[fon]{ai!{\ene 乂}} 
{\Large 2 [4, 1] \textbf{ \begin{tabular}{| c | c | c |} \hline
yì & ài  & sì \\ \hline\end{tabular} } }\\ \par
verbo\ uno\ \textbf{yì} regularizar; dirigir, enderezar, poner en orden; 
dos\ \textbf{yì} \enlugarde\ {\ene 刈} (\textit{segar, cortar});
I\ \textbf{sì} numeral\ cifra 4 (\textit{trazo comercial abreviado}).
\entry{乂粟}{yì-sù}{}{dialectal\ maíz, mazorca} %%%粟 mwfd
%\Large{  \def\stackalignment{r}%
%\topinset{ \begin{turn}{-4}\ene 乀\end{turn}}{\ene   乂}{}{2.9pt}} 2 [4, 1]
\subsubsection{\ene  乡}\label{乡} %%%% 乡 vvh %%%%鄉 vhi3
\index[esquinas]{$2020_{2}$!{\ene 乡}}\index[fon]{xiang!{\ene 乡}} 
 {\Large 3 [4, 2]} \abreviacion\ \enlugarde\ {\ene 鄉} \vease\ \textnumero\ ref.
 
 \subsubsection{\ene  久}\label{久} %%%% no 久 %%%%鄉 vhi3
 \index[esquinas]{$2780_{0}$!{\ene 久}}\index[fon]{jiu!{\ene 久}}
 {\Large 3 [4, 2]  \textbf{ \begin{tabular}{|c | } \hline
ji\utres \\ \hline\end{tabular} } }\\ \par 
 I. djetivo/adverbio\ uno\ extenso, largo; continuación, prolongación 
 
 \subsubsection{\ene  乇}\label{乇} %%%%  %%%%hp 乇
 \index[esquinas]{$2071_{4}$!{\ene 乇}}\index[fon]{zhe!{\ene 乇}}
{\Large 3 [4, 2] \textbf{ \begin{tabular}{|c | } \hline
zhè \\ \hline\end{tabular} } }\\ \par
sustantivo\ hojas [de pasto] vegetales; crecimiento [de pasto].
 
  \subsubsection{\ene  么}\label{么} %%%% 么 hi2
 \index[esquinas]{$2073_{2}$!{\ene 么 }}
\index[fon]{yao!{\ene 么}}\index[fon]{ma!{\ene 么}}
{\Large 3 [4, 2]} 
\begin{enumerate}[label=\Roman*]
  \item
 \abreviacion\ \enlugarde\ {\ene 幺} \vease\ \textnumero\ ref.
\item
\abreviacion\ \enlugarde\ {\ene 麽} \vease\ \textnumero\ ref.
 \end{enumerate}
 
 \subsubsection{\ene 币}\label{币} %%%% hlb 币 %%%% slb 匝
 \index[esquinas]{$2022_{7}$!{\ene 币}}\index[fon]{za!{\ene 币}}
 {\Large 4 [4, 3], [50, 1]} \abreviacion\ \enlugarde\ {\ene 匝} \vease\ \textnumero\ ref.
 
\subsubsection{\ene 长}\label{长} %%%% lhmo 长 %%%% smv 長
\index[esquinas]{$4273_{0}$!{\ene 长}}
  \index[fon]{chang!{\ene 长}}\index[fon]{zhang!{\ene 长}}
 {\Large 4 [4, 3], [168, 0]} \abreviacion\ \enlugarde\ {\ene 長} \vease\ \textnumero\ ref.
 
 \subsubsection{\ene 之}\label{之} %%%% lhmo 长 %%%% smv 長
 \index[esquinas]{$3030_{2}$!{\ene 之}}
\index[fon]{zhi!{\ene 之}}
{\Large 4 [4, 3] \textbf{ \begin{tabular}{| c  | c |} \hline
zh\iuno & -zh\icero   \\ \hline\end{tabular} } }\\ \par
I. \textbf{-zh\r{\i}} pauxiliar\\
uno\ \textit{palabra auxiliar del lenguaje literario, usada para separar un determinante previo del siguiente; forma palabras compuestas atributivas;}
{\ene 魯迅之文章} obra de Lu Xìn. %%%% 魯 nwfa

\subsubsection{\ene 玍}\label{玍} %
 \index[esquinas]{$2110_{4}$!{\ene 玍}}\index[fon]{ga!{\ene 玍}}   
{\Large 5 [4, 4] \textbf{ \begin{tabular}{| c |} \hline
g\v{a} \\ \hline\end{tabular} } }\\ \par
I. adjetivo\ uno\ dialectal\ reacio; pendenciero; intransigente; {\ene 他脾氣很玍} el es todo un terco; dos\ ávido; tacaño, avaro.\par
II. verbo\ dialectal\ uno\ cortar, rebanar.\par
\entry{玍古}{g\v{a}'g\={u}}{}{uno\ extraño, anormal; raramente; terco, obstinado.}
 

\subsubsection{\ene 乍}\label{乍} %%%% hs2 乍 %%%% sm
  \index[esquinas]{$8021_{1}$!{\ene 乍}}
\index[fon]{zha!{\ene 乍}}\index[fon]{zuo!{\ene 乍}}
{\Large 5 [4, 4] \textbf{ \begin{tabular}{| c | c |} \hline
zhà & zuò  \\ \hline\end{tabular} } }\\ \par
I. \textbf{zhà} adverbio\ uno\ súbito, de pronto; casualmente, fortuitamente;
\textit{en duplicación} en caso\ldots, en caso \ldots; {\ene 富乍} enriquecimiento fortuito; {\ene 天氣乍冷熱} el clima de pronto se enfría, de pronto se sofoca; dos\ primero, sólo si; {\ene 乍到} sólo si llega [al lugar]; {\ene 看很好} a primera vista parece muy bueno; tres\ preferentemente, mejor. \par
II. verbo\ uno\ \textbf{zhà} poner (colocar) en puntas; poner en guardia, poner en alerta, {\ene 毛乍} poner los pelos de punta (POREJEMPLO\ \textit{de miedo}). \par
III. nombrepropio\ Zhà (familia)


\subsubsection{\ene 乎}\label{乎} %%%% hfd 乎 %%%% sm
 \index[esquinas]{$2040_{9}$!{\ene 乎}}\index[fon]{hu!{\ene 乎}}
{\Large 5 [4, 4] \textbf{ \begin{tabular}{| c |} \hline
h\={u}  \\ \hline\end{tabular} } }\\ \par
I. pmodal: uno\ \textit{partícula interrogativa del lenguaje literario, colocada al final de la oración, sin pertenecer a la palabra interrogativa} comparar\ si \textit{de la} ; a) \textit{en oraciones interrogativas comunes}; 
{\ene 子見夫子乎}? ¿si has visto al maestro (\textit{Confucio})?; b) en preguntas retóricas;  {\ene 可以人而不如乎鳥} ¿siendo hombre se puede estar peor que un ave?

 \subsubsection{\ene 鸟} %%%% 鸟 hvsm
 \index[esquinas]{$2712_{7}$!{\ene 鸟}}
\index[fon]{niao!{\ene 鸟}}\index[fon]{diao!{\ene 鸟}}
 {\Large 5 [4, 4], [196, 0]} \abreviacion\ \enlugarde\ {\ene 鳥} \vease\ \textnumero\ ref.
 
\subsubsection{\ene 乏}\label{乏} %%%% 乏 hino
 \index[esquinas]{$2030_{2}$!{\ene 乏}}\index[fon]{fa!{\ene 乏}}
{\Large 5 [4, 4] \textbf{ \begin{tabular}{| c |} \hline
fá  \\ \hline\end{tabular} } }\\ \par 
I. verbo\ uno\ no bastar, hacer falta.\par
II. sustantivo\ uno\ insuficiencia, escasez, carencia, deficit; necesidad, pobre; vacante.\par
III. adjetivo\ uno\ escaso, insuficiente; dos\ impropio; incompleto.

 \subsubsection{\ene 后}\label{后} %%%% hmr 后
\index[esquinas]{$7226_{1}$!{\ene 后}}\index[fon]{hou!{\ene 后}}
 {\Large 6 [4, 5], [30, 3]} \abreviacion\ \enlugarde\ {\ene 從} \vease\ \textnumero\ ref.
 \subsubsection*{\ene 后}%\label{后} %%%% hmr 后
\index[esquinas]{$7226_{1}$!{\ene 后}}\index[fon]{hou!{\ene 后}}
 {\Large 6 [4, 5], [30, 3]} \abreviacion\ \enlugarde\ {\ene 從} \vease\ \textnumero\ ref.

 \subsubsection{\ene 𠂤}\label{𠂤} %%%% hrlr 𨸏
  \index[esquinas]{$2777_{7}$!{\ene 𠂤}}
\index[fon]{dui!{\ene 𠂤}}\index[fon]{zui!{\ene 𠂤}}
 {\Large 6 [4, 5]} \abreviacion\ \enlugarde\ {\ene 堆} \vease\ \textnumero\ ref.
 
 \subsubsection{\ene 乒}\label{乒} %%%% omh 乒 %%%% sm
 \index[esquinas]{$7220_{1}$!{\ene 乒}}\index[fon]{ping!{\ene 乒}}
{\Large 6 [4, 5] \textbf{ \begin{tabular}{| c |} \hline
p\={\i}ng  \\ \hline\end{tabular} } }\\ \par 
onomatopeyico\ ¡pam! ¡bang! \textit{onomatopeya de golpe, crujido.}
\entry{乒乓}{p\={\i} ngpang'}{}{uno\ \textit{partícula onomatopéyica del choque de guijarros,} POREJEMPLO\ \textit{granizo en el tejado}; dos\ \abreviacion\ ping-pong, tenis de mesa.}
\entry{乒乓球}{p\={\i} ng nangqiú'}{}{uno\ pelota de ping-pong; dos\ tenis de mesa.}

 \subsubsection{\ene 㐆}\label{㐆} %%%% hsms 㐆 %%%% sm
\index[esquinas]{$2022_{7}$!{\ene 㐆}}\index[fon]{yin!{\ene 㐆}}
{\Large 6 [4, 5] \textbf{ \begin{tabular}{| c |} \hline
y\={\i} n  \\ \hline\end{tabular} } }\\ \par 
verbo\*\ volverse, voltearse.

 \subsubsection{\ene 丟}\label{丟} %%%% omh 乒 %%%% sm
   \index[esquinas]{$2073_{2}$!{\ene 丟}}\index[fon]{diu!{\ene 丟}}
{\Large 6 [4, 5] \textbf{ \begin{tabular}{| c |} \hline
di\={u} \\ \hline\end{tabular} } }\\ \par 
verbo\ uno\  perder, extraviar; dos\ echar, arrojar, lanzar; tres\ abandonar, dejar.

\subsubsection{\ene 乖}\label{乖} %%%% hjlp 乖 %%%% sm
\index[esquinas]{$2011_{2}$!{\ene 乖}}\index[fon]{guai!{\ene 乖}}
{\Large 8 [4, 7] \textbf{ \begin{tabular}{| c |} \hline
gu\={a}i  \\ \hline
\end{tabular} } }\\ \par 
I. adjetivo\ uno\ obediente; {\ene 㧡} %%%% qyvo 㧡
razonable; dos\ ingenioso, hábil, ágil.\par
II. verbo\ uno\ negar; ir en contra (ser contrario a); perturbar; retroceder, retirarse (de algo); renunciar; apartarse.

 \subsubsection{\ene 乘}\label{乘} %%%% hdlp 乘 %%%% sm
  \index[esquinas]{$2090_{1}$!{\ene 乘}}
\index[fon]{cheng!{\ene 乘}}\index[fon]{sheng!{\ene 乘}}
{\Large 9 [4, 8] \textbf{ \begin{tabular}{| c | c |} \hline
chéng & shèng  \\ \hline\end{tabular} } }\\ \par 
I. \textbf{chéng} verbo\ uno\ sentarse adentro de\ldots, en\ldots (POREJEMPLO\ \textit{en el vagón, en el bote}); cargar en\ldots (adentro de); ir (navegar) en\ldots; dos\ servirse (POREJEMPLO\ \textit{ de una oportunidad}), utilizar, usar.
%4
\setcounter{subsss}{\value{subsubsection}} 

\subsection{\ene 乙}\label{cincos} 
\setcounter{subsubsection}{\arabic{subsss}} %\input{1trazo/dos} 
\subsubsection{\ene ㄥ}\label{ㄥ} %%% 
 \index[esquinas]{$2071_{0}$!{\ene ㄥ}}\index[fon]{gong!{\ene ㄥ}}
{\Large 1 [5, 0] \*} \enlugarde {\ene 肱} %%%% bki2 %% yyyyk ㄥ  
\vease\ \textnumero\ ref.

\subsubsection{\ene 乚}\label{乚} %% xu
 \index[esquinas]{$2071_{0}$!{\ene 乚}}\index[fon]{yin!{\ene 乚}}
{\Large 1 [5, 0] \*} \enlugarde {\ene 隱} %%%% nlbmp 隱 %% yyyyk ㄥ  
\vease\ \textnumero\ ref.

\subsubsection{\ene 乙}\label{乙} %nu
 \index[esquinas]{$1771_{0}$!{\ene 乙}}\index[fon]{yi!{\ene 乙}}
 {\Large 1 [5, 0]  \textbf{ \begin{tabular}{|c | } \hline %\large
y\={\i}  \\ \hline\end{tabular} } }\\ \par
I. CONVENCION\ uno\ y (\textit{segundo símbolo del ciclo decimal; asociado con el sector SEE de la cúpula celeste, elemento {\ene 木} árbol en la medicina china con el hígado}); dos\ \textit{segundo punto de conteo:}
II, 2), B, b), $\beta$; {\ene 乙細胞} casilla B. \par
II.sustantivo\*\  golondrina. \par %ласточка
 III. nombrepropio\ antiguo\ Yi (familia).
 
\subsubsection{\ene 乜}\label{乜} %pn
 \index[esquinas]{$4771_{0}$!{\ene 乜}}\index[fon]{nie!{\ene 乜}}
 {\Large 2 [5, 1] \textbf{ \begin{tabular}{|c | } \hline
niè \\ \hline
\end{tabular} } } {\Large \textbf{
\begin{tabular}{|c | } \hline
mi\euno \\ \hline\end{tabular} } }\\ \par 
nombrepropio\ Ne (familia).

\subsubsection{\ene 九} \label{九}%kn
 \index[esquinas]{$4001_{6}$!{\ene 九}}\index[fon]{jiu!{\ene 九}}
{\Large 2 [5, 1] \textbf{ \begin{tabular}{| c | c |} \hline
ji\utres & ji\uuno   \\ \hline\end{tabular} } }\\ \par
I. \textbf{ji\v{u}} numeral/adjetivo/adverbio\ uno\ nueve, 9; {\ene 九天} noveno día; {\ene 九十九} 99.

\subsubsection{\ene 也}\label{也} 
\index[esquinas]{$4471_{2}$!{\ene 也}}
\index[fon]{ye!{\ene 也}}
 {\Large 3 [5, 2] \textbf{  \begin{tabular}{|c | } \hline %\large
y\etres  \\ \hline\end{tabular} } }\\ \par
I. \textit{conjunción adverbial}\par
uno\ también; tal; y (\textit{con negación:} y además); {\ene 他去, 我也去}
[si] el va, yo también; {\ene 他去, 我也不去} [incluso si] el va, yo además no voy.\par
dos\ y$\ldots$; y$\ldots$, y$\ldots$; (\textit{relaciona algunos predicados con la misma partícula verbal, habitualmente repetida ante cada uno de ellos}); {\ene 他也看書, 也看報} %%%% gjsle
el lee libros y revistas.\par
II. nombrepropio\par
uno\ (\abreviacion\ \enlugarde\ {\ene 也門}) Yemen; yemenita.\par
dos\ Ye (familia).

\subsubsection{\ene 飞}\label{飞} %no3 % 飛 nohto
 \index[esquinas]{$1201_{3}$!{\ene 飞}}\index[fon]{fei!{\ene 飞}}
{\Large 3 [5, 2]} \abreviacion\ \enlugarde\  {\ene 飛} \vease\ \textnumero\ ref{}


\subsubsection{\ene 乞}\label{乞} %on 
 \index[esquinas]{$8071_{7}$!{\ene 乞}}\index[fon]{qi!{\ene 乞}}
{\Large 3 [5, 2] \textbf{ \begin{tabular}{| c | c |} \hline
q\itres & qì   \\ \hline\end{tabular} } }\\ \par
I. verbo\ uno\ \textbf{q\v{\i}} pedir, solicitar; implorar, suplicar.\par
dos\ \textbf{q\v{\i}} pedir limosna, limosnear; {\ene 行乞} vagar y limosnear. \par
II. sustantivo\ uno\ \textbf{q\v{\i}} petición, ruego.

\subsubsection{\ene 乣}\label{乣} % %viu 
 \index[esquinas]{$2271_{0}$!{\ene 乣}}\index[fon]{diu!{\ene 乣}}
 {\Large 4 [5, 3] \textbf{ \begin{tabular}{|c | } \hline %\large
di\={u}  \\ \hline\end{tabular}} }\\ \par
soloencompuestos.\par %%% bjwj 軍

\subsubsection{\ene 电}\label{电} %lwu
 \index[esquinas]{$5071_{6}$!{\ene 电}}\index[fon]{dian!{\ene 电}}
{\Large 5 [5, 4]} \abreviacion\ \enlugarde\  {\ene 電} \vease\ \textnumero\ ref{} %nxu

\subsubsection{\ene 乩}\label{乩} %xyru 
 \index[esquinas]{$2261_{0}$!{\ene 乩}}\index[fon]{ji!{\ene 乩}}
 {\Large 4 [5, 3] \textbf{  \begin{tabular}{|c | } \hline %\large
j\={\i}  \\ \hline\end{tabular} } }\\ \par
verbo\ antiguo\  adivinar en el plato con arena;
{\ene 扶乩} adivinar mediante preguntas directas a los espíritus (\textit{en el animismo chino}).

\subsubsection{\ene 乱}\label{乱} % hru2
 \index[esquinas]{$2261_{0}$!{\ene 乱}}
\index[fon]{luan!{\ene 乱}}\index[fon]{lan!{\ene 乱}}
{\Large 7 [5, 6]} \abreviacion\ \enlugarde\  {\ene 亂} \vease\ \textnumero\ \ref{亂}

\subsubsection{\ene 龟}\label{龟} % nwu 亀 
 \index[esquinas]{$2771_{6}$!{\ene 龟}}\index[fon]{gui!{\ene 龟}}
\index[fon]{qiu!{\ene 龟}}\index[fon]{jun!{\ene 龟}}
{\Large 7 [5, 6]} \abreviacion\ \enlugarde\  {\ene 龜} \vease\ \textnumero\ ref{} %nxu

\subsubsection{\ene 乹}\label{乹} %jju 乹 xjju
 \index[esquinas]{$4241_{0}$!{\ene 乹}}
\index[fon]{gan!{\ene 乹}}\index[fon]{qian!{\ene 乹}}
{\Large 9 [5, 8]} \abreviacion\ \enlugarde\  {\ene 乾} \vease\ \textnumero\ \ref{乾} %nxu

\subsubsection{\ene 乾}\label{乾} %乾 jjon 
 \index[esquinas]{$4841_{7}$!{\ene 乾}}
\index[fon]{gan!{\ene 乾}}\index[fon]{qian!{\ene 乾}}
{\Large 11 [5, 10] \textbf{  \begin{tabular}{| c | c |} \hline
g\auno n & qián   \\ \hline\end{tabular} } }\\ \par
I. \textbf{g\={a}n} adjetivo/adverbio/ uno\ seco, secado, desecado; 
dos\ sin agua, desértico.

\subsubsection{\ene 亁}\label{亁} %on 亁 jjon2
 \index[esquinas]{$4841_{7}$!{\ene 亁}}
\index[fon]{gan!{\ene 亁}}\index[fon]{qian!{\ene 亁}}
{\Large 12 [5, 11]} \abreviacion\ \enlugarde\  {\ene 乾} \vease\ \textnumero\ \ref{乾} %nxu

\subsubsection{\ene 亂}\label{亂} %bbu 
 \index[esquinas]{$2221_{0}$!{\ene 亂}}
\index[fon]{luan!{\ene 亂}}\index[fon]{lan!{\ene 亂}}
{\Large 11 [5, 10] \textbf{ \begin{tabular}{| c c |} \hline
luàn & làn   \\ \hline\end{tabular} }} \  
{ \begin{tabular}{ c  } %\hline
\textit{en dialecto}   \\ %\hline
\textit{pekinés también}  \\% \hline
\end{tabular} }\ 
 {\Large \textbf{ \begin{tabular}{|c | } \hline %\large
luán \\ \hline\end{tabular} } }\\ \par
I. adjetivo/adverbio\ uno\ enmarañado, enredado, confuso; mezclado; desordenado; dos\ irreflexivo, desconsiderado; absurdo; excesivo, desorbitante, desmedido; tres\ rebelde, traicionero.\par
II. verbo\ enredar, enmarañar; confundirse.\par
III. sustantivo\ uno\ desorden, caos; {\ene 大亂} caos total; dos\ disturbio, discordia.
\entry{亂亡}{luàng-wáng}{}{uno\ perderse en el revuelo; dos\ devastar, arruinar (POREJEMPLO\ \textit{un país}) como resultado de la rebelión.}
\entry{亂忙}{luàngmáng'}{}{agitarse, agitación, alboroto.}
\entry{亂臣}{luàn-chén}{}{uno\* dignatario activo, funcionario estatal capaz; dos\ súbdito sedicioso; funcionario rebelde.}
\entry{亂世}{luànshì'}{}{siglo agitado; tiempos revueltos; tiempos de guerra, época de revueltas}
%\\ \par 
\setcounter{subsss}{\value{subsubsection}} 

\subsection{\ene 亅}\label{seiss} 
\setcounter{subsubsection}{\arabic{subsss}} %\input{1trazo/dos} 
\subsubsection{\ene 亅}\label{亅}%xn n2
\index[esquinas]{$2000_{2}$!{\ene 亅}}
\index[fon]{jue!{\ene 亅}}
{\Large 2 [6, 1] \textbf{ \begin{tabular}{|c | } \hline
jué  \\ \hline\end{tabular} } }\\ \par
CONVENCION\ gancho (\textit{en caligrafía}).

\subsubsection{\ene 了}\label{了}
\index[esquinas]{$1720_{7}$!{\ene 了}}\index[fon]{liao!{\ene 了}} 
\index[fon]{le!{\ene 了}} 
 {\Large 2 [6, 1] \textbf{ \begin{tabular}{|c | c | c | } \hline
li\v{a}o & liào & -l\={e}  \\ \hline\end{tabular} } }\\ \par
\textit{observación: no confundir con el símbolo} {\ene ^^^^31e1 }\textit{ del alfabeto zhuyin}\\
I. verbo\ uno\ \textbf{li\v{a}o} dividir, partir.


\subsubsection{\ene 予}\label{予} %%%%% ninn 予
\index[esquinas]{$1720_{2}$!{\ene 予}}\index[fon]{yuo!{\ene 予}} 
\index[fon]{yu!{\ene 予}} \index[fon]{zhu!{\ene 予}} 
 {\Large 4 [6, 3] \textbf{ \begin{tabular}{|c | c | c | } \hline
y\v{u} o & yú & zh\v{u}  \\ \hline\end{tabular} } }\\ \par
I. \textbf{y\v{u}} verbo\ uno\ dar; anhelar.
\subsubsection{\ene 事}\label{事} %\IN{er!{\enese 二}}
\index[esquinas]{$5000_{7}$!{\ene 事}}\index[fon]{shi!{\ene 事}}
{\Large 8 [6, 7]  \textbf{\begin{tabular}{|c | } \hline %\large
shì  \\ \hline \end{tabular} } } \\ \par
\begin{enumerate} [noitemsep,label=\Roman{enumi}.,ref=\Roman{enumi}, leftmargin=*]
\item \textit{sustantivo/clasificador}  {\begin{enumerate*}[label=\textbf{\arabic*})]
\item hecho, acto; \item incidente, suceso.
\end{enumerate*}}
\item \textit{verbo} {\begin{enumerate*}[label=\textbf{\arabic*})]
\item  hacer, ocuparse (con); servir.
\end{enumerate*}}
\end{enumerate}
\setcounter{subsss}{\value{subsubsection}} 

 \end{multicols}
 \section{2 trazos}\label{2trazos}\setcounter{subsection}{6}
 \begin{multicols}{2}
\input{2trazos/inte19a}%1,23,209
\end{multicols}
\section{3 trazos}\label{3trazos} \setcounter{subsection}{29} \begin{multicols}{2}
\input{3trazos/seccion_3f}
\end{multicols}

\section{4 trazos}\label{4trazos}\setcounter{subsection}{60}
\begin{multicols}{2}
 \subsection{\ene 心 (忄, 㣺)}\label{sesentayunos}
\setcounter{subsubsection}{\arabic{subsss}} 
%%%%% %%%%% %%%%% %%%%% %%%%% %%%%% %%%%% %%%%%
\subsubsection{\ene 心}\label{心} %%% p 841
\index[esquinas]{$3300_{0}$!{\ene 心}} \index[fon]{xin!{\ene 心}}
{\Large 4 [61, 0]  \textbf{\begin{tabular}{|c|} \hline 
x\={\i}n\\ \hline \end{tabular} } } \\% \par
%%%%% %%%%% %%%%% %%%%% %%%%% %%%%% %%%%% %%%%%
\subsubsection{\ene 必}\label{必} %%% ph 901 
\index[esquinas]{$3300_{4}$!{\ene 必}} \index[fon]{bi!{\ene 必}}
{\Large 5 [61, 1]  \textbf{\begin{tabular}{|c|} \hline 
bì\\ \hline \end{tabular} } } \\% \par
%%%%% %%%%% %%%%% %%%%% %%%%% %%%%% %%%%% %%%%%
\subsubsection{\ene 忙}\label{忙} %%% pyv 311
\index[esquinas]{$9001_{0}$!{\ene 忙}} \index[fon]{mang!{\ene 忙}}
{\Large 6 [61, 3]  \textbf{\begin{tabular}{|c|} \hline 
máng\\ \hline \end{tabular} } } \\% \par
%%%%% %%%%% %%%%% %%%%% %%%%% %%%%% %%%%% %%%%%
\subsubsection{\ene 怀}\label{怀} %%% pmf 600
\index[esquinas]{$9109_{0}$!{\ene 怀}} \index[fon]{huai!{\ene 怀}} 
{\Large 7 [61, 4]} %% pywv
\abreviacion\ \enlugarde\ {\ene  懷} \vease\ \textnumero\ \ref{懷}.%pywv 
%%%%% %%%%% %%%%% %%%%% %%%%% %%%%% %%%%% %%%%%

%%%%% %%%%% %%%%% %%%%% %%%%% %%%%% %%%%% %%%%%
%%%%% %%%%% %%%%% %%%%% %%%%% %%%%% %%%%% %%%%%
\subsubsection{\ene 快}\label{快} %%% pdk 688
\index[esquinas]{$9508_{0}$!{\ene 快}} \index[fon]{xin!{\ene 快}}
{\Large 7 [61, 4]  \textbf{\begin{tabular}{|c|} \hline 
kuài \\ \hline \end{tabular} } } \\% \par
%%%%% %%%%% %%%%% %%%%% %%%%% %%%%% %%%%% %%%%%
\subsubsection{\ene 念}\label{念} %%% oinp 878
\index[esquinas]{$8033_{2}$!{\ene 念}} \index[fon]{nian!{\ene 念}}
{\Large 8 [61, 4]  \textbf{\begin{tabular}{|c|} \hline 
niàn\\ \hline \end{tabular} } } \\% \par
%%%%% %%%%% %%%%% %%%%% %%%%% %%%%% %%%%% %%%%%
\subsubsection{\ene 态}\label{态} %%% kip 886 
%\index[esquinas]{$3300_{0}$!{\ene 态}}
\index[fon]{tai!{\ene 态}}
{\Large 8 [61, 4]} %%% ipp 894
\abreviacion\ \enlugarde\ {\ene 態} \vease\ \textnumero\ \ref{態}. 
%%%%% %%%%% %%%%% %%%%% %%%%% %%%%% %%%%% %%%%%
\subsubsection{\ene 性}\label{性} %%% phqm 192
\index[esquinas]{$9501_{0}$!{\ene 性}} \index[fon]{xing!{\ene 性}}
{\Large 8 [61, 5]  \textbf{\begin{tabular}{|c|} \hline 
xìng\\ \hline \end{tabular} } } \\% \par
%%%%% %%%%% %%%%% %%%%% %%%%% %%%%% %%%%% %%%%%
\subsubsection{\ene 怕}\label{怕} %%%% 怕 pha 603
\index[esquinas]{$9600_{2}$!{\ene 怕}} \index[fon]{pa!{\ene 怕}}
{\Large 8 [61, 5]  \textbf{\begin{tabular}{| c|} \hline 
pà\\ \hline \end{tabular} } } %\\ \par
\begin{enumerate} [noitemsep,label=\Roman{enumi}.,ref=\Roman{enumi}, leftmargin=*]
\item \textit{verbo} temer.
\item  \textit{adverbio} quizá(s), tal vez; es posible que.
\end{enumerate}

\subsubsection{\ene 恶}\label{恶} %%% mcp4 854
\index[esquinas]{$1033_{1}$!{\ene 恶 }} \index[fon]{e!{\ene 恶 }} \index[fon]{wu!{\ene 恶 }} 
{\Large 10 [61, 6]} \abreviacion\ \enlugarde\ {\ene  惡} \vease\ \textnumero\ \ref{惡}.%pywv 
%%%%% %%%%% %%%%% %%%%% %%%%% %%%%% %%%%% %%%%%
\subsubsection{\ene 急}\label{急} %%% nsp 858
\index[esquinas]{$2733_{7}$!{\ene 急}} \index[fon]{ji!{\ene 急}}
{\Large 9 [61, 5]  \textbf{\begin{tabular}{|c|} \hline 
jí \\ \hline \end{tabular} } } \\% \par
%%%%% %%%%% %%%%% %%%%% %%%%% %%%%% %%%%% %%%%%
\subsubsection{\ene 思}\label{思} %%% wp 867
\index[esquinas]{$6033_{0}$!{\ene 思}}
\index[fon]{si!{\ene 思}} \index[fon]{sai!{\ene 思}}
{\Large 9 [61, 5]  \textbf{\begin{tabular}{|c|c|c|} \hline 
s\={\i} & sì & s\={a}i\\ \hline \end{tabular} } } \\% \par
%%%%% %%%%% %%%%% %%%%% %%%%% %%%%% %%%%% %%%%%
\subsubsection{\ene 怎}\label{怎} %%% hsp 875
\index[esquinas]{$8033_{1}$!{\ene 怎}}
\index[fon]{zen!{\ene 怎}}\index[fon]{zem!{\ene 怎}}
{\Large 9 [61, 5]  \textbf{\begin{tabular}{|c|c|c|} \hline 
z\v{e}n & z\v{e}m \\ \hline \end{tabular} } } \\% \par
%%%%% %%%%% %%%%% %%%%% %%%%% %%%%% %%%%% %%%%%
\subsubsection{\ene 恤}\label{恤} %%% phbt 303
\index[esquinas]{$9701_{2}$!{\ene 恤}} \index[fon]{xu!{\ene 恤}}
{\Large 9 [61, 6]  \textbf{\begin{tabular}{|c|} \hline 
xù \\ \hline \end{tabular} } } \\% \par
%%%%% %%%%% %%%%% %%%%% %%%%% %%%%% %%%%% %%%%%
%%%%% %%%%% %%%%% %%%%% %%%%% %%%%% %%%%% %%%%%
\subsubsection{\ene 您}\label{您} %%% ofp 896
\index[esquinas]{$2733_{9}$!{\ene 您}} \index[fon]{nin!{\ene 您}}
{\Large 11 [61, 7]  \textbf{\begin{tabular}{|c|} \hline 
nín\\ \hline \end{tabular} } } \\% \par
%%%%% %%%%% %%%%% %%%%% %%%%% %%%%% %%%%% %%%%%
\subsubsection{\ene 情}\label{情} %%% pq 202
\index[esquinas]{$9502_{7}$!{\ene 情}} \index[fon]{qing!{\ene 情}}
{\Large 11 [61, 8]  \textbf{\begin{tabular}{|c|} \hline 
qíng\\ \hline \end{tabular} } } \\% \par
%%%%% %%%%% %%%%% %%%%% %%%%% %%%%% %%%%% %%%%%
\subsubsection{\ene 惡}\label{惡} %%% ppm 852
\index[esquinas]{$1033_{1}$!{\ene 惡}} \index[fon]{e!{\ene 惡}} 
\index[fon]{wu!{\ene 惡}} 
{\Large 12 [61, 8]  \textbf{\begin{tabular}{|c|c|c|} \hline 
è & wù & w\={u}\\ \hline \end{tabular} } } %\\% \par
\begin{enumerate} [noitemsep,label=\Roman{enumi}.,ref=\Roman{enumi}, leftmargin=*]
\item \textit{sustantivo}  {\begin{enumerate*}[label=\textbf{\arabic*})]
\item  \textbf{è} mal; \item vicio.
\end{enumerate*}}
\item \textit{adjetivo/adverbio} {\begin{enumerate*}[label=\textbf{\arabic*})]
\item  malo, detestable; \item burdo, tosco;
\end{enumerate*}}
\end{enumerate}
%%%%% %%%%% %%%%% %%%%% %%%%% %%%%% %%%%% %%%%%
%\subsubsection{\ene 情}\label{情} %%% pq 202
%\index[esquinas]{$9502_{7}$!{\ene 情}} \index[fon]{qing!{\ene 情}}
%{\Large 11 [61, 8]  \textbf{\begin{tabular}{|c|} \hline 
%qíng\\ \hline \end{tabular} } } \\% \par
%%%%% %%%%% %%%%% %%%%% %%%%% %%%%% %%%%% %%%%%
\subsubsection{\ene 愛}\label{愛} %%% bbpe 22
\index[esquinas]{$2040_{7}$!{\ene 愛}} \index[fon]{ai!{\ene 愛}}
{\Large 13 [61, 9]  \textbf{\begin{tabular}{|c|} \hline 
ài\\ \hline \end{tabular} } } \\% \par
%%%%% %%%%% %%%%% %%%%% %%%%% %%%%% %%%%% %%%%%
\subsubsection{\ene 想}\label{想} %%% dup 865
\index[esquinas]{$4633_{0}$!{\ene 想}} \index[fon]{xiang!{\ene 想}}
{\Large 13 [61, 9]  \textbf{\begin{tabular}{|c|} \hline 
xi\v{a}ng\\ \hline \end{tabular} } } \\% \par
%%%%% %%%%% %%%%% %%%%% %%%%% %%%%% %%%%% %%%%%
\subsubsection{\ene 意}\label{意} %%% ytap 863
\index[esquinas]{$0033_{6}$!{\ene 意}} \index[fon]{yi!{\ene 意}}
{\Large 13 [61, 9]  \textbf{\begin{tabular}{|c|c|} \hline 
yì & y\={\i}\\ \hline \end{tabular} } } \\% \par
%%%%% %%%%% %%%%% %%%%% %%%%% %%%%% %%%%% %%%%%
\subsubsection{\ene 態}\label{態} %%% ipp 894
\index[esquinas]{$2133_{2}$!{\ene 態}} \index[fon]{tai!{\ene 態}}
{\Large 14 [61, 10]  \textbf{\begin{tabular}{|c|} \hline 
tài\\ \hline \end{tabular} } } \\% \par
%%%%% %%%%% %%%%% %%%%% %%%%% %%%%% %%%%% %%%%%
\subsubsection{\ene 愿}\label{愿} %%% mfp 896
\index[esquinas]{$7123_{9}$!{\ene 愿}} \index[fon]{yuan!{\ene 愿}}
{\Large 14 [61, 10]  \textbf{\begin{tabular}{|c|} \hline 
yuàn\\ \hline \end{tabular} } } \\% \par
%%%%% %%%%% %%%%% %%%%% %%%%% %%%%% %%%%% %%%%%
\subsubsection{\ene 慣}\label{慣} %%% pwj 692
\index[esquinas]{$9708_{6}$!{\ene 慣}} \index[fon]{guang!{\ene 慣}}
{\Large 14 [61, 11]  \textbf{\begin{tabular}{|c|} \hline 
guàng\\ \hline \end{tabular} } } \\% \par
%%%%% %%%%% %%%%% %%%%% %%%%% %%%%% %%%%% %%%%%
\subsubsection{\ene 懂}\label{懂} %%% pthg 220
\index[esquinas]{$9401_{5}$!{\ene 懂}} \index[fon]{dong!{\ene 懂}}
{\Large 16 [61, 13]  \textbf{\begin{tabular}{|c|} \hline 
d\v{o}ng\\ \hline \end{tabular} } } \\% \par
%%%%% %%%%% %%%%% %%%%% %%%%% %%%%% %%%%% %%%%%
\subsubsection{\ene 應}\label{應} %%%% 應 igp 860
\index[esquinas]{$0023_{1}$!{\ene 應}} \index[fon]{ying!{\ene 應}}
{\Large 17 [61, 13]  \textbf{\begin{tabular}{| c| c|} \hline 
yìng & y\={\i}ng \\ \hline \end{tabular} } } %\\ \par
\begin{enumerate} [noitemsep,label=\Roman{enumi}.,ref=\Roman{enumi}, leftmargin=*]
\item \textit{verbo}  {\begin{enumerate*}[label=\textbf{\arabic*})]
\item  \textbf{yìng} responder; \item corresponder, acordar. 
\end{enumerate*}}
\item \textbf{y\={\i}ng} \textit{palabra modal}
\item \textbf{yìng} \textit{sustantivo} predvestvie, predznamenovanie.
\end{enumerate}
\entry{應驗}{yìng'yàn}{}{realizarse, cumplirse.}
%%%%% %%%%% %%%%% %%%%% %%%%% %%%%% %%%%% %%%%%
%%%%% %%%%% %%%%% %%%%% %%%%% %%%%% %%%%% %%%%%

%%%%% %%%%% %%%%% %%%%% %%%%% %%%%% %%%%% %%%%%
\subsubsection{\ene 懷}\label{懷} %%% pywv 882
\index[esquinas]{$9403_{2}$!{\ene 懷}} \index[fon]{huai!{\ene 懷}}
{\Large 19 [61, 16]  \textbf{\begin{tabular}{|c|} \hline 
huái \\ \hline \end{tabular} } } %\\% \par
\begin{enumerate} [noitemsep,label=\Roman{enumi}.,ref=\Roman{enumi}, leftmargin=*]
\item \textit{sustantivo}  {\begin{enumerate*}[label=\textbf{\arabic*})]
\item   pecho, seno, entrañas; 
\item alma, corazón, sentimientos.
\end{enumerate*}}
\item \textit{verbo} {\begin{enumerate*}[label=\textbf{\arabic*})]
\item  guardar en el pecho; \item conservar oculto, guardar secreto.
\item pensar; rememorar; soñar con\ldots.
\end{enumerate*}}
\end{enumerate}
%\subsubsection{ff}sdf

\setcounter{subsss}{\value{subsubsection}} dfg
 \subsection{\ene 戈 }\label{sesentaydoss}
\setcounter{subsubsection}{\arabic{subsss}} 
%%%%% %%%%% %%%%% %%%%% %%%%% %%%%% %%%%% %%%%%
\subsubsection{\ene 戈}\label{戈} %%% i 207
\index[esquinas]{$5300_{0}$!{\ene 戈}} \index[fon]{ge!{\ene 戈}}
{\Large 4 [62, 0]  \textbf{\begin{tabular}{|c|} \hline 
g\={e}\\ \hline \end{tabular} } } \\% \par
%%%%% %%%%% %%%%% %%%%% %%%%% %%%%% %%%%% %%%%%
\subsubsection{\ene 我}\label{我} %%% hqi 217
\index[esquinas]{$2355_{0}$!{\ene 我}}
\index[fon]{wo!{\ene 我}}\index[fon]{e!{\ene 我}}
{\Large 7 [62, 3]  \textbf{\begin{tabular}{|c|c|} \hline 
w\v{o} & \v{e}\\ \hline \end{tabular} } } \\% \par
%%%%% %%%%% %%%%% %%%%% %%%%% %%%%% %%%%% %%%%%
\subsubsection{\ene 成}\label{成} %%% ihs 253
\index[esquinas]{$5320_{0}$!{\ene 成}} \index[fon]{cheng!{\ene 成}} 
{\Large 7 (6) [62, 3 (2)] \textbf{\begin{tabular}{| c|} \hline 
chéng\\ \hline \end{tabular} } } %\\ \par
\begin{enumerate} [noitemsep,label=\Roman{enumi}.,ref=\Roman{enumi}, leftmargin=*]
\item \textit{verbo} {\begin{enumerate*}[label=\textbf{\Alph*})]
\item   concluir (exitosamente); completar; \item terminar, llevar a término
\end{enumerate*}}
\item  \textit{adjetivo/adverbio} anual, fuerte, bueno.
\end{enumerate}
%%%%% %%%%% %%%%% %%%%% %%%%% %%%%% %%%%% %%%%%
\setcounter{subsss}{\value{subsubsection}} dfg
 \subsection{\ene 戶 (户)}\label{sesentaytress}
\setcounter{subsubsection}{\arabic{subsss}} 
%%%%% %%%%% %%%%% %%%%% %%%%% %%%%% %%%%% %%%%%
\subsubsection{\ene 户}\label{户} %%% is 382 戶 hs
%\index[esquinas]{$3027_{7}$!{\ene 户}}
\index[fon]{hu!{\ene 户}}
{\Large 4 [63, 0]  \textbf{\begin{tabular}{|c|} \hline 
hù\\ \hline \end{tabular} } } \\% \par
%%%%% %%%%% %%%%% %%%%% %%%%% %%%%% %%%%% %%%%%

%%%%% %%%%% %%%%% %%%%% %%%%% %%%%% %%%%% %%%%% 7 --> 8
\subsubsection{\ene 房}\label{房} %%% hsyhs 421
\index[esquinas]{$3022_{7}$!{\ene 房}} \index[fon]{fang!{\ene 房}}
{\Large 8 [63, 4]  \textbf{\begin{tabular}{|c|} \hline 
fáng\\ \hline \end{tabular} } } \\% \par
%%%%% %%%%% %%%%% %%%%% %%%%% %%%%% %%%%% %%%%%
\subsubsection{\ene 所}\label{所} %%% hshm 733
\index[esquinas]{$7222_{1}$!{\ene 所}} \index[fon]{suo!{\ene 所}}
{\Large 8 [63, 4]  \textbf{\begin{tabular}{|c|} \hline 
su\v{o} \\ \hline \end{tabular} } } %\\% \par
\begin{enumerate} [noitemsep,label=\Roman{enumi}.,ref=\Roman{enumi}, leftmargin=*]
\item \textit{sustantivo/cuantificador}  {\begin{enumerate*}[label=\textbf{\arabic*})]
\item   lugar; locación; ubicación.
\item lado; domicilio.
\end{enumerate*}}
\item \textit{palabra auxiliar} {\begin{enumerate*}[label=\textbf{\arabic*})]
\item  en libros 734
\end{enumerate*}}
\end{enumerate}
\entry{所懷}{su\v{o}-huái}{}{\begin{enumerate*}[label=\textbf{\arabic*})]
\item   sueño arcano; \item persona que entiende las ideas; amigo.
\end{enumerate*}}
\setcounter{subsss}{\value{subsubsection}}
 \subsection{\ene 手(扌)}\label{sesentaycuatros}
\setcounter{subsubsection}{\arabic{subsss}} 
%%%%% %%%%% %%%%% %%%%% %%%%% %%%%% %%%%% %%%%%
\subsubsection{\ene 手}\label{手} %%% q 88 
\index[esquinas]{$2050_{0}$!{\ene 手}} \index[fon]{shou!{\ene 手}}
{\Large 4 [64, 0] \textbf{\begin{tabular}{|c|} \hline 
sh\v{o}u\\ \hline \end{tabular} } } \\% \par
%%%%% %%%%% %%%%% %%%%% %%%%% %%%%% %%%%% %%%%%
\subsubsection{\ene 才}\label{才} %%% dh 76
\index[esquinas]{$4020_{0}$!{\ene 才}} \index[fon]{cai!{\ene 才}}
{\Large 3 [64, 0] \textbf{\begin{tabular}{|c|} \hline 
cái\\ \hline \end{tabular} } } \\% \par
%%%%% %%%%% %%%%% %%%%% %%%%% %%%%% %%%%% %%%%%
\subsubsection{\ene 打}\label{打} %%% 打qmn 1032
\index[esquinas]{$5102_{0}$!{\ene 打}} \index[fon]{da!{\ene 打}}
{\Large 5 [64, 2] \textbf{\begin{tabular}{|c|} \hline 
d\v{a}\\ \hline \end{tabular} } } \\% \par
%%%%% %%%%% %%%%% %%%%% %%%%% %%%%% %%%%% %%%%%
\subsubsection{\ene 托}\label{托} %%% qhp 377
\index[esquinas]{$5201_{4}$!{\ene 托}} \index[fon]{tuo!{\ene 托}}
{\Large 6 [64, 3] \textbf{\begin{tabular}{|c|} \hline 
tu\={o}\\ \hline \end{tabular} } } \\% \par
%%%%% %%%%% %%%%% %%%%% %%%%% %%%%% %%%%% %%%%%
\subsubsection{\ene 扼}\label{扼} %%% qmsu 328
\index[esquinas]{$1234_{0}$!{\ene 扼}} \index[fon]{e!{\ene 扼}}
{\Large 7 [64, 4] \textbf{\begin{tabular}{|c|} \hline 
è\\ \hline \end{tabular} } } \\% \par
%%%%% %%%%% %%%%% %%%%% %%%%% %%%%% %%%%% %%%%%
\subsubsection{\ene 把}\label{把} %%% qau 353
\index[esquinas]{$5701_{7}$!{\ene 把}} \index[fon]{ba!{\ene 把}}
{\Large 7 [64, 4] \textbf{\begin{tabular}{|c|c|} \hline 
b\v{a} & bà\\ \hline \end{tabular} } } \\% \par
%%%%% %%%%% %%%%% %%%%% %%%%% %%%%% %%%%% %%%%%
\subsubsection{\ene 护}\label{护} %%% qis 385
\index[esquinas]{$5300_{7}$!{\ene 护}} \index[fon]{zhi!{\ene 护}}
{\Large 7 [64, 4]} %%% yrtoe 1000
\abreviacion\ \enlugarde\ {\ene 護}\vease\ \textnumero\
ref. {\ene 護}. 
%%%%% %%%%% %%%%% %%%%% %%%%% %%%%% %%%%% %%%%%
\subsubsection{\ene 承}\label{承} %%% nnqo 833
\index[esquinas]{$1723_{2}$!{\ene 承}} \index[fon]{cheng!{\ene 承}}
\index[fon]{zheng!{\ene 承}}\index[fon]{zeng!{\ene 承}}
{\Large 8 [64, 4] \textbf{\begin{tabular}{|c|c|c|} \hline 
chéng & zh\v{e}ng & zèng\\ \hline \end{tabular} } } \\% \par
%%%%% %%%%% %%%%% %%%%% %%%%% %%%%% %%%%% %%%%%
\subsubsection{\ene 拉}\label{拉} %%% qyt 251
\index[esquinas]{$5001_{8}$!{\ene 拉}} \index[fon]{la!{\ene 拉}}
{\Large 8 [64, 5] \textbf{\begin{tabular}{|c|} \hline 
l\={a} \\ \hline \end{tabular} } } \\% \par
%%%%% %%%%% %%%%% %%%%% %%%%% %%%%% %%%%% %%%%%
\subsubsection{\ene 扫}\label{扫} %%% qsm 333
\index[esquinas]{$_{0}$!{\ene 扫}} \index[fon]{sao!{\ene 扫}}
{\Large 6 [64, 5]} %% qsmb 303
\abreviacion\ \enlugarde\ {\ene 掃} \vease\ \textnumero\ \ref{掃}. 
%%%%% %%%%% %%%%% %%%%% %%%%% %%%%% %%%%% %%%%% 
\subsubsection{\ene 括}\label{括} %%% qhjr 494
\index[esquinas]{$5206_{4}$!{\ene 括}}
\index[fon]{gua!{\ene 括}}\index[fon]{kuo!{\ene 括}}
{\Large 9 [64, 6] \textbf{\begin{tabular}{|c|c|} \hline 
gu\={a} & kuò\\ \hline \end{tabular} } } \\% \par
%%%%% %%%%% %%%%% %%%%% %%%%% %%%%% %%%%% %%%%%
\subsubsection{\ene 指}\label{指} %%% qpa 585
\index[esquinas]{$5206_{1}$!{\ene 指}} \index[fon]{zhi!{\ene 指}}
{\Large 9 [64, 6] \textbf{\begin{tabular}{|c|} \hline 
zh\v{\i}\\ \hline \end{tabular} } } \\% \par
%%%%% %%%%% %%%%% %%%%% %%%%% %%%%% %%%%% %%%%%
\subsubsection{\ene 拼}\label{拼} %%% qtt 844
\index[esquinas]{$5204_{1}$!{\ene 拼}} \index[fon]{pin!{\ene 拼}}
{\Large 9 [64, 6] \textbf{\begin{tabular}{|c|} \hline 
p\={\i}n\\ \hline \end{tabular} } } \\% \par
%%%%% %%%%% %%%%% %%%%% %%%%% %%%%% %%%%% %%%%%
\subsubsection{\ene 拿}\label{拿} %%% omrq 91
\index[esquinas]{$8050_{2}$!{\ene 拿}} \index[fon]{na!{\ene 拿}}
{\Large 10 [64, 6] \textbf{\begin{tabular}{|c|} \hline 
ná\\ \hline \end{tabular} } } \\% \par
%%%%% %%%%% %%%%% %%%%% %%%%% %%%%% %%%%% %%%%%
\subsubsection{\ene 按}\label{按} %%% qjv 1034
\index[esquinas]{$5304_{4}$!{\ene 按}} \index[fon]{an!{\ene 按}}
{\Large 9 [64, 6] \textbf{\begin{tabular}{|c|} \hline 
àn\\ \hline \end{tabular} } } \\% \par
%%%%% %%%%% %%%%% %%%%% %%%%% %%%%% %%%%% %%%%%
\subsubsection{\ene 捕}\label{捕} %%% qijb 221
\index[esquinas]{$5302_{7}$!{\ene 捕}} \index[fon]{bu!{\ene 捕}}
{\Large 10 [64, 7] \textbf{\begin{tabular}{|c|c|} \hline 
bù & b\v{u}\\ \hline \end{tabular} } } \\% \par
%%%%% %%%%% %%%%% %%%%% %%%%% %%%%% %%%%% %%%%%
\subsubsection{\ene 挽}\label{挽} %%% qnau 514
\index[esquinas]{$5701_{2}$!{\ene 挽}} \index[fon]{wan!{\ene 挽}}
{\Large 10 [64, 7] \textbf{\begin{tabular}{|c|} \hline 
w\v{a}n\\ \hline \end{tabular} } } \\% \par
%%%%% %%%%% %%%%% %%%%% %%%%% %%%%% %%%%% %%%%%
\subsubsection{\ene 推}\label{推} %%% qog 339
\index[esquinas]{$5001_{5}$!{\ene 推}} \index[fon]{tui!{\ene 推}}
{\Large 11 [64, 8] \textbf{\begin{tabular}{|c|} \hline 
t\={u}i\\ \hline \end{tabular} } } \\% \par
%%%%% %%%%% %%%%% %%%%% %%%%% %%%%% %%%%% %%%%%
\subsubsection{\ene 排}\label{排} %%% qlmy 965
\index[esquinas]{$5101_{1}$!{\ene 排}} \index[fon]{pai!{\ene 排}}
{\Large 11 [64, 8] \textbf{\begin{tabular}{|c|} \hline 
pái\\ \hline \end{tabular} } } \\% \par
%%%%% %%%%% %%%%% %%%%% %%%%% %%%%% %%%%% %%%%%
\subsubsection{\ene 掃}\label{掃} %%% qsmb 303
\index[esquinas]{$5702_{7}$!{\ene 掃}} \index[fon]{sao!{\ene 掃}}
{\Large 11 [64, 8] \textbf{\begin{tabular}{|c|c|} \hline 
s\v{a}o & sào\\ \hline \end{tabular} } } \\% \par
%%%%% %%%%% %%%%% %%%%% %%%%% %%%%% %%%%% %%%%%
\subsubsection{\ene 授}\label{授} %%% qbbe 1015
\index[esquinas]{$5204_{7}$!{\ene 授}} \index[fon]{shou!{\ene 授}}
{\Large 11 [64, 8] \textbf{\begin{tabular}{|c|} \hline 
shòu \\ \hline \end{tabular} } } \\% \par
%%%%% %%%%% %%%%% %%%%% %%%%% %%%%% %%%%% %%%%%
\subsubsection{\ene 接}\label{接} %%% qytv 1015
\index[esquinas]{$5004_{4}$!{\ene 接}} \index[fon]{jie!{\ene 接}}
{\Large 11 [64, 8] \textbf{\begin{tabular}{|c|} \hline 
ji\={e}\\ \hline \end{tabular} } } \\% \par
%%%%% %%%%% %%%%% %%%%% %%%%% %%%%% %%%%% %%%%%
\subsubsection{\ene 採}\label{採} %%% qbd 735
\index[esquinas]{$5209_{4}$!{\ene 採}} \index[fon]{cai!{\ene 採}}
{\Large 11 [64, 8] \textbf{\begin{tabular}{|c|} \hline 
c\v{a}i\\ \hline \end{tabular} } } \\% \par
%%%%% %%%%% %%%%% %%%%% %%%%% %%%%% %%%%% %%%%%
\subsubsection{\ene 描}\label{描} %%% qtw 648
\index[esquinas]{$5406_{0}$!{\ene 描}} \index[fon]{miao!{\ene 描}}
{\Large 12 [64, 9] \textbf{\begin{tabular}{|c|} \hline 
miáo\\ \hline \end{tabular} } } \\% \par
%%%%% %%%%% %%%%% %%%%% %%%%% %%%%% %%%%% %%%%%
\subsubsection{\ene 換}\label{換} %%% qnbk 
\index[esquinas]{$5703_{4}$!{\ene 換}} \index[fon]{huàn!{\ene 換}}
{\Large 12 [64, 9] \textbf{\begin{tabular}{|c|} \hline 
huàn\\ \hline \end{tabular} } } \\% \par
%%%%% %%%%% %%%%% %%%%% %%%%% %%%%% %%%%% %%%%%
\subsubsection{\ene 提}\label{提} %%% qamo 948
\index[esquinas]{$5608_{1}$!{\ene 提}} \index[fon]{ti!{\ene 提}}
{\Large 12 [64, 9] \textbf{\begin{tabular}{|c|} \hline 
tí\\ \hline \end{tabular} } } \\% \par
%%%%% %%%%% %%%%% %%%%% %%%%% %%%%% %%%%% %%%%%
\subsubsection{\ene 摩}\label{摩} %%% idq 96
\index[esquinas]{$0025_{2}$!{\ene 摩}} \index[fon]{mo!{\ene 摩}}
{\Large 15 [64, 11] \textbf{\begin{tabular}{|c|} \hline 
mó\\ \hline \end{tabular} } } \\% \par
%%%%% %%%%% %%%%% %%%%% %%%%% %%%%% %%%%% %%%%%
\subsubsection{\ene 撒}\label{撒} %%% qtbk 1074
\index[esquinas]{$5804_{0}$!{\ene 撒}} \index[fon]{sa!{\ene 撒}}
{\Large 15 [64, 12] \textbf{\begin{tabular}{|c|c|} \hline 
s\={a} & s\v{a}\\ \hline \end{tabular} } } \\% \par
%%%%% %%%%% %%%%% %%%%% %%%%% %%%%% %%%%% %%%%%
\setcounter{subsss}{\value{subsubsection}} 
\subsection{\ene 支 }\label{sesentaycincos}
\setcounter{subsubsection}{\arabic{subsss}} 
%%%%% %%%%% %%%%% %%%%% %%%%% %%%%% %%%%% %%%%%
\subsubsection{\ene 支}\label{支} %%% je 1016
\index[esquinas]{$4040_{7}$!{\ene 支}} \index[fon]{zhi!{\ene 支}}
{\Large 4 [65, 0] \textbf{\begin{tabular}{|c|} \hline 
zh\={\i} \\ \hline \end{tabular} } } \\% \par
%%%%% %%%%% %%%%% %%%%% %%%%% %%%%% %%%%% %%%%%
\setcounter{subsss}{\value{subsubsection}} 
\subsection{\ene 攴 (攵)}\label{sesentayseiss}
\setcounter{subsubsection}{\arabic{subsss}} 
%%%%% %%%%% %%%%% %%%%% %%%%% %%%%% %%%%% %%%%%
\subsubsection{\ene 攴}\label{攴} %%% ye 1014
\index[esquinas]{$2140_{7}$!{\ene 攴}} \index[fon]{pu!{\ene 攴}}
{\Large 4 [66, 0] \textbf{\begin{tabular}{|c|} \hline 
p\={u}\\ \hline \end{tabular} } } \\% \par
%%%%% %%%%% %%%%% %%%%% %%%%% %%%%% %%%%% %%%%%
\subsubsection{\ene 攵}\label{攵} %%% ok 2
%\index[esquinas]{$2140_{7}$!{\ene 攵}}
\index[fon]{pu!{\ene 攵}}
{\Large 4 [66, 0] \textbf{\begin{tabular}{|c|} \hline 
p\={u}\\ \hline \end{tabular} } } \\% \par
%%%%% %%%%% %%%%% %%%%% %%%%% %%%%% %%%%% %%%%%
\subsubsection*{\ene 攵}%\label{攵} %%% ok2
%\index[esquinas]{$1234_{}$!{\ene 攵}}
\index[fon]{wen!{\ene 攵}}
{\Large 4 [66, 0]}
 \enlugarde\ {\ene 文} \vease\ \textnumero\ ref. {\ene 文}. 
%%%%% %%%%% %%%%% %%%%% %%%%% %%%%% %%%%% %%%%%
%%%%% %%%%% %%%%% %%%%% %%%%% %%%%% %%%%% %%%%%
\subsubsection{\ene 改}\label{改} %%% suok 1089
\index[esquinas]{$1874_{0}$!{\ene 改}} \index[fon]{gai!{\ene 改}}
{\Large 7 [66, 3]  \textbf{\begin{tabular}{|c|} \hline 
g\v{a}i\\ \hline \end{tabular} } } \\% \par
%%%%% %%%%% %%%%% %%%%% %%%%% %%%%% %%%%% %%%%%
\subsubsection{\ene 放}\label{放} %%% ysok 1079
\index[esquinas]{$0824_{0}$!{\ene 放}} \index[fon]{fang!{\ene 放}}
{\Large 8 [66, 4]  \textbf{\begin{tabular}{|c|c|} \hline 
fàng & f\v{a}ng\\ \hline \end{tabular} } } \\% \par
%%%%% %%%%% %%%%% %%%%% %%%%% %%%%% %%%%% %%%%%
\subsubsection{\ene 政}\label{政} %%% mmok 1060
\index[esquinas]{$1814_{0}$!{\ene 政}} \index[fon]{zheng!{\ene 政}}
{\Large 9 [66, 5]  \textbf{\begin{tabular}{|c|c|} \hline 
zhèng & zh\={e}ng\\ \hline \end{tabular} } } \\% \par
%%%%% %%%%% %%%%% %%%%% %%%%% %%%%% %%%%% %%%%%
\subsubsection{\ene 故}\label{故} %%% jrok 1061
\index[esquinas]{$4864_{0}$!{\ene 故}} \index[fon]{gu!{\ene 故}}
{\Large 9 [66, 5]  \textbf{\begin{tabular}{|c|} \hline 
gù\\ \hline \end{tabular} } } \\% \par
%%%%% %%%%% %%%%% %%%%% %%%%% %%%%% %%%%% %%%%%
\subsubsection{\ene 教}\label{教} %%% jdok 1069
\index[esquinas]{$4844_{0}$!{\ene 教}} \index[fon]{jiao!{\ene 教}}
{\Large 11 [66, 7]  \textbf{\begin{tabular}{|c|c|} \hline 
jiào & ji\={a}o \\ \hline \end{tabular} } } \\% \par
%%%%% %%%%% %%%%% %%%%% %%%%% %%%%% %%%%% %%%%%
\subsubsection{\ene 救 }\label{救} %%% ieok 1088
\index[esquinas]{$4814_{0}$!{\ene 救}} \index[fon]{e!{\ene 救}} \index[jiu]{wu!{\ene 救}} 
{\Large 12 [61, 8]  \textbf{\begin{tabular}{|c|} \hline 
jiù\\ \hline \end{tabular} } } %\\% \par
\begin{enumerate} [noitemsep,label=\Roman{enumi}.,ref=\Roman{enumi}, leftmargin=*]
 \item \textit{verbo }{\begin{enumerate*}[label=\textbf{\arabic*})]
\item salvar, ayudar, socorrer; \item remediar, asistir;
\end{enumerate*}}
\item \textit{sustantivo} ayuda, asistencia.
\end{enumerate}
%%%%% %%%%% %%%%% %%%%% %%%%% %%%%% %%%%% %%%%%
%%%%% %%%%% %%%%% %%%%% %%%%% %%%%% %%%%% %%%%%
\subsubsection{\ene 敬}\label{敬} %%% trok 1085
\index[esquinas]{$4844_{0}$!{\ene 敬}} \index[fon]{jing!{\ene 敬}}
{\Large 13 [66, 9]  \textbf{\begin{tabular}{|c|} \hline 
jìng\\ \hline \end{tabular} } } \\% \par
%%%%% %%%%% %%%%% %%%%% %%%%% %%%%% %%%%% %%%%%
\subsubsection{\ene 数}\label{数} %%% fvok
\index[esquinas]{$9844_{0}$!{\ene 数}}
\index[fon]{shu!{\ene 数}}\index[fon]{shuo!{\ene 数}}
\index[fon]{cu!{\ene 数}}%\index[fon]{shu!{\ene 数}}
{\Large 13 [66, 9]}
\abreviacion\ \enlugarde\ {\ene 數} \vease\ \textnumero\ \ref{數}. 
%%%%% %%%%% %%%%% %%%%% %%%%% %%%%% %%%%% %%%%%
\subsubsection{\ene 數}\label{數} %%% lvok 1095
\index[esquinas]{$5844_{0}$!{\ene 數}}
\index[fon]{shu!{\ene 数}}\index[fon]{shuo!{\ene 数}}
\index[fon]{cu!{\ene 数}}%
{\Large 15 [66, 11]  \textbf{\begin{tabular}{|c|c|c|} \hline 
sh\v{u} & shù & cù \\ \hline \end{tabular} } } \\% \par
%%%%% %%%%% %%%%% %%%%% %%%%% %%%%% %%%%% %%%%%
\subsubsection{\ene 整}\label{整} %%% dkmym 229
\index[esquinas]{$5810_{1}$!{\ene 整}} \index[fon]{zhi!{\ene 整}}
{\Large 16 [66, 12]  \textbf{\begin{tabular}{|c|c|} \hline 
zhèng & zh\v{e}ng \\ \hline \end{tabular} } } \\% \par
%%%%% %%%%% %%%%% %%%%% %%%%% %%%%% %%%%% %%%%%
\setcounter{subsss}{\value{subsubsection}}
\subsection{\ene 文}\label{sesentaysietes}
\setcounter{subsubsection}{\arabic{subsss}} 
%%%%% %%%%% %%%%% %%%%% %%%%% %%%%% %%%%% %%%%%
\subsubsection{\ene 文}\label{文} %%% yk 58
\index[esquinas]{$0040_{0}$!{\ene 文}} \index[fon]{wen!{\ene 文}}
{\Large 4 [67, 0]  \textbf{\begin{tabular}{|c|c|} \hline 
wén & wèn\\ \hline \end{tabular} } } \\% \par
%%%%% %%%%% %%%%% %%%%% %%%%% %%%%% %%%%% %%%%%
\setcounter{subsss}{\value{subsubsection}}
\subsection{\ene 斗}\label{sesentayochos}
\setcounter{subsubsection}{\arabic{subsss}} 
%%%%% %%%%% %%%%% %%%%% %%%%% %%%%% %%%%% %%%%%
\subsubsection{\ene 斗}\label{斗} %%% yj 斗 936
\index[esquinas]{$3400_{0}$!{\ene 斗}} \index[fon]{dou!{\ene 斗}}
{\Large 4 [68, 0] \textbf{\begin{tabular}{|c|} \hline 
d\v{o}u\\ \hline \end{tabular} } } \\% \par
%%%%% %%%%% %%%%% %%%%% %%%%% %%%%% %%%%% %%%%%
%%%%% %%%%% %%%%% %%%%% %%%%% %%%%% %%%%% %%%%% 7 --> 8
\subsubsection*{\ene 斗}%\label{斗} %%% 
%\index[esquinas]{$1234_{}$!{\ene 斗}}
%\index[fon]{meng!{\ene 斗}}
{\Large 4 [68, 0]} %%%% lnrml, 118, 7712_1 
\abreviacion\ \enlugarde\ {\ene 鬭} \vease\ \textnumero\ ref. {\ene 鬭}. 
%%%%% %%%%% %%%%% %%%%% %%%%% %%%%% %%%%% %%%%% lnrml
\setcounter{subsss}{\value{subsubsection}}
\subsection{\ene 斤 }\label{sesentaynueves}
\setcounter{subsubsection}{\arabic{subsss}} 
%%%%% %%%%% %%%%% %%%%% %%%%% %%%%% %%%%% %%%%%
\subsubsection{\ene 斤}\label{斤} %%% hml 725
\index[esquinas]{$7222_{1}$!{\ene 斤}} \index[fon]{jin!{\ene 斤}}
{\Large 4 [69, 0]  \textbf{\begin{tabular}{|c|} \hline 
j\={\i}n \\ \hline \end{tabular} } } \\% \par
%%%%% %%%%% %%%%% %%%%% %%%%% %%%%% %%%%% %%%%%
\subsubsection{\ene 斯}\label{斯} %%% tchml 738
\index[esquinas]{$4282_{1}$!{\ene 斯}} \index[fon]{si!{\ene 斯}}
{\Large 12 [69, 8]  \textbf{\begin{tabular}{|c|} \hline 
s\={\i} \\ \hline \end{tabular} } } \\% \par
%%%%% %%%%% %%%%% %%%%% %%%%% %%%%% %%%%% %%%%%
\subsubsection{\ene 新}\label{新} %%% ydhml 
\index[esquinas]{$0292_{1}$!{\ene 新}} \index[fon]{xin!{\ene 新}}
{\Large 13 [69, 9]  \textbf{\begin{tabular}{|c|} \hline 
x\={\i}n \\ \hline \end{tabular} } } \\% \par
%%%%% %%%%% %%%%% %%%%% %%%%% %%%%% %%%%% %%%%%
\setcounter{subsss}{\value{subsubsection}} dfg
\subsection{\ene 方 }\label{setentas}
\setcounter{subsubsection}{\arabic{subsss}} 
%%%%% %%%%% %%%%% %%%%% %%%%% %%%%% %%%%% %%%%%
\subsubsection{\ene 方}\label{方} %%% yhs 414
\index[esquinas]{$0022_{7}$!{\ene 方}} \index[fon]{fang!{\ene 方}}
{\Large 4 [70, 0]  \textbf{\begin{tabular}{|c|} \hline 
f\={a}ng\\ \hline \end{tabular} } } \\% \par
%%%%% %%%%% %%%%% %%%%% %%%%% %%%%% %%%%% %%%%%
\subsubsection{\ene 於}\label{於} %%% 1000
\index[esquinas]{$0823_{2}$!{\ene 於}}
\index[fon]{yu!{\ene 於}}\index[fon]{wu!{\ene 於}}
{\Large 8 [70, 4]  \textbf{\begin{tabular}{|c|c|} \hline 
yú & w\={u}\\ \hline \end{tabular} } } \\% \par
%%%%% %%%%% %%%%% %%%%% %%%%% %%%%% %%%%% %%%%%
\subsubsection{\ene 旁}\label{旁} %%% ybyhs
\index[esquinas]{$0022_{7}$!{\ene 旁}}
\index[fon]{pang!{\ene 旁}}\index[fon]{bang!{\ene 旁}}
{\Large 10 [70, 6]  \textbf{\begin{tabular}{|c|c|c|} \hline 
páng & b\={a}ng & bàng\\ \hline \end{tabular} } } \\% \par
%%%%% %%%%% %%%%% %%%%% %%%%% %%%%% %%%%% %%%%%
\setcounter{subsss}{\value{subsubsection}} dfg
\subsection{\ene 无 (旡) }\label{setentayunos}
\setcounter{subsubsection}{\arabic{subsss}} 
%%%%% %%%%% %%%%% %%%%% %%%%% %%%%% %%%%% %%%%%
\subsubsection{\ene 无}\label{无} %%% mku 505
\index[esquinas]{$1041_{2}$!{\ene 无}} \index[fon]{wu!{\ene 无}}
{\Large 4 [71, 0]} %% otf, 8033_1
\abreviacion\ \enlugarde\ {\ene 無} \vease\ \textnumero\ ref. {\ene 無}. 
%%%%% %%%%% %%%%% %%%%% %%%%% %%%%% %%%%% %%%%%
\setcounter{subsss}{\value{subsubsection}} dfg
\subsection{\ene 日}\label{setentaydoss}
\setcounter{subsubsection}{\arabic{subsss}} 
%%%%% %%%%% %%%%% %%%%% %%%%% %%%%% %%%%% %%%%%
\subsubsection{\ene 日}\label{日} %%%  549 a
\index[esquinas]{$6010_{0}$!{\ene 日}} \index[fon]{ri!{\ene 日}}
{\Large 4 [72, 0]  \textbf{\begin{tabular}{|c|} \hline 
rì\\ \hline \end{tabular} } } \\% \par
%%%%% %%%%% %%%%% %%%%% %%%%% %%%%% %%%%% %%%%%
\subsubsection{\ene 早}\label{早} %%% aj 801
\index[esquinas]{$6040_{0}$!{\ene 早}} \index[fon]{zao!{\ene 早}}
{\Large 6 [72, 2]  \textbf{\begin{tabular}{|c|} \hline 
z\v{a}o \\ \hline \end{tabular} } } \\% \par
%%%%% %%%%% %%%%% %%%%% %%%%% %%%%% %%%%% %%%%%
\subsubsection{\ene 时}\label{时} %%% adi2 25
%\index[esquinas]{$6400_{0}$!{\ene 时}}
\index[fon]{shi!{\ene 时}}
{\Large 7 [72, 3]} %% 40 adi
\abreviacion\ \enlugarde\ {\ene 時} \vease\ \textnumero\ \ref{時}. 
%%%%% %%%%% %%%%% %%%%% %%%%% %%%%% %%%%% %%%%%
\subsubsection{\ene 明}\label{明} %%% ab 163
\index[esquinas]{$6702_{0}$!{\ene 明}}\index[fon]{ming!{\ene 明}}
{\Large 8 [72, 4]  \textbf{\begin{tabular}{|c|} \hline 
míng \\ \hline \end{tabular} } } \\% \par
%%%%% %%%%% %%%%% %%%%% %%%%% %%%%% %%%%% %%%%%
\subsubsection{\ene 易}\label{易} %%% aphh 480
\index[esquinas]{$6022_{7}$!{\ene 易}} \index[fon]{yi!{\ene 易}}
{\Large 8 [72, 4]  \textbf{\begin{tabular}{|c|} \hline 
yì\\ \hline \end{tabular} } } \\% \par
%%%%% %%%%% %%%%% %%%%% %%%%% %%%%% %%%%% %%%%%
\subsubsection{\ene 星}\label{星} %%% ah 196
\index[esquinas]{$6010_{5}$!{\ene 星}} \index[fon]{xing!{\ene 星}}
{\Large 9 [72, 5]  \textbf{\begin{tabular}{|c|} \hline 
x\={\i}ng\\ \hline \end{tabular} } } \\% \par
%%%%% %%%%% %%%%% %%%%% %%%%% %%%%% %%%%% %%%%%
\subsubsection{\ene 显}\label{显} %%% 显 atc2 234
\index[esquinas]{$6010_{2}$!{\ene 显}} \index[fon]{xian!{\ene 显 }}
{\Large 9 [72, 5]  } \abreviacion\ \enlugarde\ {\ene 顯  } \vease\ \textnumero\ \ref{顯}. %afmbc
%%%%% %%%%% %%%%% %%%%% %%%%% %%%%% %%%%% %%%%%
\subsubsection{\ene 春}\label{春} %%% qka 577
\index[esquinas]{$5060_{8}$!{\ene 春}} \index[fon]{chun!{\ene 春}}
{\Large 9 [72, 5]  \textbf{\begin{tabular}{|c|} \hline 
ch\={u}n \\ \hline \end{tabular} } } \\% \par
%%%%% %%%%% %%%%% %%%%% %%%%% %%%%% %%%%% %%%%%
\subsubsection{\ene 昨}\label{昨} %%% aos 744
\index[esquinas]{$6801_{1}$!{\ene 昨}} \index[fon]{zuo!{\ene 昨}}
{\Large 9 [72, 5]  \textbf{\begin{tabular}{|c|} \hline 
zuó\\ \hline \end{tabular} } } \\% \par
%%%%% %%%%% %%%%% %%%%% %%%%% %%%%% %%%%% %%%%%
\subsubsection{\ene 是}\label{是} %%% amyo 945
\index[esquinas]{$6080_{1}$!{\ene 是}} \index[fon]{shi!{\ene 是}}
{\Large 9 [72, 5]  \textbf{\begin{tabular}{|c|} \hline 
shì\\ \hline \end{tabular} } } \\% \par
%%%%% %%%%% %%%%% %%%%% %%%%% %%%%% %%%%% %%%%%
\subsubsection{\ene 時}\label{時} %%% adi 40
\index[esquinas]{$6404_{1}$!{\ene 時}} \index[fon]{shi!{\ene 時}}
{\Large 10 [72, 6]  \textbf{\begin{tabular}{|c|} \hline 
shí\\ \hline \end{tabular} } } \\% \par
%%%%% %%%%% %%%%% %%%%% %%%%% %%%%% %%%%% %%%%%
\subsubsection{\ene 晚}\label{晚} %%% anau 512
\index[esquinas]{$6701_{2}$!{\ene 晚}} \index[fon]{wan!{\ene 晚}}
{\Large 11 [72, 7]  \textbf{\begin{tabular}{|c|} \hline 
w\v{a}n \\ \hline \end{tabular} } } \\% \par
%%%%% %%%%% %%%%% %%%%% %%%%% %%%%% %%%%% %%%%%
\subsubsection{\ene 晨}\label{晨} %%% ammv 857
\index[esquinas]{$6023_{2}$!{\ene 晨}} \index[fon]{chen!{\ene 晨}}
{\Large 11 [72, 7]  \textbf{\begin{tabular}{|c|} \hline 
chén\\ \hline \end{tabular} } } \\% \par
%%%%% %%%%% %%%%% %%%%% %%%%% %%%%% %%%%% %%%%%
\subsubsection{\ene 暖}\label{暖} %%% abme 988
\index[esquinas]{$6204_{7}$!{\ene 暖}} \index[fon]{nuan!{\ene 暖}}
{\Large 13 [72, 9]  \textbf{\begin{tabular}{|c|} \hline 
nu\v{a}n \\ \hline \end{tabular} } } \\% \par
%%%%% %%%%% %%%%% %%%%% %%%%% %%%%% %%%%% %%%%%
\subsubsection{\ene 普}\label{普} %%% tca 554
\index[esquinas]{$8060_{1}$!{\ene 普}} \index[fon]{pu!{\ene 普}}
{\Large 12 [72, 8]  \textbf{\begin{tabular}{|c|} \hline 
p\v{u}\\ \hline \end{tabular} } } \\% \par
%%%%% %%%%% %%%%% %%%%% %%%%% %%%%% %%%%% %%%%%
\subsubsection{\ene 暗}\label{暗} %%% ayta 562
\index[esquinas]{$6006_{1}$!{\ene 暗}} \index[fon]{an!{\ene 暗}}
{\Large 13 [72, 9]  \textbf{\begin{tabular}{|c|} \hline 
àn\\ \hline \end{tabular} } } \\% \par
%%%%% %%%%% %%%%% %%%%% %%%%% %%%%% %%%%% %%%%%
\setcounter{subsss}{\value{subsubsection}}
\subsection{\ene 曰 } \label{setentaytress}
\setcounter{subsubsection}{\arabic{subsss}} 
%%%%% %%%%% %%%%% %%%%% %%%%% %%%%% %%%%% %%%%%
\subsubsection{\ene 曰}\label{曰} %%% xa 
\index[esquinas]{$6010_{0}$!{\ene 曰}} \index[fon]{yue!{\ene 曰}}
{\Large 4 [73, 0]  \textbf{\begin{tabular}{|c|} \hline 
yu\={e}\\ \hline \end{tabular} } } \\% \par
%%%%% %%%%% %%%%% %%%%% %%%%% %%%%% %%%%% %%%%%
\subsubsection{\ene 更}\label{更} %%% mlwk 71
\index[esquinas]{$1050_{6}$!{\ene 更}} 
 \index[fon]{geng!{\ene 更}} \index[fon]{jing!{\ene 更}}
{\Large 7 [73, 4]  \textbf{\begin{tabular}{|c|c|c|} \hline 
g\={e}ng & j\={\i}ng & gèng\\ \hline \end{tabular} } } \\% \par
%%%%% %%%%% %%%%% %%%%% %%%%% %%%%% %%%%% %%%%%
\subsubsection{\ene 書}\label{書} %%% lga 552
\index[esquinas]{$5060_{1}$!{\ene 書}} \index[fon]{shu!{\ene 書}}
{\Large 10 [73, 6]  \textbf{\begin{tabular}{|c|} \hline 
sh\={u} \\ \hline \end{tabular} } } \\% \par
%%%%% %%%%% %%%%% %%%%% %%%%% %%%%% %%%%% %%%%%
\subsubsection{\ene 最}\label{最} %%% asje 970
%\index[esquinas]{$_{0}$!{\ene 最}}
\index[fon]{zui!{\ene 最}}
{\Large 12 [73, 8]  \textbf{\begin{tabular}{|c|} \hline 
zùi\\ \hline \end{tabular} } } \\% \par
%%%%% %%%%% %%%%% %%%%% %%%%% %%%%% %%%%% %%%%%
\subsubsection{\ene 會}\label{會} %%% omwa 567
\index[esquinas]{$8060_{6}$!{\ene 會}}
\index[fon]{hui!{\ene 會}}\index[fon]{kuai!{\ene 會}}
{\Large 13 [73, 9]  \textbf{\begin{tabular}{|c|c|c|} \hline 
hùi & h\v{u}i kuài\\ \hline \end{tabular} } } \\% \par
%%%%% %%%%% %%%%% %%%%% %%%%% %%%%% %%%%% %%%%%
\setcounter{subsss}{\value{subsubsection}}
\subsection{\ene 月 (⺝)}\label{setentaycuatros}
\setcounter{subsubsection}{\arabic{subsss}} 
%%%%% %%%%% %%%%% %%%%% %%%%% %%%%% %%%%% %%%%%
\subsubsection{\ene 月}\label{月} %%% 156
\index[esquinas]{$7722_{0}$!{\ene 月}} \index[fon]{yue!{\ene 月}}
{\Large 6 [74, 0]  \textbf{\begin{tabular}{|c|} \hline 
yuè\\ \hline \end{tabular} } } \\% \par
%%%%% %%%%% %%%%% %%%%% %%%%% %%%%% %%%%% %%%%%
\subsubsection{\ene 有}\label{有} %%% kb 174
\index[esquinas]{$4022_{7}$!{\ene 有}} \index[fon]{you!{\ene 有}}
{\Large 6 [74, 2]  \textbf{\begin{tabular}{|c|} \hline 
y\v{o}u\\ \hline \end{tabular} } } \\% \par
%%%%% %%%%% %%%%% %%%%% %%%%% %%%%% %%%%% %%%%%
\subsubsection{\ene 朋}\label{朋} %%% bb 174
\index[esquinas]{$7722_{0}$!{\ene 朋}} \index[fon]{peng!{\ene 朋}}
{\Large 8 [74, 4]  \textbf{\begin{tabular}{|c|} \hline 
péng\\ \hline \end{tabular} } } \\% \par
%%%%% %%%%% %%%%% %%%%% %%%%% %%%%% %%%%% %%%%%
\subsubsection{\ene 服}\label{服} %%% bsle 978
\index[esquinas]{$7724_{7}$!{\ene 服}} \index[fon]{fu!{\ene 服}}
{\Large 8 [74, 4]  \textbf{\begin{tabular}{|c|c|} \hline 
fú & fù\\ \hline \end{tabular} } } \\% \par
%%%%% %%%%% %%%%% %%%%% %%%%% %%%%% %%%%% %%%%%
\subsubsection{\ene 朗}\label{朗} %%% iib 173
\index[esquinas]{$3772_{0}$!{\ene 朗}} \index[fon]{lang!{\ene 朗}}
{\Large 11 [74, 7]  \textbf{\begin{tabular}{|c|} \hline 
l\v{a}ng \\ \hline \end{tabular} } } \\% \par
%%%%% %%%%% %%%%% %%%%% %%%%% %%%%% %%%%% %%%%%
\subsubsection{\ene 期}\label{期} %%% tcb 173
\index[esquinas]{$4782_{0}$!{\ene 期}}
\index[fon]{qi!{\ene 期}}\index[fon]{ji!{\ene 期}}
{\Large 12 [74, 8]  \textbf{\begin{tabular}{|c|c|c|} \hline 
qí & q\={\i} & j\={\i} \\ \hline \end{tabular} } } \\% \par
%%%%% %%%%% %%%%% %%%%% %%%%% %%%%% %%%%% %%%%%
\setcounter{subsss}{\value{subsubsection}}
\subsection{\ene 木 (朩)}\label{setentaycincos}
\setcounter{subsubsection}{\arabic{subsss}} 
%%%%% %%%%% %%%%% %%%%% %%%%% %%%%% %%%%% %%%%%
\subsubsection{\ene 木}\label{木} %%% d 699
\index[esquinas]{$4090_{0}$!{\ene 木}} \index[fon]{mu!{\ene 木}}
{\Large 4 [75, 0] \textbf{\begin{tabular}{|c|} \hline 
mù\\ \hline \end{tabular} } } \\% \par
%%%%% %%%%% %%%%% %%%%% %%%%% %%%%% %%%%% %%%%%
\subsubsection{\ene 本}\label{本} %%% dm 741
\index[esquinas]{$5023_{0}$!{\ene 本}} \index[fon]{ben!{\ene 本}}
{\Large 5 [75, 1] \textbf{\begin{tabular}{|c|} \hline 
b\v{e}n \\ \hline \end{tabular} } } \\% \par
%%%%% %%%%% %%%%% %%%%% %%%%% %%%%% %%%%% %%%%%
\subsubsection{\ene 机}\label{机} %%% dhn 553
\index[esquinas]{$4791_{0}$!{\ene 机}} \index[fon]{ji!{\ene 机}}
{\Large 6 [75, 2]} %% 機 dv
\abreviacion\ \enlugarde\ {\ene 機} \vease\ \textnumero\ \ref{機}. 
%%%%% %%%%% %%%%% %%%%% %%%%% %%%%% %%%%% %%%%%
\subsubsection{\ene 乐}\label{乐} %%% 
\index[esquinas]{$7290_{4}$!{\ene 乐}}\index[fon]{le!{\ene 乐}}
\index[fon]{yue!{\ene 乐}}\index[fon]{yao!{\ene 乐}}
{\Large 5 [75, 2]} %% 樂
\abreviacion\ \enlugarde\ {\ene 樂} \vease\ \textnumero\\ref{樂}. 
%%%%% %%%%% %%%%% %%%%% %%%%% %%%%% %%%%% %%%%%
%%%%% %%%%% %%%%% %%%%% %%%%% %%%%% %%%%% %%%%%
\subsubsection{\ene 东}\label{东} %%% kd 835
\index[esquinas]{$5090_{4}$!{\ene 东}}\index[fon]{dong!{\ene 东}}
{\Large 5 [75, 2]}%% dw
\abreviacion\ \enlugarde\ {\ene 東} \vease\ \textnumero\ \ref{東}. 
%%%%% %%%%% %%%%% %%%%% %%%%% %%%%% %%%%% %%%%%
\subsubsection{\ene 来}\label{来} %%% dt 753
%\index[esquinas]{$1234_{}$!{\ene 来}}
\index[fon]{lai!{\ene 来}}
{\Large 7 [75, 3]}%% doo 4090_8
\abreviacion\ \enlugarde\ {\ene 來} \vease\ \textnumero\ ref. {\ene 來}. 
%%%%% %%%%% %%%%% %%%%% %%%%% %%%%% %%%%% %%%%%
\subsubsection{\ene 析}\label{析} %%% dhml 735 
\index[esquinas]{$4292_{1}$!{\ene 析}} \index[fon]{xi!{\ene 析}}
{\Large 8 [75, 4] \textbf{\begin{tabular}{|c|} \hline 
x\={\i}\\ \hline \end{tabular} } } \\% \par
%%%%% %%%%% %%%%% %%%%% %%%%% %%%%% %%%%% %%%%%
%%%%% %%%%% %%%%% %%%%% %%%%% %%%%% %%%%% %%%%%
\subsubsection{\ene 林}\label{林} %%% dd 703
\index[esquinas]{$4499_{0}$!{\ene 林}} \index[fon]{lin!{\ene 林}}
{\Large 8 [75, 4] \textbf{\begin{tabular}{|c|} \hline 
lín\\ \hline \end{tabular} } } \\% \par
%%%%% %%%%% %%%%% %%%%% %%%%% %%%%% %%%%% %%%%%
\subsubsection{\ene 果}\label{果} %%% wd 762
\index[esquinas]{$6090_{4}$!{\ene 果}} \index[fon]{guo!{\ene 果}}
{\Large 8 [75, 4] \textbf{\begin{tabular}{|c|} \hline 
gu\v{o} \\ \hline \end{tabular} } } \\% \par
%%%%% %%%%% %%%%% %%%%% %%%%% %%%%% %%%%% %%%%%
\subsubsection{\ene 東}\label{東} %%% dw 773 
\index[esquinas]{$5090_{6}$!{\ene 東}} \index[fon]{dong!{\ene 東}}
{\Large 8 [75, 4] \textbf{\begin{tabular}{|c|} \hline 
d\={o}ng \\ \hline \end{tabular} } } \\% \par
%%%%% %%%%% %%%%% %%%%% %%%%% %%%%% %%%%% %%%%%
\subsubsection{\ene 柏}\label{柏} %%% dha 604
\index[esquinas]{$4690_{2}$!{\ene 柏}}
\index[fon]{bo!{\ene 柏}} \index[fon]{bai!{\ene 柏}}
{\Large 9 [75, 5] \textbf{\begin{tabular}{|c|c|} \hline 
bó & bái\\ \hline \end{tabular} } } \\% \par
%%%%% %%%%% %%%%% %%%%% %%%%% %%%%% %%%%% %%%%%
%%%%% %%%%% %%%%% %%%%% %%%%% %%%%% %%%%% %%%%%
\subsubsection{\ene 柠}\label{柠} %%% djmn
%\index[esquinas]{$1234_{0}$!{\ene 柠}}
\index[fon]{ning!{\ene 柠}}
{\Large 9 [75, 5]}% 檸djpn % 4392_1
\abreviacion\ \enlugarde\ {\ene 檸} \vease\ \textnumero\ \ref{檸}.
%%%%% %%%%% %%%%% %%%%% %%%%% %%%%% %%%%% %%%%%
%%%%% %%%%% %%%%% %%%%% %%%%% %%%%% %%%%% %%%%%
\subsubsection{\ene 桥}\label{桥} %%% dhkl
%\index[esquinas]{$1234_{}$!{\ene 桥}}
\index[fon]{qiao!{\ene 桥}}\index[fon]{jiao!{\ene 桥}}
{\Large 10 [75, 6]} %% dhkb % 4292_7
\abreviacion\ \enlugarde\ {\ene 橋} \vease\ \textnumero\ \ref{橋}. 
%%%%% %%%%% %%%%% %%%%% %%%%% %%%%% %%%%% %%%%%
\subsubsection{\ene 样}\label{样} %%% dtq 879
\index[esquinas]{$4895_{1}$!{\ene 样}}
\index[fon]{xiang!{\ene 样}}\index[fon]{yang!{\ene 样}}
{\Large 10 [75, 6]} % dtge % 4899_2
\abreviacion\ \enlugarde\ {\ene 樣} \vease\ \textnumero\ \ref{樣}. 
%%%%% %%%%% %%%%% %%%%% %%%%% %%%%% %%%%% %%%%%
\subsubsection{\ene 桌}\label{桌} %%% yad 718
\index[esquinas]{$2190_{4}$!{\ene 桌}} \index[fon]{zhuo!{\ene 桌}}
{\Large 10 [75, 6] \textbf{\begin{tabular}{|c|} \hline 
zh\={o} \\ \hline \end{tabular} } } \\% \par
%%%%% %%%%% %%%%% %%%%% %%%%% %%%%% %%%%% %%%%%
%%%%% %%%%% %%%%% %%%%% %%%%% %%%%% %%%%% %%%%%
\subsubsection{\ene 校}\label{校} %%% dyck 55
\index[esquinas]{$4094_{8}$!{\ene 校}}
\index[fon]{jiao!{\ene 校}}\index[fon]{xiao!{\ene 校}}
{\Large 10 [75, 6] \textbf{\begin{tabular}{|c|c|} \hline 
jiào & xiào\\ \hline \end{tabular} } } \\% \par
{ \textbf{\begin{tabular}{|c|c|} \hline 
ji\v{a}o & xiáo\\ \hline \end{tabular} } } \\% \par
%%%%% %%%%% %%%%% %%%%% %%%%% %%%%% %%%%% %%%%%
\subsubsection{\ene 桃}\label{桃} %%% 
\index[esquinas]{$4291_{3}$!{\ene 桃}} \index[fon]{tao!{\ene 桃}}
{\Large 10 [75, 6] \textbf{\begin{tabular}{|c|} \hline 
táo\\ \hline \end{tabular} } } \\% \par
%%%%% %%%%% %%%%% %%%%% %%%%% %%%%% %%%%% %%%%%
\subsubsection{\ene 條}\label{條} %%% olod 728
\index[esquinas]{$2829_{4}$!{\ene 條}} \index[fon]{tiao!{\ene 條}}
{\Large 11 [75, 7] \textbf{\begin{tabular}{|c|} \hline 
tiáo\\ \hline \end{tabular} } } \\% \par
%%%%% %%%%% %%%%% %%%%% %%%%% %%%%% %%%%% %%%%%
\subsubsection{\ene 椅}\label{椅} %%% dkmr 1075
\index[esquinas]{$4492_{1}$!{\ene 椅}} \index[fon]{yi!{\ene 椅}}
{\Large 12 [75, 8] \textbf{\begin{tabular}{|c|c|} \hline 
y\v{\i} & y\={\i} \\ \hline \end{tabular} } } \\% \par
%%%%% %%%%% %%%%% %%%%% %%%%% %%%%% %%%%% %%%%%
\subsubsection{\ene 楚}\label{楚} %%% ddnyo 958
\index[esquinas]{$4480_{0}$!{\ene 楚}} \index[fon]{chu!{\ene 楚}}
{\Large 13 [75, 9] \textbf{\begin{tabular}{|c|} \hline 
ch\v{u} \\ \hline \end{tabular} } } \\% \par
%%%%% %%%%% %%%%% %%%%% %%%%% %%%%% %%%%% %%%%%
\subsubsection{\ene 楼}\label{楼} %%% dfdv
%\index[esquinas]{$4994_{4}$!{\ene 楼}}
\index[fon]{lou!{\ene 楼}}
{\Large 13 [75, 9]}%%% dlwv 1024
\abreviacion\ \enlugarde\ {\ene 樓} \vease\ \textnumero\ \ref{樓}. 
%%%%% %%%%% %%%%% %%%%% %%%%% %%%%% %%%%% %%%%%
\subsubsection{\ene 概}\label{概} %%% % daiu 槪 dhpu
\index[esquinas]{$4191_{2}$!{\ene 概}} \index[fon]{gai!{\ene 概}}
{\Large 15 [75, 11] \textbf{\begin{tabular}{|c|} \hline 
gài\\ \hline \end{tabular} } } \\% \par
%%%%% %%%%% %%%%% %%%%% %%%%% %%%%% %%%%% %%%%%
\subsubsection{\ene 樂}\label{樂} %%% vid 737
\index[esquinas]{$2290_{4}$!{\ene 樂}}\index[fon]{le!{\ene 樂}}
\index[fon]{yue!{\ene 樂}}\index[fon]{yao!{\ene 樂}}
{\Large 15 [75, 11] \textbf{\begin{tabular}{|c|c|c|} \hline 
lè & yuè & yào \\ \hline \end{tabular} } } \\% \par
%%%%% %%%%% %%%%% %%%%% %%%%% %%%%% %%%%% %%%%%
\subsubsection{\ene 標}\label{標} %%% dmwf 
\index[esquinas]{$4199_{1}$!{\ene 標}} \index[fon]{biao!{\ene 標}}
{\Large 15 [75, 11] \textbf{\begin{tabular}{|c|} \hline 
bi\={a}o\\ \hline \end{tabular} } } \\% \par
%%%%% %%%%% %%%%% %%%%% %%%%% %%%%% %%%%% %%%%%
\subsubsection{\ene 樱}\label{樱} %%% dbov 1045
%\index[esquinas]{$_{}$!{\ene 樱}}
\index[fon]{ying!{\ene 樱}}
{\Large 15 [75, 11]}%櫻 dbcv
\abreviacion\ \enlugarde\ {\ene 櫻} \vease\ \textnumero\ \ref{櫻}. 
%%%%% %%%%% %%%%% %%%%% %%%%% %%%%% %%%%% %%%%%
\subsubsection{\ene 橋}\label{橋} %%% dhkb 255
\index[esquinas]{$4292_{7}$!{\ene 橋}}
\index[fon]{qiao!{\ene 橋}} \index[fon]{jiao!{\ene 橋}}
{\Large 16 [75, 12] \textbf{\begin{tabular}{|c|c|c|} \hline 
qiáo & jiào & ji\v{a}o \\ \hline \end{tabular} } } \\% \par
%%%%% %%%%% %%%%% %%%%% %%%%% %%%%% %%%%% %%%%
\subsubsection{\ene 橘}\label{橘} %%% dnhb 275
\index[esquinas]{$4792_{7}$!{\ene 橘}} \index[fon]{ju!{\ene 橘}}
{\Large 16 [75, 12] \textbf{\begin{tabular}{|c|} \hline 
jù\\ \hline \end{tabular} } } \\% \par
%%%%% %%%%% %%%%% %%%%% %%%%% %%%%% %%%%% %%%%%
\subsubsection{\ene 機}\label{機} %%% dv 264
\index[esquinas]{$4295_{3}$!{\ene 機}} \index[fon]{ji!{\ene 機}}
{\Large 16 [75, 12] \textbf{\begin{tabular}{|c|} \hline 
j\={\i}\\ \hline \end{tabular} } } \\% \par
%%%%% %%%%% %%%%% %%%%% %%%%% %%%%% %%%%% %%%%%
\subsubsection{\ene 樓}\label{樓} %%% dlwv 1024
\index[esquinas]{$4594_{4}$!{\ene 樓}} \index[fon]{lou!{\ene 樓}}
{\Large 15 [75, 11] \textbf{\begin{tabular}{|c|} \hline 
lóu \\ \hline \end{tabular} } } \\% \par
%%%%% %%%%% %%%%% %%%%% %%%%% %%%%% %%%%% %%%%%
\subsubsection{\ene 樣}\label{樣} %%% dtge 818
\index[esquinas]{$4899_{2}$!{\ene 樣}}
\index[fon]{yang!{\ene 樣}}\index[fon]{xiang!{\ene 樣}}
{\Large 15 [75, 11] \textbf{\begin{tabular}{|c|c|} \hline 
yàng & xiàng\\ \hline \end{tabular} } } \\% \par
%%%%% %%%%% %%%%% %%%%% %%%%% %%%%% %%%%% %%%%%
\subsubsection{\ene 檸}\label{檸} %%% djpn 1043
\index[esquinas]{$4392_{1}$!{\ene 檸}} \index[fon]{ning!{\ene 檸}}
{\Large 18 [75, 14] \textbf{\begin{tabular}{|c|} \hline 
níng\\ \hline \end{tabular} } } \\% \par
%%%%% %%%%% %%%%% %%%%% %%%%% %%%%% %%%%% %%%%%
\subsubsection{\ene 櫻}\label{櫻} %%% dtbo 838
\index[esquinas]{$4694_{4}$!{\ene 櫻}} \index[fon]{ying!{\ene 櫻}}
{\Large 18 [75, 14] \textbf{\begin{tabular}{|c|} \hline 
méng\\ \hline \end{tabular} } } \\% \par
%%%%% %%%%% %%%%% %%%%% %%%%% %%%%% %%%%% %%%%%
\subsubsection{\ene 檬}\label{檬} %%% dtbo 838
\index[esquinas]{$4493_{2}$!{\ene 檬}} \index[fon]{meng!{\ene 檬}}
{\Large 18 [75, 14] \textbf{\begin{tabular}{|c|} \hline 
y\={\i}ng\\ \hline \end{tabular} } } \\% \par
%%%%% %%%%% %%%%% %%%%% %%%%% %%%%% %%%%% %%%%%
\setcounter{subsss}{\value{subsubsection}}
\subsection{\ene 欠}\label{setentayseiss}
\setcounter{subsubsection}{\arabic{subsss}} 
%%%%% %%%%% %%%%% %%%%% %%%%% %%%%% %%%%% %%%%%
\subsubsection{\ene 欠}\label{欠} %%% no2 592
\index[esquinas]{$1234_{0}$!{\ene 欠}} \index[fon]{qian!{\ene 欠}}
{\Large 4 [76, 0] \textbf{\begin{tabular}{|c|} \hline 
qiàn \\ \hline \end{tabular} } } \\% \par
%%%%% %%%%% %%%%% %%%%% %%%%% %%%%% %%%%% %%%%%

%%%%% %%%%% %%%%% %%%%% %%%%% %%%%% %%%%% %%%%%
%%%%% %%%%% %%%%% %%%%% %%%%% %%%%% %%%%% %%%%%
\subsubsection{\ene 欢}\label{欢} %%% eno 606
\index[esquinas]{$7748_{2}$!{\ene 欢}} \index[fon]{huan!{\ene 欢}}
{\Large 6 [76, 2]} % 歡 tgno
\abreviacion\ \enlugarde\ {\ene 歡} \vease\ \textnumero\ \ref{歡}.       
%%%%% %%%%% %%%%% %%%%% %%%%% %%%%% %%%%% %%%%%
\subsubsection{\ene 歌}\label{歌} %%% mrno 
\index[esquinas]{$1768_{2}$!{\ene 歌}} \index[fon]{ge!{\ene 歌}}
{\Large 14 [76, 10] \textbf{\begin{tabular}{|c|} \hline 
g\={e} \\ \hline \end{tabular} } } \\% \par
%%%%% %%%%% %%%%% %%%%% %%%%% %%%%% %%%%% %%%%%
\subsubsection{\ene 歡}\label{歡} %%% tgno 596 
\index[esquinas]{$4728_{2}$!{\ene 歡}} \index[fon]{huan!{\ene 歡}}
{\Large 22 [76, 18] \textbf{\begin{tabular}{|c|} \hline 
hu\={a}n \\ \hline \end{tabular} } } \\% \par
%%%%% %%%%% %%%%% %%%%% %%%%% %%%%% %%%%% %%%%%
\setcounter{subsss}{\value{subsubsection}} dfg
\subsection{\ene 止}\label{setentaysietes}
\setcounter{subsubsection}{\arabic{subsss}} 
%%%%% %%%%% %%%%% %%%%% %%%%% %%%%% %%%%% %%%%%
\subsubsection{\ene 止}\label{止} %%% ylm
\index[esquinas]{$2110_{0}$!{\ene 止}} \index[fon]{zhi!{\ene 止}}
{\Large 4 [77, 0] \textbf{\begin{tabular}{|c|} \hline 
zh\v{\i} \\ \hline \end{tabular} } } \\% \par
%%%%% %%%%% %%%%% %%%%% %%%%% %%%%% %%%%% %%%%%
\subsubsection{\ene 此}\label{此} %%% ymp 292
\index[esquinas]{$2210_{0}$!{\ene 此}} \index[fon]{ci!{\ene 此}}
{\Large 6 [77, 2] \textbf{\begin{tabular}{|c|} \hline 
c\v{\i} \\ \hline \end{tabular} } } \\% \par
%%%%% %%%%% %%%%% %%%%% %%%%% %%%%% %%%%% %%%%%
\subsubsection{\ene 歲}\label{歲} %%% ymihh 251 
\index[esquinas]{$2125_{3}$!{\ene 歲}} \index[fon]{sui!{\ene 歲}}
{\Large 13 [77, 9] \textbf{\begin{tabular}{|c|} \hline 
sùi\\ \hline \end{tabular} } } \\% \par
%%%%% %%%%% %%%%% %%%%% %%%%% %%%%% %%%%% %%%%%
\setcounter{subsss}{\value{subsubsection}}
\subsection{\ene 歹 (歺)}\label{setentayochos}
\setcounter{subsubsection}{\arabic{subsss}} 
%%%%% %%%%% %%%%% %%%%% %%%%% %%%%% %%%%% %%%%%
\subsubsection{\ene 歹}\label{歹} %%% mni 395
\index[esquinas]{$1020_{7}$!{\ene 歹}}
\index[fon]{dai!{\ene 歹}}\index[fon]{e!{\ene 歹}}
{\Large 4 [78, 0] \textbf{\begin{tabular}{|c|c|} \hline 
d\v{a}i & è \\ \hline \end{tabular} } } \\% \par
%%%%% %%%%% %%%%% %%%%% %%%%% %%%%% %%%%% %%%%%
\setcounter{subsss}{\value{subsubsection}}
\subsection{\ene 殳}\label{setentaynueves}
\setcounter{subsubsection}{\arabic{subsss}} 
%%%%% %%%%% %%%%% %%%%% %%%%% %%%%% %%%%% %%%%%
\subsubsection{\ene 殳}\label{殳} %%% hne 1036
\index[esquinas]{$7740_{7}$!{\ene 殳}} \index[fon]{shu!{\ene 殳}}
{\Large 4 [79, 0] \textbf{\begin{tabular}{|c|} \hline 
sh\={u} \\ \hline \end{tabular} } } \\% \par
%%%%% %%%%% %%%%% %%%%% %%%%% %%%%% %%%%% %%%%%
\subsubsection{\ene 段}\label{段} %%% hjhne 1039
\index[esquinas]{$7744_{7}$!{\ene 段}} \index[fon]{duan!{\ene 段}}
{\Large 9 [79, 5] \textbf{\begin{tabular}{|c|} \hline 
duàn \\ \hline \end{tabular} } } \\% \par
%%%%% %%%%% %%%%% %%%%% %%%%% %%%%% %%%%% %%%%%
\setcounter{subsss}{\value{subsubsection}}
\subsection{\ene 毌 (母)}\label{ochentas}
\setcounter{subsubsection}{\arabic{subsss}} 
%%%%% %%%%% %%%%% %%%%% %%%%% %%%%% %%%%% %%%%%
\subsubsection{\ene 毋}\label{毋} %%% wyi
\index[esquinas]{$7750_{0}$!{\ene 毋}} \index[fon]{guan!{\ene 毋}}
{\Large 4 [80, 0] \textbf{\begin{tabular}{|c|} \hline 
gu\={a}n \\ \hline \end{tabular} } } \\% \par
%%%%% %%%%% %%%%% %%%%% %%%%% %%%%% %%%%% %%%%%
\subsubsection{\ene 母}\label{母} %%% wyi 576
\index[esquinas]{$7775_{0}$!{\ene 母}} \index[fon]{mu!{\ene 母}}
{\Large 5 [80, 1]  \textbf{\begin{tabular}{|c|} \hline 
m\v{u}\\ \hline \end{tabular} } } \\% \par
%%%%% %%%%% %%%%% %%%%% %%%%% %%%%% %%%%% %%%%%
\subsubsection{\ene 每}\label{每} %%% owyi
\index[esquinas]{$8075_{7}$!{\ene 每}} \index[fon]{mei!{\ene 每}}
{\Large 7 [80, 3] \textbf{\begin{tabular}{|c|} \hline 
m\v{e}i \\ \hline \end{tabular} } } \\% \par
%%%%% %%%%% %%%%% %%%%% %%%%% %%%%% %%%%% %%%%%
\setcounter{subsss}{\value{subsubsection}}
\subsection{\ene 比}\label{ochentayunos}
\setcounter{subsubsection}{\arabic{subsss}} 
%%%%% %%%%% %%%%% %%%%% %%%%% %%%%% %%%%% %%%%%
\subsubsection{\ene 比}\label{比} %%% pp 279
\index[esquinas]{$2271_{0}$!{\ene 比}} \index[fon]{bi!{\ene 比}}
{\Large 4 [81, 0] \textbf{\begin{tabular}{|c|c|} \hline 
b\v{\i} & bì \\ \hline \end{tabular} } } \\% \par
%%%%% %%%%% %%%%% %%%%% %%%%% %%%%% %%%%% %%%%%
\setcounter{subsss}{\value{subsubsection}}
\subsection{\ene 毛}\label{ochentaydoss}
\setcounter{subsubsection}{\arabic{subsss}} 
%%%%% %%%%% %%%%% %%%%% %%%%% %%%%% %%%%% %%%%%
\subsubsection{\ene 毛}\label{毛} %%% hqu 379
\index[esquinas]{$2071_{5}$!{\ene 毛}}
\index[fon]{mao!{\ene 毛}}\index[fon]{mu!{\ene 毛}}
{\Large 4 [82, 0] \textbf{\begin{tabular}{|c|c|} \hline 
máo & mú \\ \hline \end{tabular} } } \\% \par
%%%%% %%%%% %%%%% %%%%% %%%%% %%%%% %%%%% %%%%%
\setcounter{subsss}{\value{subsubsection}}
\subsection{\ene 氏}\label{ochentaytress}
\setcounter{subsubsection}{\arabic{subsss}} 
%%%%% %%%%% %%%%% %%%%% %%%%% %%%%% %%%%% %%%%%
\subsubsection{\ene 氏}\label{氏} %%% hvp 188
\index[esquinas]{$7274_{0}$!{\ene 氏}}
\index[fon]{zhi!{\ene 氏}}\index[fon]{shi!{\ene 氏}}
{\Large 4 [83, 0] \textbf{\begin{tabular}{|c|c|} \hline 
shì & zhì\\ \hline \end{tabular} } } \\% \par
%%%%% %%%%% %%%%% %%%%% %%%%% %%%%% %%%%% %%%%%
\setcounter{subsss}{\value{subsubsection}} 
\subsection{\ene 气}\label{ochentaycuatross} 
\setcounter{subsubsection}{\arabic{subsss}} 
%%%%% %%%%% %%%%% %%%%% %%%%% %%%%% %%%%% %%%%%
\subsubsection{\ene 气}\label{气} %%% omn2 530
\index[esquinas]{$8001_{7}$!{\ene 气}} \index[fon]{qi!{\ene 气 }}
{\Large 4 [84, 0]}
\abreviacion\ \enlugarde\ {\ene 氣} \vease\ \textnumero\ \ref{氣}. 
%%%%% %%%%% %%%%% %%%%% %%%%% %%%%% %%%%% %%%%%
\subsubsection{\ene 氣}\label{氣} %%% onfd 532
\index[esquinas]{$8091_{7}$!{\ene 氣}} \index[fon]{qi!{\ene 氣}}
{\Large 10 [84, 6] \textbf{\begin{tabular}{|c|} \hline 
qì\\ \hline \end{tabular} } } \\% \par
%%%%% %%%%% %%%%% %%%%% %%%%% %%%%% %%%%% %%%%%
\setcounter{subsss}{\value{subsubsection}}
\subsection{\ene 水 (氵, 氺)}\label{ochentaycincos}
\setcounter{subsubsection}{\arabic{subsss}} 
%%%%% %%%%% %%%%% %%%%% %%%%% %%%%% %%%%% %%%%%
\subsubsection{\ene 水}\label{水} %%% e 804
\index[esquinas]{$1290_{0}$!{\ene 水}} \index[fon]{shui!{\ene 水}}
{\Large 4 [85, 0] \textbf{\begin{tabular}{|c|} \hline 
sh\v{u}i \\ \hline \end{tabular} } } \\% \par
%%%%% %%%%% %%%%% %%%%% %%%%% %%%%% %%%%% %%%%%
\subsubsection{\ene 汉}\label{汉} %%% ee3 
\index[esquinas]{$3714_{0}$!{\ene 汉}} \index[fon]{han!{\ene 汉}}
{\Large 5 [85, 2]}%% 漢 etl2 
\abreviacion\ \enlugarde\ {\ene 漢} \vease\ \textnumero\ \ref{漢}. 
%%%%% %%%%% %%%%% %%%%% %%%%% %%%%% %%%%% %%%%%
\subsubsection{\ene 江}\label{江} %%% em 68
\index[esquinas]{$3111_{2}$!{\ene 江}} \index[fon]{jiang!{\ene 江}}
{\Large 6 [85, 3] \textbf{\begin{tabular}{|c|} \hline 
ji\={a}ng\\ \hline \end{tabular} } } \\% \par
%%%%% %%%%% %%%%% %%%%% %%%%% %%%%% %%%%% %%%%%
\subsubsection{\ene 沒}\label{沒} %%% ez 1033
\index[esquinas]{$3714_{7}$!{\ene 沒}} \index[fon]{mei!{\ene 沒}}
{\Large 7 [85, 4] \textbf{\begin{tabular}{|c|} \hline 
méi\\ \hline \end{tabular} } } \\% \par
%%%%% %%%%% %%%%% %%%%% %%%%% %%%%% %%%%% %%%%%
\subsubsection{\ene 沙}\label{沙} %%% efh 362
\index[esquinas]{$3912_{0}$!{\ene 沙}} \index[fon]{sha!{\ene 沙}}
{\Large 7 [85, 4] \textbf{\begin{tabular}{|c|} \hline 
sh\={a} \\ \hline \end{tabular} } } \\% \par
%%%%% %%%%% %%%%% %%%%% %%%%% %%%%% %%%%% %%%%%
\subsubsection{\ene 汽}\label{汽} %%% eomn
\index[esquinas]{$3811_{7}$!{\ene 汽}} \index[fon]{qi!{\ene 汽}}
{\Large 7 [85, 4] \textbf{\begin{tabular}{|c|} \hline 
qì\\ \hline \end{tabular} } } \\% \par
%%%%% %%%%% %%%%% %%%%% %%%%% %%%%% %%%%% %%%%%
\subsubsection{\ene 注}\label{注} %%% eyg
\index[esquinas]{$3011_{4}$!{\ene 注}}
\index[fon]{zhu!{\ene 注}}\index[fon]{zhou!{\ene 注}}
{\Large 8 [85, 5] \textbf{\begin{tabular}{|c|c|} \hline 
zhù & zhòu \\ \hline \end{tabular} } } \\% \par
%%%%% %%%%% %%%%% %%%%% %%%%% %%%%% %%%%% %%%%%
\subsubsection{\ene 況}\label{況} %%% erhu
\index[esquinas]{$3611_{2}$!{\ene 況}} \index[fon]{kuang!{\ene 況}}
{\Large 8 [85, 5] \textbf{\begin{tabular}{|c|} \hline 
kuàng\\ \hline \end{tabular} } } \\% \par
%%%%% %%%%% %%%%% %%%%% %%%%% %%%%% %%%%% %%%%%
\subsubsection{\ene 泉}\label{泉} %%% hae 814
\index[esquinas]{$2690_{2}$!{\ene 泉}} \index[fon]{quan!{\ene 泉}}
{\Large 9 [85, 5] \textbf{\begin{tabular}{|c|} \hline 
quán\\ \hline \end{tabular} } } \\% \par
%%%%% %%%%% %%%%% %%%%% %%%%% %%%%% %%%%% %%%%%
\subsubsection{\ene 波}\label{波} %%% edhe 1031
\index[esquinas]{$3414_{7}$!{\ene 波}}
\index[fon]{bo!{\ene 波}}\index[fon]{po!{\ene 波}}
{\Large 8 [85, 5] \textbf{\begin{tabular}{|c|c|} \hline 
b\={o} & p\={o} \\ \hline \end{tabular} } } \\% \par
%%%%% %%%%% %%%%% %%%%% %%%%% %%%%% %%%%% %%%%%

\subsubsection{\ene 法}\label{法} %%% egi 952
\index[esquinas]{$3413_{2}$!{\ene 法}} \index[fon]{fa!{\ene 法}}
{\Large 8 [85, 5] \textbf{\begin{tabular}{|c|c|c|} \hline 
f\={a} & fá & f\v{a}\\ \hline \end{tabular} } } \\% \par
%%%%% %%%%% %%%%% %%%%% %%%%% %%%%% %%%%% %%%%%
\subsubsection{\ene 活}\label{活} %%% ehjr 
\index[esquinas]{$3216_{4}$!{\ene 活}} \index[fon]{huo!{\ene 活}}
{\Large 9 [85, 6] \textbf{\begin{tabular}{|c|} \hline 
huó\\ \hline \end{tabular} } } \\% \par
%%%%% %%%%% %%%%% %%%%% %%%%% %%%%% %%%%% %%%%%
\subsubsection{\ene 浅}\label{浅} %%% eij 268
\index[esquinas]{$3315_{0}$!{\ene 浅}} \index[fon]{qian!{\ene 浅}}
{\Large 8 [85, 6]}
\abreviacion\ \enlugarde\ {\ene 淺} \vease\ \textnumero\ \ref{淺}. 

%%%%% %%%%% %%%%% %%%%% %%%%% %%%%% %%%%% %%%%%
\subsubsection{\ene 洗}\label{洗} %%% ehgu 435
\index[esquinas]{$3411_{2}$!{\ene 洗}}
\index[fon]{xi!{\ene 洗}}\index[fon]{xian!{\ene 洗}}
{\Large 9 [85, 6] \textbf{\begin{tabular}{|c|c|} \hline 
x\v{\i} & xi\v{a}n\\ \hline \end{tabular} } } \\% \par
%%%%% %%%%% %%%%% %%%%% %%%%% %%%%% %%%%% %%%%%
\subsubsection{\ene 海}\label{海} %%% eow 580
\index[esquinas]{$3815_{7}$!{\ene 海}} \index[fon]{hai!{\ene 海}}
{\Large 10 [85, 7] \textbf{\begin{tabular}{|c|} \hline 
h\v{a}i \\ \hline \end{tabular} } } \\% \par
%%%%% %%%%% %%%%% %%%%% %%%%% %%%%% %%%%% %%%%%
\subsubsection{\ene 消}\label{消} %%% efb 195
\index[esquinas]{$3912_{7}$!{\ene 消}} \index[fon]{xaoi!{\ene 消}}
{\Large 10 [85, 7] \textbf{\begin{tabular}{|c|} \hline 
xi\={a}o\\ \hline \end{tabular} } } \\% \par
%%%%% %%%%% %%%%% %%%%% %%%%% %%%%% %%%%% %%%%%
\subsubsection{\ene 清}\label{清} %%% 清 eq 206
\index[esquinas]{$3512_{7}$!{\ene 清}} \index[fon]{qngi!{\ene 清}}
{\Large 11 [85, 8] \textbf{\begin{tabular}{|c|} \hline 
q\={\i}ng \\ \hline \end{tabular} } } \\% \par
%%%%% %%%%% %%%%% %%%%% %%%%% %%%%% %%%%% %%%%%
\subsubsection{\ene 深}\label{深} %%% ebcd 731 
\index[esquinas]{$3719_{4}$!{\ene 深}} \index[fon]{shen!{\ene 深}}
{\Large 11 [85, 8] \textbf{\begin{tabular}{|c|} \hline 
sh\={e}n \\ \hline \end{tabular} } } \\% \par
%%%%% %%%%% %%%%% %%%%% %%%%% %%%%% %%%%% %%%%%
\subsubsection{\ene 淺}\label{淺} %%% eii 
\index[esquinas]{$3315_{3}$!{\ene 淺}} \index[fon]{qian!{\ene 淺}}
{\Large 11 [85, 8] \textbf{\begin{tabular}{|c|} \hline 
qi\v{a}n \\ \hline \end{tabular} } } \\% \par
%%%%% %%%%% %%%%% %%%%% %%%%% %%%%% %%%%% %%%%%
\subsubsection{\ene 涼}\label{涼} %%% eyrf
\index[esquinas]{$3019_{6}$!{\ene 涼}} \index[fon]{liang!{\ene 涼}}
{\Large 11 [85, 8] \textbf{\begin{tabular}{|c|c|}\hline 
liáng & liàng\\ \hline \end{tabular} } } \\% \par
%%%%% %%%%% %%%%% %%%%% %%%%% %%%%% %%%%% %%%%%
\subsubsection{\ene 滅}\label{滅} %%% eihf 258
\index[esquinas]{$3315_{0}$!{\ene 滅}} \index[fon]{mie!{\ene 滅}}
{\Large 13 [85, 10] \textbf{\begin{tabular}{|c|} \hline 
miè\\ \hline \end{tabular} } } \\% \par
%%%%% %%%%% %%%%% %%%%% %%%%% %%%%% %%%%% %%%%%
\subsubsection{\ene 漢}\label{漢} %%% etlo 698
\index[esquinas]{$3418_{5}$!{\ene 漢}} \index[fon]{han!{\ene 漢}}
{\Large 13 [85, 11] \textbf{\begin{tabular}{|c|} \hline 
hàn\\ \hline \end{tabular} } } \\% \par
%%%%% %%%%% %%%%% %%%%% %%%%% %%%%% %%%%% %%%%%
\subsubsection{\ene 演}\label{演} %%% ejmc 769
\index[esquinas]{$3318_{6}$!{\ene 演}} \index[fon]{yan!{\ene 演}}
{\Large 14 [85, 11] \textbf{\begin{tabular}{|c|} \hline 
y\v{a}n \\ \hline \end{tabular} } } \\% \par
%%%%% %%%%% %%%%% %%%%% %%%%% %%%%% %%%%% %%%%%
\subsubsection{\ene 漂}\label{漂} %%% emwf 
\index[esquinas]{$3119_{1}$!{\ene 漂}} \index[fon]{piao!{\ene 漂}}
{\Large 14 [85, 11] \textbf{\begin{tabular}{|c|c|c|} \hline 
pi\={a}o & pi\v{a}o & piào\\ \hline \end{tabular} } } \\% \par
%%%%% %%%%% %%%%% %%%%% %%%%% %%%%% %%%%% %%%%%
\subsubsection{\ene 激}\label{激} %%% ehsk 1083
\index[esquinas]{$3814_{0}$!{\ene 激}} \index[fon]{ji!{\ene 激}}
{\Large 16 [85, 13] \textbf{\begin{tabular}{|c|} \hline 
j\={\i} \\ \hline \end{tabular} } } \\% \par
%%%%% %%%%% %%%%% %%%%% %%%%% %%%%% %%%%% %%%%%
\setcounter{subsss}{\value{subsubsection}}
\subsection{\ene 火 (灬)}\label{ochentayseiss}   
\setcounter{subsubsection}{\arabic{subsss}} 
%%%%% %%%%% %%%%% %%%%% %%%%% %%%%% %%%%% %%%%%
\subsubsection{\ene 火}\label{火} %%% f 788
\index[esquinas]{$9080_{0}$!{\ene 火}} \index[fon]{huo!{\ene 火}}
{\Large 4 [86, 0] \textbf{\begin{tabular}{|c|} \hline 
hu\v{o} \\ \hline \end{tabular} } } \\% \par
%%%%% %%%%% %%%%% %%%%% %%%%% %%%%% %%%%% %%%%%
\subsubsection{\ene 灭}\label{灭} %%% 
\index[esquinas]{$1080_{0}$!{\ene 灭}} \index[fon]{mie!{\ene 灭}}
{\Large 5 [86, 1]} %% 滅 eihf
\abreviacion\ \enlugarde\ {\ene 滅} \vease\ \textnumero\ \ref{滅}.
%%%%% %%%%% %%%%% %%%%% %%%%% %%%%% %%%%% %%%%%
%%%% %%%%% %%%%% %%%%% %%%%% %%%%% %%%%% %%%%%
\subsubsection{\ene 点}\label{点} %%% yrf2 924
%\index[esquinas]{$1234_{}$!{\ene 点}}
\index[fon]{dian!{\ene 点}}
{\Large 9 [86, 5]} %% wfyr
\abreviacion\ \enlugarde\ {\ene 點} \vease\ \textnumero\ ref. {\ene 點}. 
%%%%% %%%%% %%%%% %%%%% %%%%% %%%%% %%%%% %%%%%
\subsubsection{\ene 無}\label{無} %%% otf 912
\index[esquinas]{$8033_{1}$!{\ene 無}} \index[fon]{wu!{\ene 無}}
{\Large 12 [86, 8] \textbf{\begin{tabular}{|c|} \hline 
wú \\ \hline \end{tabular} } } \\% \par
%%%%% %%%%% %%%%% %%%%% %%%%% %%%%% %%%%% %%%%%
\subsubsection{\ene 照}\label{照} %%% arf 925
\index[esquinas]{$6733_{6}$!{\ene 照}} \index[fon]{zhao!{\ene 照}}
{\Large 13 [86, 5] \textbf{\begin{tabular}{|c|} \hline 
zhào  \\ \hline \end{tabular} } } \\% \par
%%%%% %%%%% %%%%% %%%%% %%%%% %%%%% %%%%% %%%%% 7 --> 8
\subsubsection{\ene 热}\label{热} %%% qif 941
\index[esquinas]{$5533_{1}$!{\ene 热}}\index[fon]{re!{\ene 热}}
{\Large 10 [86, 6]}%% gif 熱 938
\abreviacion\ \enlugarde\ {\ene 熱} \vease\ \textnumero\ \ref{熱}. 
%%%%% %%%%% %%%%% %%%%% %%%%% %%%%% %%%%% %%%%%
\subsubsection{\ene 熱}\label{熱} %%% gif 938
\index[esquinas]{$4533_{1}$!{\ene 熱}} \index[fon]{re!{\ene 熱}}
{\Large 15 [86, 11] \textbf{\begin{tabular}{|c|} \hline 
rè \\ \hline \end{tabular} } } \\% \par
\subsubsection{\ene 灰}\label{灰} %%%  kf 796 
\index[esquinas]{$4080_{9}$!{\ene 灰}}\index[fon]{hui!{\ene 灰}}
{\Large 6 [86, 2] \textbf{\begin{tabular}{|c|} \hline 
h\={u}i\\ \hline \end{tabular} } } \\% \par
%%%%% %%%%% %%%%% %%%%% %%%%% %%%%% %%%%% %%%%%
%%%% %%%%% %%%%% %%%%% %%%%% %%%%% %%%%% %%%%%
\setcounter{subsss}{\value{subsubsection}}
\subsection{\ene 爪 (爫, ⺥)}\label{ochentaysietes}
\setcounter{subsubsection}{\arabic{subsss}} 
%%%%% %%%%% %%%%% %%%%% %%%%% %%%%% %%%%% %%%%%
\subsubsection{\ene 爪}\label{爪} %%% HLO 
\index[esquinas]{$7223_{0}$!{\ene 爪}} \index[fon]{zhao!{\ene 爪}}
{\Large 4 [87, 0] \textbf{\begin{tabular}{|c|} \hline 
zh\v{a}o \\ \hline \end{tabular} } } \\ %\par
%%%% %%%%% %%%%% %%%%% %%%%% %%%%% %%%%% %%%%%
\subsubsection{\ene 爱}\label{爱} %%% bbke 991
\index[esquinas]{$2040_{7}$!{\ene 爱}} \index[fon]{ai!{\ene 爱}}
{\Large 10 [87, 6]}%% bbpe 22
\abreviacion\ \enlugarde\ {\ene 愛} \vease\ \textnumero\ \ref{愛}. 
%%%%% %%%%% %%%%% %%%%% %%%%% %%%%% %%%%% %%%%%
\subsubsection{\ene 爲}\label{爲} %%% bhnf 485
\index[esquinas]{$2022_{7}$!{\ene 爲}} \index[fon]{wei!{\ene 爲}}
{\Large 12 [87, 8] \textbf{\begin{tabular}{|c|c|} \hline 
wéi & wèi \\ \hline \end{tabular} } } \\ %\par
%%%%% %%%%% %%%%% %%%%% %%%%% %%%%% %%%%% %%%%%
\setcounter{subsss}{\value{subsubsection}}
\subsection{\ene 父}\label{ochentayochos}
\setcounter{subsubsection}{\arabic{subsss}} 
%%%%% %%%%% %%%%% %%%%% %%%%% %%%%% %%%%% %%%%%
\subsubsection{\ene 父}\label{父} %%% ck 47
\index[esquinas]{$8040_{0}$!{\ene 父}} \index[fon]{fu!{\ene 父}}
{\Large 4 [88, 0]\textbf{\begin{tabular}{|c|c|} \hline 
fù & f\v{u} \\ \hline \end{tabular} } } \\ %\par
%%%%% %%%%% %%%%% %%%%% %%%%% %%%%% %%%%% %%%%%
\subsubsection{\ene 爷}\label{爷} %%% cksl 773
\index[esquinas]{$8022_{7}$!{\ene 爷}} \index[fon]{ye!{\ene 爷}}
{\Large 6 [88, 2]}%%% 爺 cksj 781
\abreviacion\ \enlugarde\ {\ene 爺} \vease\ \textnumero\ \ref{爺}. 
%%%%% %%%%% %%%%% %%%%% %%%%% %%%%% %%%%% %%%%%
\subsubsection{\ene 爺}\label{爺} %%% cksj  781
\index[esquinas]{$8012_{7}$!{\ene 爺}} \index[fon]{ye!{\ene 爺}}
{\Large 13 [88, 9] \textbf{\begin{tabular}{|c|} \hline 
yé \\ \hline \end{tabular} } } \\ %\par
%%%%% %%%%% %%%%% %%%%% %%%%% %%%%% %%%%% %%%%%
\subsubsection{\ene 爸}\label{爸} %%% ckau 359
\index[esquinas]{$8071_{7}$!{\ene 爸}} \index[fon]{ba!{\ene 爸}}
{\Large 8 [88, 4] \textbf{\begin{tabular}{|c|} \hline 
bà \\ \hline \end{tabular} } } \\ %\par
%%%%% %%%%% %%%%% %%%%% %%%%% %%%%% %%%%% %%%%%
\setcounter{subsss}{\value{subsubsection}}
\subsection{\ene 爻}\label{ochentaynueves} 
\setcounter{subsubsection}{\arabic{subsss}} 
%%%%% %%%%% %%%%% %%%%% %%%%% %%%%% %%%%% %%%%%
\subsubsection{\ene 爻}\label{爻} %%% kk 58
\index[esquinas]{$4040_{0}$!{\ene 爻}}
\index[fon]{yao!{\ene 爻}} \index[fon]{xiao!{\ene 爻}}
{\Large 4 [89, 0] \textbf{\begin{tabular}{|c|c|} \hline 
yáo & xiáo \\ \hline \end{tabular} } } \\ %\par
%%%%% %%%%% %%%%% %%%%% %%%%% %%%%% %%%%% %%%%%
%%%%% %%%%% %%%%% %%%%% %%%%% %%%%% %%%%% %%%%%
\subsubsection{\ene 爾}\label{爾} %%% MFBK 321
\index[esquinas]{$1022_{7}$!{\ene 爾}} \index[fon]{er!{\ene 爾}}
{\Large 14 [89, 10] \textbf{\begin{tabular}{|c|} \hline 
\v{e}r \\ \hline \end{tabular} } } \\ %\par
%%%%% %%%%% %%%%% %%%%% %%%%% %%%%% %%%%% %%%%%
%%%%% %%%%% %%%%% %%%%% %%%%% %%%%% %%%%% %%%%%
\setcounter{subsss}{\value{subsubsection}}
\subsection{\ene 爿}\label{noventas} %vlm 
\setcounter{subsubsection}{\arabic{subsss}} 
%\subsubsection{\ene 将}\label{将} %%% lmnii 将 62 将
%\index[esquinas]{$2724_{2}$!{\ene }} \index[fon]{jiang!{\ene }} 
%{\Large 9 [90, 6]} \abreviacion\ \enlugarde\ {\ene  將} \vease\ \textnumero\ \ref{將}.%vmbdi 
%%%%% %%%%% %%%%% %%%%% %%%%% %%%%% %%%%% %%%%%
\subsubsection{\ene 爿}\label{爿} %%% vlm 750 
\index[esquinas]{$2220_{0}$!{\ene 爿}}
\index[fon]{qiang!{\ene 爿}} \index[fon]{ban!{\ene 爿}}
{\Large 4 [90, 0] \textbf{\begin{tabular}{|c|c|} \hline 
qiáng & bàn \\ \hline \end{tabular} } } \\ %\par
%%%%% %%%%% %%%%% %%%%% %%%%% %%%%% %%%%% %%%%%
\subsubsection{\ene 牀}\label{牀} %%% vmd 703
\index[esquinas]{$2429_{0}$!{\ene 牀}} \index[fon]{chuang!{\ene 牀}}
{\Large 8 [90, 4] \textbf{\begin{tabular}{|c|} \hline 
chuáng \\ \hline \end{tabular} } } \\ %\par
%%%%% %%%%% %%%%% %%%%% %%%%% %%%%% %%%%% %%%%%
\subsubsection{\ene 牆}\label{牆} %%% vmgow 670
\index[esquinas]{$2426_{1}$!{\ene 牆}} \index[fon]{qiang!{\ene 牆}}
{\Large 17 [90, 13] \textbf{\begin{tabular}{|c|} \hline 
qiáng \\ \hline \end{tabular} } } \\ %\par
%%%%% %%%%% %%%%% %%%%% %%%%% %%%%% %%%%% %%%%%
\setcounter{subsss}{\value{subsubsection}}
\subsection{\ene 片}\label{noventayunos}
\setcounter{subsubsection}{\arabic{subsss}} 
%%%%% %%%%% %%%%% %%%%% %%%%% %%%%% %%%%% %%%%%
\subsubsection{\ene 片}\label{片} %%% llml 999
\index[esquinas]{$2202_{7}$!{\ene 片}} \index[fon]{pian!{\ene 片}}
{\Large 4 [91, 0] \textbf{\begin{tabular}{|c|c|} \hline 
piàn & pi\={a}n \\ \hline \end{tabular} } } \\ %\par
%%%%% %%%%% %%%%% %%%%% %%%%% %%%%% %%%%% %%%%%
\subsubsection{\ene 版}\label{版} %%% llhe % lhne 
\index[esquinas]{$2204_{7}$!{\ene 版}} \index[fon]{ban!{\ene 版}}
{\Large 8 [91, 4] \textbf{\begin{tabular}{|c|} \hline 
b\v{a}n\\ \hline \end{tabular} } } \\ %\par
%%%%% %%%%% %%%%% %%%%% %%%%% %%%%% %%%%% %%%%%
\subsubsection{\ene 牌}\label{牌} %%% 牌 LLHH 813
\index[esquinas]{$2604_{0}$!{\ene 牌}} \index[fon]{pai!{\ene 牌}}
{\Large 12 [91, 8] \textbf{\begin{tabular}{|c|} \hline 
pái \\ \hline \end{tabular} } } \\ %\par
%%%%% %%%%% %%%%% %%%%% %%%%% %%%%% %%%%% %%%%%
\setcounter{subsss}{\value{subsubsection}}
\subsection{\ene 牙}\label{noventaydoss}
\setcounter{subsubsection}{\arabic{subsss}} 
%%%%% %%%%% %%%%% %%%%% %%%%% %%%%% %%%%% %%%%%
\subsubsection{\ene 牙}\label{牙} %%% mvdh 83
\index[esquinas]{$1024_{0}$!{\ene 牙}} \index[fon]{ya!{\ene 牙}}
{\Large 4 [92, 0] \textbf{\begin{tabular}{|c|} \hline 
yá \\ \hline \end{tabular} } } \\ %\par
%%%%% %%%%% %%%%% %%%%% %%%%% %%%%% %%%%% %%%%%
\setcounter{subsss}{\value{subsubsection}}
\subsection{\ene 牛 (牜)}\label{noventaytress}
\setcounter{subsubsection}{\arabic{subsss}} 
%%%%% %%%%% %%%%% %%%%% %%%%% %%%%% %%%%% %%%%%
\subsubsection{\ene 牛}\label{牛} %%% hq 868
\index[esquinas]{$2500_{0}$!{\ene 牛}} \index[fon]{niu!{\ene 牛}}
{\Large 4 [93, 0] \textbf{\begin{tabular}{|c|} \hline 
niú \\ \hline \end{tabular} } } \\ %\par
%%%%% %%%%% %%%%% %%%%% %%%%% %%%%% %%%%% %%%%%
\subsubsection{\ene 物}\label{物} %%% hqphn 
\index[esquinas]{$2752_{0}$!{\ene 物}} \index[fon]{wu!{\ene 物}}
{\Large 8 [93, 4] \textbf{\begin{tabular}{|c|} \hline 
wú \\ \hline \end{tabular} } } \\ %\par
%%%%% %%%%% %%%%% %%%%% %%%%% %%%%% %%%%% %%%%%
\setcounter{subsss}{\value{subsubsection}} 
\subsection{\ene 犬 (犭)}\label{noventaycuatross}
\setcounter{subsubsection}{\arabic{subsss}} 
%%%%% %%%%% %%%%% %%%%% %%%%% %%%%% %%%%% %%%%%
\subsubsection{\ene 犬}\label{犬} %%% ik 646
\index[esquinas]{$4380_{0}$!{\ene 犬}} \index[fon]{quan!{\ene 犬}}
{\Large 4 [94, 0] \textbf{\begin{tabular}{|c|} \hline 
qu\v{a}n\\ \hline \end{tabular} } } \\ %\par
%%%%% %%%%% %%%%% %%%%% %%%%% %%%%% %%%%% %%%%%
\subsubsection{\ene 犹}\label{犹} %%% khiku 
%\index[esquinas]{$1234_{}$!{\ene 犹}}
\index[fon]{you!{\ene 犹}}
{\Large 8 [94, 5]}%%  khtcw 706
\abreviacion\ \enlugarde\ {\ene 猶} \vease\ \textnumero\ \ref{猶}. 
%%%%% %%%%% %%%%% %%%%% %%%%% %%%%% %%%%% %%%%%
\subsubsection{\ene 狗}\label{狗} %%% khpr 448
\index[esquinas]{$4722_{0}$!{\ene 狗}}
\index[fon]{gou!{\ene 狗}}\index[fon]{hou!{\ene 狗}}
{\Large 8 [94, 5] \textbf{\begin{tabular}{|c|c|} \hline 
g\v{o}u & hòu \\ \hline \end{tabular} } } \\ %\par
%%%%% %%%%% %%%%% %%%%% %%%%% %%%%% %%%%% %%%%%
\subsubsection{\ene 猪}\label{猪} %%% khjka 594
\index[esquinas]{$4426_{0}$!{\ene 猪}}\index[fon]{zhu!{\ene 猪}}
{\Large 12 [94, 9]}%% mojka 1426-0
\abreviacion\ \enlugarde\ {\ene 豬} \vease\ \textnumero\ ref.{\ene 豬}. 
%%%%% %%%%% %%%%% %%%%% %%%%% %%%%% %%%%% %%%%%
\subsubsection{\ene 猶}\label{猶} %%% khtcw 706
\index[esquinas]{$4826_{4}$!{\ene 猶}} \index[fon]{you!{\ene 猶}}
{\Large 12 [94,9] \textbf{\begin{tabular}{|c|} \hline 
yóu \\ \hline \end{tabular} } } \\ %\par
%%%%% %%%%% %%%%% %%%%% %%%%% %%%%% %%%%% %%%%%
\subsubsection{\ene 猴}\label{猴} %%% khonk 679
\index[esquinas]{$4728_{4}$!{\ene 猴}} \index[fon]{hou!{\ene 猴}}
{\Large 12 [94, 9] \textbf{\begin{tabular}{|c|} \hline 
hóu \\ \hline \end{tabular} } } \\ %\par
%%%%% %%%%% %%%%% %%%%% %%%%% %%%%% %%%%% %%%%%
\setcounter{subsss}{\value{subsubsection}}
\end{multicols}

\section{5 trazos}\label{5trazos}\setcounter{subsection}{94}
\begin{multicols}{2}
\input{5trazos/seccion_5b}
\end{multicols}

\section{6 trazos}\label{6trazos}\setcounter{subsection}{117}
\begin{multicols}{2}
\input{6trazos/seccion_6b}
\end{multicols}

\section{7 trazos}\label{7trazos}\setcounter{subsection}{148}
\begin{multicols}{2}
\input{7trazos/seccion_7ab}
\end{multicols}
\section{8 trazos}\label{8trazos}\setcounter{subsection}{166}
rpe asdf.
\begin{multicols}{2}
\input{8trazos/seccion_8a}
\end{multicols}
\section{9 trazos}\label{9trazos}\setcounter{subsection}{175}
rpe asdf.
\begin{multicols}{2}
\input{9trazos/seccion_9a}
\end{multicols}
\section{10 trazos}\label{10trazos}\setcounter{subsection}{186}
\begin{multicols}{2}
\input{10trazos/seccion_10a}
\end{multicols}
\section{11 trazos}\label{11trazos}\setcounter{subsection}{194}
rpe asdf.
\begin{multicols}{2}
\input{10trazos/seccion_11a}
\end{multicols}

\input{10trazos/seccion_12}

\begin{enumerate}[label=\Roman*]
  \item One 
  \begin{enumerate*}[label=(\alph*)]
\item first item
\item second item
\end{enumerate*}
  \item Two \begin{enumerate*}[label=\textbf{\arabic*})]
\item first item
\item second item
\end{enumerate*}
  \item Three
\end{enumerate}

\begin{enumerate}[label=\roman*]
  \item {\ene a あ気 お ああアオイキ}
  \item Two
  \item Three
\end{enumerate}
\section{Indentaciones}
\begin{enumerate} [label=Step \arabic{enumi}.,ref=Step \arabic{enumi}, leftmargin=*]
\item Step A Step A Step A Step A Step A Step A Step A Step A Step A Step A Step A Step A Step A Step A Step A Step A
\item Step B
\item Step C
\item Step D
\end{enumerate}

\noindent
Some text Some text Some text Some text Some text Some text Some text Some text Some text Some text

\begin{enumerate} [label=Step \arabic{enumi}.,ref=Step \arabic{enumi}, wide=0pt]
\item Step A Step A Step A Step A Step A Step A Step A Step A Step A Step A Step A Step A Step A Step A Step A Step A
\item Step B
\item Step C
\item Step D
\end{enumerate}

\noindent
Some text Some text Some text Some text Some text Some text Some text Some text Some text Some text

\begin{enumerate} [label=Step \arabic{enumi}.,ref=Step \arabic{enumi}, align=left]
\item Step A Step A Step A Step A Step A Step A Step A Step A Step A Step A Step A Step A Step A Step A Step A Step A
\item Step B
\item Step C
\item Step D
\end{enumerate}
%\pagebreak
%\section{Bibliograf\'{i}a}
%\nocite{*}
%\printbibliography[heading=bibintoc]%<- add a ToC entry for the bibliography
$_{1}^{0}${\ene 百}$_{1}^{0}$
\pagebreak
\input{tabla/hyper_tabla}
%\printindex[fon]
\pagebreak
%\input{tabla/hyp_cuarto}
\printindex[fon]
%\pagebreak
\printindex[esquinas]
\end{document}