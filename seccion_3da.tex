%\begin{multicols}{2}
\subsection{\ene  口 } \label{treintas} \setcounter{subsubsection}{\arabic{subsss}} %

%%%%% %%%%% %%%%% %%%%% %%%%% %%%%% %%%%% %%%%%
\subsubsection{\ene 口}\label{口} %%% rc 667
\index[esquinas]{$6000_{0}$!{\ene 口}} \index[fon]{kou!{\ene 口}}
{\Large 3 [30, 0]  \textbf{\begin{tabular}{|c|} \hline 
k\v{o}u\\ \hline \end{tabular} } } \\% \par
%%%%% %%%%% %%%%% %%%%% %%%%% %%%%% %%%%% %%%%%
\subsubsection{\ene 右}\label{右} %%% rc 667
\index[esquinas]{$4060_{0}$!{\ene 右}} \index[fon]{you!{\ene 右}}
{\Large 5 [30, 2]  \textbf{\begin{tabular}{|c|c|} \hline 
yòu & y\v{o}u\\ \hline \end{tabular} } } \\% \par
%%%%% %%%%% %%%%% %%%%% %%%%% %%%%% %%%%% %%%%%
\subsubsection{\ene 台}\label{台} %%% ir 534
\index[esquinas]{$2360_{0}$!{\ene 台}}
\index[fon]{tai!{\ene 台}}\index[fon]{yi!{\ene 台}}
{\Large 5 [30, 2]  \textbf{\begin{tabular}{|c|c|c|} \hline 
tái & t\={a}i & yí\\ \hline \end{tabular} } } \\% \par
%%%%% %%%%% %%%%% %%%%% %%%%% %%%%% %%%%% %%%%%
\subsubsection{\ene 叫}\label{叫} %%% rvl 942
\index[esquinas]{$6200_{0}$!{\ene 叫}}\index[fon]{jiao!{\ene 叫}}
{\Large 5 [30, 2]  \textbf{\begin{tabular}{|c|} \hline 
jiào\\ \hline \end{tabular} } } \\% \par
%%%%% %%%%% %%%%% %%%%% %%%%% %%%%% %%%%% %%%%%
\subsubsection{\ene 可}\label{可} %%% mnr 1058
\index[esquinas]{$1062_{0}$!{\ene 可}} \index[fon]{ke!{\ene 可}}
{\Large 5 [30, 2]  \textbf{\begin{tabular}{|c|c|} \hline 
k\v{e} & kè\\ \hline \end{tabular} } } \\% \par
%%%%% %%%%% %%%%% %%%%% %%%%% %%%%% %%%%% %%%%%
\subsubsection{\ene 号}\label{号} %%% rmvs 491
\index[esquinas]{$6002_{7}$!{\ene 号}}\index[fon]{hao!{\ene 号}}
{\Large 5 [30, 2] }
\abreviacion\ \enlugarde\ {\ene 號 } \vease\ \textnumero\ ref. {\ene 號}. 
%%%%% %%%%% %%%%% %%%%% %%%%% %%%%% %%%%% %%%%%
\subsubsection{\ene 只}\label{只} %%% rc 667
\index[esquinas]{$6080_{0}$!{\ene 只}} \index[fon]{zhi!{\ene 只}}
{\Large 5 [30, 2]  \textbf{\begin{tabular}{|c | c | c|} \hline 
zh\v{\i} & zh\={\i} & zhí\\ \hline \end{tabular} } } \\% \par
\abreviacion\ \tambien\ \enlugarde\ {\ene 戠} %% yai
\textit{en la parte derecha de los ideogramas, por ejemplo} 
 {\ene 识} \textbf{ shí, zhì} \enlugarde\  {\ene 識}
\begin{enumerate} [noitemsep,label=\Roman{enumi}.,ref=\Roman{enumi}, leftmargin=*]
\item   \textit{adverbio} sólo, solamente, únicamente;
\item \textit{sustantivo/cuantificador} {\begin{enumerate*}[label=\textbf{\arabic*})]
\item \enlugarde\ {\ene 隻 } (\textit{sin par, único, aislado}).
\item \enlugarde\ {\ene 隻 } (\textit{impar}).
\end{enumerate*}}
\end{enumerate}
\entry{只管}{zh\v{\i}'guàn}{}{solo para saber que}
\entry{只}{zh\={\i}ng}{}{.}

\subsubsection{\ene 合}\label{合} %%% omr 449
\index[esquinas]{$8060_{1}$!{\ene 合}} \index[fon]{he!{\ene 合}}
{\Large 6 [30, 3]  \textbf{\begin{tabular}{|c|c|} \hline 
hé & h\v{e}\\ \hline \end{tabular} } } \\% \par
%%%%% %%%%% %%%%% %%%%% %%%%% %%%%% %%%%% %%%%%
\subsubsection{\ene 名}\label{名} %%% nir 505
\index[esquinas]{$2760_{2}$!{\ene 名}} \index[fon]{ming!{\ene 名}}
{\Large 6 [30, 3]  \textbf{\begin{tabular}{|c|} \hline 
míng\\ \hline \end{tabular} } } \\% \par
%%%%% %%%%% %%%%% %%%%% %%%%% %%%%% %%%%% %%%%%
\subsubsection{\ene 各}\label{各} %%% her 519
\index[esquinas]{$2760_{4}$!{\ene 各}} \index[fon]{ge!{\ene 各}}
{\Large 6 [30, 3]  \textbf{\begin{tabular}{|c|c|} \hline 
gè & gé \\ \hline \end{tabular} } } \\% \par
%%%%% %%%%% %%%%% %%%%% %%%%% %%%%% %%%%% %%%%%
\subsubsection{\ene 同}\label{同} %%% bmr 258
\index[esquinas]{$7722_{0}$!{\ene 同}} \index[fon]{tong!{\ene 同}}
{\Large 6 [30, 3]  \textbf{\begin{tabular}{|c|c|} \hline 
tóng & tòng \\ \hline \end{tabular} } } \\% \par
%%%%% %%%%% %%%%% %%%%% %%%%% %%%%% %%%%% %%%%%
\subsubsection{\ene 向} %%% hbr 向 284
\index[esquinas]{$2722_{0}$!{\ene 向}} \index[fon]{xiang!{\ene 向}}
\index[fon]{shang!{\ene 向}}
{\Large 6 [30, 3]  \textbf{\begin{tabular}{|c | c |} \hline 
xiàng & xi\r{a}ng \\ \hline \end{tabular} } } 
{\Large  \textbf{\begin{tabular}{|c|} \hline 
shàng \\ \hline \end{tabular} } } \\% \par\\ \par
 \textbf{xiàng} %\adjetivo/\adverbio\ \uno\
  dirigirse a; voltear el rostro hacia;
%%%%% %%%%% %%%%% %%%%% %%%%% %%%%% %%%%% %%%%%
\subsubsection{\ene 吗}\label{吗} %%% rnvm
\index[esquinas]{$6702_{7}$!{\ene 吗}}
\index[fon]{ma!{\ene 吗}}
{\Large 6 [30, 3] }
\abreviacion\ \enlugarde\ {\ene 嗎 } \vease\ \textnumero\ \ref{嗎}. %% rs1 嗎 
%%%%% %%%%% %%%%% %%%%% %%%%% %%%%% %%%%% %%%%%
\subsubsection{\ene 吧 }\label{吧 } %%% rau 353
\index[esquinas]{$6701_{1}$!{\ene 吧}}\index[fon]{xiang!{\ene 吧}}
{\Large 6 [30, 3]  \textbf{\begin{tabular}{|c|c|} \hline 
b\={a} & -b\r{a} \\ \hline \end{tabular} } } \\% \par
%%%%% %%%%% %%%%% %%%%% %%%%% %%%%% %%%%% %%%%%
\subsubsection{\ene 吃 }\label{吃 } %%% ron 569 
\index[esquinas]{$6801_{1}$!{\ene 吃}}\index[fon]{chi!{\ene 吃}}
{\Large 6 [30, 3]  \textbf{\begin{tabular}{|c|} \hline 
ch\={\i} \\ \hline \end{tabular} } } \\% \par
%%%%% %%%%% %%%%% %%%%% %%%%% %%%%% %%%%% %%%%%
\subsubsection{\ene 员}\label{员} %%% rbo 618
\index[esquinas]{$6080_{2}$!{\ene 员}}\index[fon]{yuan!{\ene 员}}
{\Large 7 [30, 4]}
\abreviacion\ \enlugarde\ {\ene 員 } \vease\ \textnumero\ \ref{員}.%% rbuc 681 
%%%%% %%%%% %%%%% %%%%% %%%%% %%%%% %%%%% %%%%%
\subsubsection{\ene 咖 }\label{咖 } %%% rksr 414
\index[esquinas]{$6600_{0}$!{\ene 咖}}
\index[fon]{ka!{\ene 咖}}\index[fon]{jia!{\ene 咖}}
\index[fon]{jie!{\ene 咖}}\index[fon]{ga!{\ene 咖}}
{\Large 8 [30, 5]  \textbf{\begin{tabular}{|c|c|c|c|} \hline 
k\={a} & ji\={a} & ji\={e} &g\={a} \\ \hline
\end{tabular} } } \\% \par
%%%%% %%%%% %%%%% %%%%% %%%%% %%%%% %%%%% %%%%%
\subsubsection{\ene 和}\label{和} %%% rksr 414
\index[esquinas]{$2690_{0}$!{\ene 和}}
\index[fon]{he!{\ene 和}}\index[fon]{huo!{\ene 和}}
\index[fon]{hu!{\ene 和}}%\index[fon]{ga!{\ene 和}}
{\Large 8 [30, 5]  \textbf{\begin{tabular}{|c|c|c|} \hline 
hé & huó & hú \\ \hline \end{tabular} } } \\% \par
%%%%% %%%%% %%%%% %%%%% %%%%% %%%%% %%%%% %%%%%
\subsubsection{\ene 呼}\label{呼} %%% rhfd 82
\index[esquinas]{$6204_{9}$!{\ene 呼}} \index[fon]{hu!{\ene 呼}}
{\Large 8 [30, 5]  \textbf{\begin{tabular}{|c|} \hline 
h\={u}\\ \hline \end{tabular} } } \\% \par
%%%%% %%%%% %%%%% %%%%% %%%%% %%%%% %%%%% %%%%%
\subsubsection{\ene 周}\label{周} %%% bgr 267
\index[esquinas]{$7722_{0}$!{\ene 周}} \index[fon]{zhou!{\ene 周}}
{\Large 8 [30, 5]  \textbf{\begin{tabular}{|c|} \hline 
zh\={o}u \\ \hline \end{tabular} } } \\% \par
%%%%% %%%%% %%%%% %%%%% %%%%% %%%%% %%%%% %%%%%
\subsubsection{\ene 呢}\label{呢} %%% rsp 308
\index[esquinas]{$6701_{2}$!{\ene 呢}} \index[fon]{ni!{\ene 呢}}
 \index[fon]{ne!{\ene 呢}}
{\Large 8 [30, 5]  \textbf{\begin{tabular}{|c|c|c|} \hline 
-n\r{\i} & -n\r{e} & ní\\ \hline \end{tabular} } } \\% \par
%%%%% %%%%% %%%%% %%%%% %%%%% %%%%% %%%%% %%%%%
\subsubsection{\ene 哈}\label{哈} %%% rsp 308
\index[esquinas]{$6701_{2}$!{\ene 哈}} \index[fon]{ha!{\ene 哈}}
 %\index[fon]{ne!{\ene 哈}}
{\Large 9 [30, 6]  \textbf{\begin{tabular}{|c|c|c|} \hline 
h\={a} & h\v{a} & hà\\ \hline \end{tabular} } } \\% \par
%%%%% %%%%% %%%%% %%%%% %%%%% %%%%% %%%%% %%%%%
\subsubsection{\ene 哪}\label{哪} %%% rsql 788
\index[esquinas]{$6702_{7}$!{\ene 哪}} \index[fon]{na!{\ene 哪}}
 \index[fon]{nei!{\ene 哪}}
{\Large 10 [30, 7]  \textbf{\begin{tabular}{|c|c|c|} \hline 
n\v{a}, n\v{e}i & n\={a} & -n\r{a}\\
\hline \end{tabular} } } \\% \par
%%%%% %%%%% %%%%% %%%%% %%%%% %%%%% %%%%% %%%%%
\subsubsection{\ene 員}\label{員} %%% rbuc 681
\index[esquinas]{$6080_{6}$!{\ene 員}} \index[fon]{yuan!{\ene 員}}
 \index[fon]{yun!{\ene 員}}
{\Large 10 [30, 7]  \textbf{\begin{tabular}{|c|c|c|} \hline 
yuán & yún & y\={u}n \\ \hline \end{tabular} } } \\% \par
%%%%% %%%%% %%%%% %%%%% %%%%% %%%%% %%%%% %%%%%
\subsubsection{\ene 哥}\label{哥} %%% mrnr 1070
\index[esquinas]{$1062_{7}$!{\ene 哥}} \index[fon]{ge!{\ene 哥}}
{\Large 10 [30, 7]  \textbf{\begin{tabular}{|c|} \hline 
g\={e}\\\hline \end{tabular} } } \\% \par
%%%%% %%%%% %%%%% %%%%% %%%%% %%%%% %%%%% %%%%%
\subsubsection{\ene 唱}\label{唱} %%% raa 566
\index[esquinas]{$6606_{0}$!{\ene 唱}} \index[fon]{chang!{\ene 唱}}
{\Large 11 [30, 8]  \textbf{\begin{tabular}{|c|} \hline 
chàng \\\hline \end{tabular} } } \\% \par
%%%%% %%%%% %%%%% %%%%% %%%%% %%%%% %%%%% %%%%%
\subsubsection{\ene 啤}\label{啤} %%% rhhj 812
\index[esquinas]{$6604_{0}$!{\ene 啤}} \index[fon]{pi!{\ene 啤}}
{\Large 11 [30, 8]  \textbf{\begin{tabular}{|c|} \hline 
pí \\\hline \end{tabular} } } \\% \par
%%%%% %%%%% %%%%% %%%%% %%%%% %%%%% %%%%% %%%%%
\subsubsection{\ene 啡}\label{啡} %%% rlmy964
\index[esquinas]{$6101_{1}$!{\ene 啡}} \index[fon]{fei!{\ene 啡}}
{\Large 11 [30, 8]  \textbf{\begin{tabular}{|c|} \hline 
f\={e}i \\\hline \end{tabular} } } \\% \par
%%%%% %%%%% %%%%% %%%%% %%%%% %%%%% %%%%% %%%%%
\subsubsection{\ene 啊}\label{啊} %%% rnlr 1067
\index[esquinas]{$6102_{0}$!{\ene 啊}} \index[fon]{a!{\ene 啊}}
\index[fon]{e!{\ene 啊}}
{\Large 11 [30, 8]  \textbf{\begin{tabular}{|c|c|c|c|c|c|} \hline 
\r{a} & \={a} & á & \v{a} & à & \={e} \\\hline
\end{tabular} } } \\% \par
%%%%% %%%%% %%%%% %%%%% %%%%% %%%%% %%%%% %%%%%
\subsubsection{\ene 問}\label{問} %%% anr 127
\index[esquinas]{$7760_{7}$!{\ene 問}} \index[fon]{wen!{\ene 問}}
{\Large 11 [30, 8]  \textbf{\begin{tabular}{|c|} \hline 
wèn  \\\hline \end{tabular} } } \\% \par
%%%%% %%%%% %%%%% %%%%% %%%%% %%%%% %%%%% %%%%%
\subsubsection{\ene 商}\label{商} %%% ycbr 276
\index[esquinas]{$0022_{7}$!{\ene 商}} \index[fon]{shang!{\ene 商}}
{\Large 11 [30, 8]  \textbf{\begin{tabular}{|c|} \hline 
sh\={a}ng\\\hline \end{tabular} } } \\% \par
%%%%% %%%%% %%%%% %%%%% %%%%% %%%%% %%%%% %%%%%
\subsubsection{\ene 善}\label{善} %%% 善 tgtr 463
\index[esquinas]{$8060_{1}$!{\ene 善}} \index[fon]{shan!{\ene 善}}
{\Large 12 [30, 9]  \textbf{\begin{tabular}{|c|} \hline 
shàn\\\hline \end{tabular} } } \\% \par
%%%%% %%%%% %%%%% %%%%% %%%%% %%%%% %%%%% %%%%%
\subsubsection{\ene 喜}\label{喜} %%% grtr 462
\index[esquinas]{$4060_{1}$!{\ene 喜}} \index[fon]{xi!{\ene 喜}}
{\Large 12 [30, 9]  \textbf{\begin{tabular}{|c|} \hline 
x\v{\i} \\\hline \end{tabular} } } \\% \par
%%%%% %%%%% %%%%% %%%%% %%%%% %%%%% %%%%% %%%%%
\subsubsection{\ene 單}\label{單} %%% rrwj 904
\index[esquinas]{$6650_{6}$!{\ene 單}}
\index[fon]{dan!{\ene 單}}\index[fon]{shan!{\ene 單}}
{\Large 12 [30, 9]  \textbf{\begin{tabular}{|c|c|} \hline 
d\={a}n & shàn \\\hline \end{tabular} } } \\% \par
%%%%% %%%%% %%%%% %%%%% %%%%% %%%%% %%%%% %%%%% 單rrwj
\subsubsection{\ene 喇}\label{喇} %%% rdln 1022
\index[esquinas]{$6200_{0}$!{\ene 喇}} \index[fon]{la!{\ene 喇}}
{\Large 12 [30, 9]  \textbf{\begin{tabular}{|c|c|} \hline 
lá & l\={a} \\\hline \end{tabular} } } \\% \par
%%%%% %%%%% %%%%% %%%%% %%%%% %%%%% %%%%% %%%%%
\subsubsection{\ene 喝}\label{喝} %%% rapv 435
\index[esquinas]{$6602_{7}$!{\ene 喝}}
\index[fon]{he!{\ene 喝}}\index[fon]{ye!{\ene 喝}}
{\Large 12 [30, 9]  \textbf{\begin{tabular}{|c|c|c|} \hline 
h\={e} & hè  & yè \\\hline \end{tabular} } } \\% \par
%%%%% %%%%% %%%%% %%%%% %%%%% %%%%% %%%%% %%%%% 喂rapv
\subsubsection{\ene 喂}\label{喂} %%% rapv 851
\index[esquinas]{$6603_{2}$!{\ene 喂}}
\index[fon]{wei!{\ene 喂}}\index[fon]{wai!{\ene 喂}}
{\Large 12 [30, 9]  \textbf{\begin{tabular}{|c|} \hline 
wèi \\ \hline \end{tabular} } } \\% \par
{\Large  \textbf{\begin{tabular}{|c|c|c|c|} \hline 
w\={e}i & w\v{e}i &w\={a}i & wái\\ \hline \end{tabular} } } \\% \par
%%%%% %%%%% %%%%% %%%%% %%%%% %%%%% %%%%% %%%%% 喂rwmv
\subsubsection{\ene 嗎}\label{嗎} %%% rsq 531
\index[esquinas]{$6102_{7}$!{\ene 嗎}}\index[fon]{ma!{\ene 嗎}}
{\Large 13 [30, 10]  \textbf{\begin{tabular}{|c|c|c|} \hline 
m\r{a} & má & m\v{a}\\\hline \end{tabular} } } \\% \par
%%%%% %%%%% %%%%% %%%%% %%%%% %%%%% %%%%% %%%%% 嗎rapv

\setcounter{subsss}{\value{subsubsection}} dfg
\subsection{\ene  囗  } \label{treintayunos}%b
\setcounter{subsubsection}{\arabic{subsss}} %\input{1trazo/dos} 
%%%%% %%%%% %%%%% %%%%% %%%%% %%%%% %%%%% %%%%%
\subsubsection{\ene 囗}\label{囗} %%% xxbm 667
\index[esquinas]{$6000_{0}$!{\ene 囗}}
\index[fon]{wei!{\ene 囗}}\index[fon]{guo!{\ene 囗}}
{\Large 3 [31, 0]  \textbf{\begin{tabular}{|c|c|} \hline 
wéi & guó\\ \hline \end{tabular} } } \\% \par
%%%%% %%%%% %%%%% %%%%% %%%%% %%%%% %%%%% %%%%%
\subsubsection{\ene 回}\label{回} %%% wr 667
\index[esquinas]{$6060_{0}$!{\ene 回}} \index[fon]{hui!{\ene 回}}
{\Large 6 [31, 3]  \textbf{\begin{tabular}{|c|c|} \hline 
húi & hùi\\ \hline \end{tabular} } } \\% \par
%%%%% %%%%% %%%%% %%%%% %%%%% %%%%% %%%%% %%%%%
\subsubsection{\ene 四}\label{四} %%% wc 
\index[esquinas]{$6021_{2}$!{\ene 四}} \index[fon]{si!{\ene 四}}
{\Large 6 [31, 3]  \textbf{\begin{tabular}{|c|} \hline 
 sì\\ \hline \end{tabular} } } \\% \par
%%%%% %%%%% %%%%% %%%%% %%%%% %%%%% %%%%% %%%%%
\subsubsection{\ene 因}\label{因} %%% wk 678
\index[esquinas]{$6080_{4}$!{\ene 因}} \index[fon]{yin!{\ene 因}}
{\Large 6 [31, 3]  \textbf{\begin{tabular}{|c|} \hline 
y\={\i}n \\ \hline \end{tabular} } } %\\ \par
\begin{enumerate} [noitemsep,label=\Roman{enumi}.,ref=\Roman{enumi}, leftmargin=*]
\item \textit{sustantivo}  {\begin{enumerate*}[label=\textbf{\arabic*})]
\item  causa, razón, motivo, conductor; \item principio, origen, fuente.
\end{enumerate*}}
\item \textit{preposición} {\begin{enumerate*}[label=\textbf{\arabic*})]
\item por causa de\ldots; gracias a\ldots; \item dependiendo de\ldots, correspondiendo a\ldots;
\end{enumerate*}}
\item \textit{adverbio}
\end{enumerate}
\entry{因子}{y\={\i}n'z\v{\i}}{\textit{matemáticas}}{factor, múltiplo; {\ene 公因子} común múltiplo.}
%%%%% %%%%% %%%%% %%%%% %%%%% %%%%% %%%%% %%%%%
\subsubsection{\ene 围}\label{围} %%% wqs 677
\index[esquinas]{$6052_{7}$!{\ene 围}}\index[fon]{wei!{\ene 围}}
{\Large  7 [31, 4]}
\abreviacion\ \enlugarde\ {\ene 圍} \vease\ \textnumero\ \ref{圍} %% wdmq
%%%%% %%%%% %%%%% %%%%% %%%%% %%%%% %%%%% %%%%%
\subsubsection{\ene 国}\label{国} %%% wmgi 667
\index[esquinas]{$6010_{3}$!{\ene 国}}\index[fon]{guo!{\ene 国}}
{\Large  8 [31, 5]}
\abreviacion\ \enlugarde\ {\ene 國} \vease\ \textnumero\ \ref{國} %% wirm
%%%%% %%%%% %%%%% %%%%% %%%%% %%%%% %%%%% %%%%%
\subsubsection{\ene 图}\label{图} %%% whey 645
\index[esquinas]{$6030_{3}$!{\ene 图}}\index[fon]{tu!{\ene 图}}
{\Large 8 [31,5] }
\abreviacion\ \enlugarde\ {\ene 圖} \vease\ \textnumero\ \ref{圖}. 
%%%%% %%%%% %%%%% %%%%% %%%%% %%%%% %%%%% %%%%%
\subsubsection{\ene 圆}\label{圆} %%% wrbo 693
\index[esquinas]{$6080_{2}$!{\ene 圆}}\index[fon]{yuan!{\ene 圆}}
{\Large 10 [31,7] }
\abreviacion\ \enlugarde\ {\ene 圓} \vease\ \textnumero\ \ref{圓}
%%%%% %%%%% %%%%% %%%%% %%%%% %%%%% %%%%% %%%%%
\subsubsection{\ene 國 }\label{國} %%% wirm 684
\index[esquinas]{$6015_{3}$!{\ene 國 }} \index[fon]{guo!{\ene 國 }}
{\Large 11 [31, 8]  \textbf{\begin{tabular}{|c|} \hline 
guó\\ \hline \end{tabular} } } \\% \par
%%%%% %%%%% %%%%% %%%%% %%%%% %%%%% %%%%% %%%%%
\subsubsection{\ene 圍}\label{圍} %%% wdmq 675
\index[esquinas]{$6050_{6}$!{\ene 圍}} \index[fon]{wei!{\ene 圍}}
{\Large 12 [31, 9]  \textbf{\begin{tabular}{|c|} \hline 
 wéi \\ \hline \end{tabular} } } \\% \par
{\Large \textbf{\begin{tabular}{|c|} \hline 
w\={e}i  \\ \hline \end{tabular} } } \\% \par
%%%%% %%%%% %%%%% %%%%% %%%%% %%%%% %%%%% %%%%%
\subsubsection{\ene 圓 }\label{圓} %%% wrbc 693
\index[esquinas]{$6080_{6}$!{\ene 圓 }} \index[fon]{yuan!{\ene 圓 }}
{\Large 13 [31, 10]  \textbf{\begin{tabular}{|c|} \hline 
guó\\ \hline \end{tabular} } } \\% \par
%%%%% %%%%% %%%%% %%%%% %%%%% %%%%% %%%%% %%%%%
\subsubsection{\ene 圖 }\label{圖} %%% wryw 673
\index[esquinas]{$6060_{0}$!{\ene 圖}} \index[fon]{ttu!{\ene 圖}}
{\Large 14 [31, 11]  \textbf{\begin{tabular}{|c|} \hline 
tú\\ \hline \end{tabular} } } \\% \par
%%%%% %%%%% %%%%% %%%%% %%%%% %%%%% %%%%% %%%%%
\setcounter{subsss}{\value{subsubsection}} dfg
\subsection{\ene  土  } \label{treintaydoss} c
\setcounter{subsubsection}{\arabic{subsss}} %\input{1trazo/dos} 
%%%%% %%%%% %%%%% %%%%% %%%%% %%%%% %%%%% %%%%%
\subsubsection{\ene 土}\label{土} %%% g 96
\index[esquinas]{$4010_{0}$!{\ene 土}} \index[fon]{tu!{\ene 土}}
{\Large 3 [32, 0]  \textbf{\begin{tabular}{|c|} \hline 
t\v{u}\\ \hline \end{tabular} } } \\% \par
%%%%% %%%%% %%%%% %%%%% %%%%% %%%%% %%%%% %%%%%
\subsubsection{\ene 圣}\label{圣} %%% rc 667
\index[esquinas]{$1710_{4}$!{\ene 圣}}\index[fon]{sheng!{\ene 圣}}
{\Large 5 [32, 2] }
\abreviacion\ \enlugarde\ {\ene 聖 } \vease\ \textnumero\ ref. {\ene 聖}. 
%%%%% %%%%% %%%%% %%%%% %%%%% %%%%% %%%%% %%%%% srhg 聖
\subsubsection{\ene 在}\label{在} %%% kl 103
\index[esquinas]{$4021_{4}$!{\ene 在}} \index[fon]{zai!{\ene 在}}
{\Large 6 [32, 3]  \textbf{\begin{tabular}{|c|} \hline 
zài \\ \hline \end{tabular} } } \\% \par
%%%%% %%%%% %%%%% %%%%% %%%%% %%%%% %%%%% %%%%%
\subsubsection{\ene 地}\label{地} %%% gpd 103
\index[esquinas]{$4411_{2}$!{\ene 地}} \index[fon]{di!{\ene 地}}
{\Large 6 [32, 3]  \textbf{\begin{tabular}{|c|} \hline 
dì \\ \hline \end{tabular} } } \\% \par
%%%%% %%%%% %%%%% %%%%% %%%%% %%%%% %%%%% %%%%%
\subsubsection{\ene 址}\label{址} %%% gylm 221
\index[esquinas]{$4111_{0}$!{\ene 址}} \index[fon]{zhi!{\ene 址}}
{\Large 7 [32, 4]  \textbf{\begin{tabular}{|c|} \hline 
zh\v{\i} \\ \hline \end{tabular} } } \\% \par
%%%%% %%%%% %%%%% %%%%% %%%%% %%%%% %%%%% %%%%%
\subsubsection{\ene 坐}\label{坐} %%% oog 142
\index[esquinas]{$8810_{4}$!{\ene 坐}} \index[fon]{zuo!{\ene 坐}}
{\Large 7 [32, 4]  \textbf{\begin{tabular}{|c|} \hline 
zuò \\ \hline \end{tabular} } } \\% \par
{\Large  \textbf{\begin{tabular}{|c|} \hline 
z\v{o} \\ \hline \end{tabular} } } \\% \par
%%%%% %%%%% %%%%% %%%%% %%%%% %%%%% %%%%% %%%%%
\subsubsection{\ene 坝}\label{坝} %%% gbo 617
%\index[esquinas]{$1234_{7}$!{\ene 坝}}
\index[fon]{ba!{\ene 坝}}
{\Large 7 [32, 4]}
\abreviacion\ \enlugarde\ {\ene 壩} \vease\ \textnumero\ \ref{壩}. 
%%%%% %%%%% %%%%% %%%%% %%%%% %%%%% %%%%% %%%% gmbb 170%
\subsubsection{\ene 块}\label{块} %%% gdk
%\index[esquinas]{$1234_{7}$!{\ene 块}}
\index[fon]{kuai!{\ene 块}}\index[fon]{kui!{\ene 块}}
{\Large 7 [32, 4]}
\abreviacion\ \enlugarde\ {\ene 塊} \vease\ \textnumero\ \ref{塊}. 
%%%%% %%%%% %%%%% %%%%% %%%%% %%%%% %%%%% %%%% gmbb 170%
\subsubsection{\ene 坦}\label{坦} %%% gam 37
\index[esquinas]{$4611_{0}$!{\ene 坦}} \index[fon]{tan!{\ene 坦}}
{\Large 8 [32, 5]  \textbf{\begin{tabular}{|c|} \hline 
t\v{a}n\\ \hline \end{tabular} } } \\% \par
%%%%% %%%%% %%%%% %%%%% %%%%% %%%%% %%%%% %%%%%
\subsubsection{\ene 城}\label{城} %%% gihs 257
\index[esquinas]{$4315_{0}$!{\ene 城}} \index[fon]{cheng!{\ene 城}}
{\Large 10 [32, 7]  \textbf{\begin{tabular}{|c|} \hline 
chéng\\ \hline \end{tabular} } } \\% \par
%%%%% %%%%% %%%%% %%%%% %%%%% %%%%% %%%%% %%%%%
\subsubsection{\ene 基}\label{基} %%% tcg  117
\index[esquinas]{$4410_{4}$!{\ene 基}} \index[fon]{ji!{\ene 基}}
{\Large 11 [30, 8]  \textbf{\begin{tabular}{|c|} \hline 
j\={\i}\\ \hline \end{tabular} } } \\% \par
%%%%% %%%%% %%%%% %%%%% %%%%% %%%%% %%%%% %%%%%
\subsubsection{\ene 墙}\label{墙} %%% ggcw 670
\index[esquinas]{$4416_{1}$!{\ene 墙}}\index[fon]{qiang!{\ene 墙}}
{\Large 14 [32, 11] }
\abreviacion\ \enlugarde\ {\ene 牆 } \vease\ \textnumero\ ref. {\ene 牆}. 
%%%%% %%%%% %%%%% %%%%% %%%%% %%%%% %%%%% %%%%%
\subsubsection{\ene 墨}\label{墨} %%% wgfg 133
\index[esquinas]{$6010_{4}$!{\ene 墨}} \index[fon]{mo!{\ene 墨}}
{\Large 15 [32, 12]  \textbf{\begin{tabular}{|c|} \hline 
mò\\ \hline \end{tabular} } } \\% \par
%%%%% %%%%% %%%%% %%%%% %%%%% %%%%% %%%%% %%%%%

%%%%% %%%%% %%%%% %%%%% %%%%% %%%%% %%%%% %%%%%
\subsubsection{\ene 塊}\label{塊} %%% ghi
\index[esquinas]{$4611_{3}$!{\ene 塊}}
\index[fon]{kuai!{\ene 塊}}\index[fon]{kui!{\ene 塊}}
{\Large 13 [32, 10]  \textbf{\begin{tabular}{|c|} \hline 
kuài\\ \hline \end{tabular} } } \\% \par
{\Large  \textbf{\begin{tabular}{|c|} \hline 
kùi\\ \hline \end{tabular} } } \\% \par
%%%%% %%%%% %%%%% %%%%% %%%%% %%%%% %%%%% %%%%%

%%%%% %%%%% %%%%% %%%%% %%%%% %%%%% %%%%% %%%%%
\subsubsection{\ene 境}\label{境} %%% gytu 453
\index[esquinas]{$4011_{2}$!{\ene 境}} \index[fon]{jing!{\ene 境}}
{\Large 14 [32, 11]  \textbf{\begin{tabular}{|c|c|} \hline 
jìng & j\d{\i}ng\\ \hline \end{tabular} } } \\% \par
%%%%% %%%%% %%%%% %%%%% %%%%% %%%%% %%%%% %%%%%
\subsubsection{\ene 壓}\label{壓} %%% mkg 121
\index[esquinas]{$7121_{4}$!{\ene 壓}} \index[fon]{ya!{\ene 壓}}
{\Large 17 [32, 14]  \textbf{\begin{tabular}{|c|} \hline 
y\={\i}\\ \hline \end{tabular} } } \\% \par
%%%%% %%%%% %%%%% %%%%% %%%%% %%%%% %%%%% %%%%%
\subsubsection{\ene 壩}\label{壩} %%% gmbb 170
\index[esquinas]{$4112_{24}$!{\ene 壩}} \index[fon]{ba!{\ene 壩}}
{\Large 24 [32, 21]  \textbf{\begin{tabular}{|c|} \hline 
bà \\ \hline \end{tabular} } } \\% \par
%%%%% %%%%% %%%%% %%%%% %%%%% %%%%% %%%%% %%%%%

\setcounter{subsss}{\value{subsubsection}} dfg
\subsection{\ene  士} \label{treintaytress}d
\setcounter{subsubsection}{\arabic{subsss}} %\input{1trazo/dos} 
%%%%% %%%%% %%%%% %%%%% %%%%% %%%%% %%%%% %%%%%
\subsubsection{\ene 士}\label{士} %%% jm 93
\index[esquinas]{$4010_{0}$!{\ene 士}} \index[fon]{shi!{\ene 士}}
{\Large 3 [33, 0]  \textbf{\begin{tabular}{|c|} \hline 
shì\\ \hline \end{tabular} } } \\% \par
%%%%% %%%%% %%%%% %%%%% %%%%% %%%%% %%%%% %%%%%
\subsubsection{\ene 壬}\label{壬} %%% hg 146
\index[esquinas]{$2010_{4}$!{\ene 壬}} \index[fon]{reb!{\ene 壬}}
{\Large 4 [33, 1]  \textbf{\begin{tabular}{|c|} \hline 
rén\\ \hline \end{tabular} } } \\% \par
%%%%% %%%%% %%%%% %%%%% %%%%% %%%%% %%%%% %%%%%
\subsubsection{\ene 声}\label{声} %%% gah  385
\index[esquinas]{$4020_{7}$!{\ene 声}}\index[fon]{sheng!{\ene 声}}
{\Large 7 [33, 4]}
\abreviacion\ \enlugarde\ {\ene 聲 } \vease\ \textnumero\ ref. {\ene 聲}. 
%%%%% %%%%% %%%%% %%%%% %%%%% %%%%% %%%%% %%%%%
\setcounter{subsss}{\value{subsubsection}} dfg
\subsection{\ene  夂} \label{treintaycuatros} 
\setcounter{subsubsection}{\arabic{subsss}} %\input{1trazo/dos} 
%%%%% %%%%% %%%%% %%%%% %%%%% %%%%% %%%%% %%%%%
\subsubsection{\ene 条}\label{条} %%% hed2 830
%\index[esquinas]{$1234_{7}$!{\ene 条}}
\index[fon]{tiao!{\ene 条}}
{\Large 7 [34, 4] }
\abreviacion\ \enlugarde\ {\ene 條 } \vease\ \textnumero\ ref. {\ene 條}. 
%%%%% %%%%% %%%%% %%%%% %%%%% %%%%% %%%%% %%%%%
\subsubsection{\ene 复}\label{复} %%% oahe 复
%\index[esquinas]{$1234_{7}$!{\ene 复}}
\index[fon]{fu!{\ene 复}}
{\Large 9 [34, 6] }
\abreviacion\ \enlugarde\ {\ene 復} \vease\ \textnumero\ ref. {\ene 復}. 
%%%%% %%%%% %%%%% %%%%% %%%%% %%%%% %%%%% %%%%%
\subsubsection*{\ene 复}%\label{复} %%% rc 667
%\index[esquinas]{$1234_{7}$!{\ene 复}}
\index[fon]{fu!{\ene 复}}
{\Large 9 [34, 6] }
\abreviacion\ \enlugarde\ {\ene 複} \vease\ \textnumero\ ref. {\ene 複}. 
%%%%% %%%%% %%%%% %%%%% %%%%% %%%%% %%%%% %%%%%
\subsubsection*{\ene 复}%\label{复} %%% rc 667
%\index[esquinas]{$1234_{7}$!{\ene 复}}
\index[fon]{fu!{\ene 复}}
{\Large 9 [34, 6] }
\abreviacion\ \enlugarde\ {\ene 覆} \vease\ \textnumero\ ref. {\ene 覆}. 
%%%%% %%%%% %%%%% %%%%% %%%%% %%%%% %%%%% %%%%%
\setcounter{subsss}{\value{subsubsection}} dfg3
\subsection{\ene  夊 } \label{treintaycincos}f
\setcounter{subsubsection}{\arabic{subsss}} %\input{1trazo/dos} 
%%%%% %%%%% %%%%% %%%%% %%%%% %%%%% %%%%% %%%%%
\subsubsection{\ene 夊}\label{夊} %%% he3
%\index[esquinas]{$1234_{0}$!{\ene 夊}}
\index[fon]{zhi!{\ene 夊}}
{\Large 3 [35, 0]  \textbf{\begin{tabular}{|c|} \hline 
zh\v{\i}\\ \hline \end{tabular} } } \\% \par
%%%%% %%%%% %%%%% %%%%% %%%%% %%%%% %%%%% %%%%%
\subsubsection{\ene 夏}\label{夏} %%% muhe 6
\index[esquinas]{$1022_{4}$!{\ene 夏}} \index[fon]{xia!{\ene 夏}}
{\Large 10 [35, 7]  \textbf{\begin{tabular}{|c|} \hline 
xià\\ \hline \end{tabular} } } \\% \par
%%%%% %%%%% %%%%% %%%%% %%%%% %%%%% %%%%% %%%%%
\setcounter{subsss}{\value{subsubsection}} dfg
\subsection{\ene  夕} \label{treintayseiss}g
\setcounter{subsubsection}{\arabic{subsss}} %\input{1trazo/dos} 
%%%%% %%%%% %%%%% %%%%% %%%%% %%%%% %%%%% %%%%%
\subsubsection{\ene 夕}\label{夕} %%% ni 387
\index[esquinas]{$2720_{0}$!{\ene 夕}} \index[fon]{x!{\ene 夕}}
{\Large 36 [3, 0]  \textbf{\begin{tabular}{|c|c|} \hline 
xì & x\={\i}\\ \hline \end{tabular} } } \\% \par
%%%%% %%%%% %%%%% %%%%% %%%%% %%%%% %%%%% %%%%%

%%%%% %%%%% %%%%% %%%%% %%%%% %%%%% %%%%% %%%%%
\subsubsection{\ene 多}\label{多} %%% nini 389
\index[esquinas]{$2720_{7}$!{\ene 多}} \index[fon]{duo!{\ene 多}}
{\Large 6 [36, 3]  \textbf{\begin{tabular}{|c|c|} \hline 
du\={o} & duó\\ \hline \end{tabular} } } \\% \par
%%%%% %%%%% %%%%% %%%%% %%%%% %%%%% %%%%% %%%%%
\subsubsection{\ene 夜}\label{夜} %%% yonk 29
\index[esquinas]{$1234_{0}$!{\ene 夜}} \index[fon]{ye!{\ene 夜}}
\index[fon]{yi!{\ene 夜}}
{\Large 8 [36, 5]  \textbf{\begin{tabular}{|c|c|} \hline 
yè & yì\\ \hline \end{tabular} } } \\% \par
%%%%% %%%%% %%%%% %%%%% %%%%% %%%%% %%%%% %%%%%
\subsubsection{\ene 梦} %%%% ddni 梦 395
\index[esquinas]{$4420_{7}$!{\ene 梦}} \index[fon]{meng!{\ene 梦}}
{\Large 11 [36, 8] }
\abreviacion\ \enlugarde\ {\ene 夢 } \vease\ \textnumero\ \ref{夢}. 
%%%%% %%%%% %%%%% %%%%% %%%%% %%%%% %%%%% %%%%%
\subsubsection{\ene 夠}\label{夠} %%% nnpr 449xo
\index[esquinas]{$2722_{0}$!{\ene 夠}} \index[fon]{gou!{\ene 夠}}
{\Large 11 [36, 8]  \textbf{\begin{tabular}{|c|} \hline 
gòu\\ \hline \end{tabular} } } \\% \par
%%%%% %%%%% %%%%% %%%%% %%%%% %%%%% %%%%% %%%%%
\subsubsection{\ene 夢}\label{夢} %%%% 夢 twln 388
\index[esquinas]{$4420_{7}$!{\ene 夢}} \index[fon]{meng!{\ene 夢}}
{\Large 14 [36, 11]  \textbf{\begin{tabular}{| c|} \hline 
mèng\\ \hline \end{tabular} } } %\\ \par
\begin{enumerate} [noitemsep,label=\Roman{enumi}.,ref=\Roman{enumi}, leftmargin=*]
\item  \textit{sustantivo} sueño, ensoñación.
\item \textit{verbo} ver en sueños.
\end{enumerate}
%sustantivo. sueño, ensoñación.\\
\entry{夢中}{mèng-zh\={o}ng}{}{en sueños.}
\entry{夢中夢}{mèng-zh\={o}ng-mèng}{}{sueño en el sueño.}
%%%%% %%%%% %%%%% %%%%% %%%%% %%%%% %%%%% %%%%%
\setcounter{subsss}{\value{subsubsection}}
\subsection{\ene  大} \label{treintaysietes}h
\setcounter{subsubsection}{\arabic{subsss}} %\input{1trazo/dos} 
%%%%% %%%%% %%%%% %%%%% %%%%% %%%%% %%%%% %%%%%
\subsubsection{\ene 大}\label{大} %%% k 618
\index[esquinas]{$4080_{0}$!{\ene 大}} \index[fon]{da!{\ene 大}}
\index[fon]{dai!{\ene 大}}\index[fon]{tai!{\ene 大}}
{\Large 3 [37, 0]  \textbf{\begin{tabular}{|c|c|c|} \hline 
dà & dài & tài\\ \hline \end{tabular} } } \\% \par
%%%%% %%%%% %%%%% %%%%% %%%%% %%%%% %%%%% %%%%%
\subsubsection{\ene 太}\label{太} %%% ki 644
\index[esquinas]{$4003_{0}$!{\ene 太}} \index[fon]{tai!{\ene 太}}
{\Large 4 [37, 1]  \textbf{\begin{tabular}{|c|} \hline 
 tài\\ \hline \end{tabular} } } \\% \par
%%%%% %%%%% %%%%% %%%%% %%%%% %%%%% %%%%% %%%%%
\subsubsection{\ene 天}\label{天} %%% 天 mk 667
\index[esquinas]{$1080_{4}$!{\ene 天}} \index[fon]{tian!{\ene 天}}
{\Large 4 [37, 1]  \textbf{\begin{tabular}{|c|} \hline 
 ti\={a}n\\ \hline \end{tabular} } } \\% \par
%%%%% %%%%% %%%%% %%%%% %%%%% %%%%% %%%%% %%%%%
\subsubsection{\ene 夫}\label{夫} %%% qo 681
\index[esquinas]{$5080_{4}$!{\ene 夫}} \index[fon]{fu!{\ene 夫}}
{\Large 4 [37, 1]  \textbf{\begin{tabular}{|c|c|} \hline 
f\={u} & fú\\ \hline \end{tabular} } } \\% \par
%%%%% %%%%% %%%%% %%%%% %%%%% %%%%% %%%%% %%%%%
\subsubsection{\ene 头}\label{头} %%% yk 621
\index[esquinas]{$3480_{0}$!{\ene 头}}
\index[fon]{tou!{\ene 头}}%\index[fon]{tou!{\ene 头}}
{\Large 5 [37, 2] }
\abreviacion\ \enlugarde\ {\ene 頭 } \vease\ \textnumero\ ref. {\ene 頭}. 
%%%%% %%%%% %%%%% %%%%% %%%%% %%%%% %%%%% %%%%% mtmbc
\subsubsection{\ene 买}\label{买} %%% nyk 621
\index[esquinas]{$1780_{4}$!{\ene 买}}\index[fon]{mai!{\ene 买}}
{\Large 6 [37, 3] }
\abreviacion\ \enlugarde\ {\ene 買 } \vease\ \textnumero\ ref. {\ene 買}. 
%%%%% %%%%% %%%%% %%%%% %%%%% %%%%% %%%%% %%%%% wlb
\subsubsection{\ene 夹}\label{夹} %%% kt2 683
%\index[esquinas]{$1234_{7}$!{\ene 夹}}
\index[fon]{jia!{\ene 夹}}\index[fon]{xia!{\ene 夹}}
\index[fon]{xie!{\ene 夹}}\index[fon]{xie!{\ene 夹}}
\index[fon]{ga!{\ene 夹}}
{\Large 6 [37, 3] }
\abreviacion\ \enlugarde\ {\ene 夾 } \vease\ \textnumero\ \ref{夾}. 
%%%%% %%%%% %%%%% %%%%% %%%%% %%%%% %%%%% %%%%%
\subsubsection{\ene 夾}\label{夾} %%% koo 652
\index[esquinas]{$4003_{8}$!{\ene 夾}}
\index[fon]{jia!{\ene 夾}}\index[fon]{xia!{\ene 夾}}
\index[fon]{xie!{\ene 夾}}\index[fon]{xie!{\ene 夾}}
\index[fon]{ga!{\ene 夾}}
{\Large 7 [37, 4]  \textbf{\begin{tabular}{|c|c|c|} \hline 
ji\={a} & jiá & xiá\\ \hline \end{tabular} } } \\% \par
{\Large 7 [37, 4]  \textbf{\begin{tabular}{|c|c|c|} \hline 
xiè & g\={a}\\ \hline \end{tabular} } } \\% \par
%%%%% %%%%% %%%%% %%%%% %%%%% %%%%% %%%%% %%%%%

%%%%% %%%%% %%%%% %%%%% %%%%% %%%%% %%%%% %%%%%
\subsubsection{\ene 套}\label{套} %%% ksmi 947
\index[esquinas]{$4073_{2}$!{\ene 套}} \index[fon]{tao!{\ene 套}}
{\Large 10 [37, 7]  \textbf{\begin{tabular}{|c|} \hline 
tào\\ \hline \end{tabular} } } \\% \par
%%%%% %%%%% %%%%% %%%%% %%%%% %%%%% %%%%% %%%%%
\setcounter{subsss}{\value{subsubsection}} dfg
\subsection{\ene  女} \label{treintayochos}i
\setcounter{subsubsection}{\arabic{subsss}} %
\subsubsection{\ene 姓}\label{姓} %%% vhq 194
\index[esquinas]{$4541_{0}$!{\ene 姓}} \index[fon]{xing!{\ene 姓}}
{\Large 8 [38, 5]  \textbf{\begin{tabular}{| c|} \hline 
xìng\\ \hline \end{tabular} } } \\ %\par 
\textit{sustantivo}  {\begin{enumerate*}[label=\textbf{\arabic*})]
\item  familia; \item linaje. \end{enumerate*}}
\subsubsection{\ene 娶}\label{娶} %%% sev 娶 1043
\index[esquinas]{$1740_{4}$!{\ene 娶}} \index[fon]{zhi!{\ene 娶 }}
{\Large 11 [38, 8]  \textbf{\begin{tabular}{|c | c | c|} \hline 
qù & {\tiny \textit{hablado}} & q\v{u} \\ \hline \end{tabular} } } \\ \par
{\begin{enumerate*}[label=\textbf{\arabic*})]
\item \textit{verbo} casarse; \item pedir la mano, pedir matrimonio.
\end{enumerate*}}
\subsubsection{ff}sdf

\setcounter{subsss}{\value{subsubsection}} dfg
\subsection{\ene  子} \label{treintaynueves}j
\setcounter{subsubsection}{\arabic{subsss}} %\input{1trazo/dos} 

\subsubsection{a}sdf
\subsubsection{d}sdf
\subsubsection{f}sdf
\subsubsection{ff}sdf
\setcounter{subsss}{\value{subsubsection}} dfg
\subsection{\ene  宀} \label{cuarentas}k
\setcounter{subsubsection}{\arabic{subsss}} %\input{1trazo/dos} 

\subsubsection{a}sdf
\subsubsection{d}sdf
\subsubsection{f}sdf
\subsubsection{ff}sdf
\setcounter{subsss}{\value{subsubsection}} dfg
\subsection{\ene  寸} \label{cuarentayunos}t
\setcounter{subsubsection}{\arabic{subsss}} %\input{1trazo/dos} 
\subsubsection{\ene 將}\label{將} %%% vmbdi 62
\index[esquinas]{$2724_{7}$!{\ene 將}} \index[fon]{jiang!{\ene 將}}
{\Large 5 [30, 2]  \textbf{\begin{tabular}{|c | c | c|} \hline 
ji\={a}ng & jiàng & qi\={a}ng\\ \hline \end{tabular} } } %\\ \par
\begin{enumerate} [noitemsep,label=\Roman{enumi}.,ref=\Roman{enumi}, leftmargin=*]
\item \textit{sustantivo} {\begin{enumerate*}[label=\textbf{\arabic*})]
\item \textbf{jiàng} general; comandante;
\item \textbf{jiàng} \textit{ajedrez} rey.
\end{enumerate*}}
\item \textit{verbo} comandar; dirigir.
\end{enumerate}
\entry{將要}{zhi\={a}ng'yào}{}{en el futuro; al rato.}
\entry{要將}{yàozhi\={a}ng'}{}{militar (importante) de alto rango.}
\subsubsection{ff}sdf
\setcounter{subsss}{\value{subsubsection}} dfg
\subsection{\ene  小} \label{cuarentaydoss} we
\setcounter{subsubsection}{\arabic{subsss}} %\input{1trazo/dos} 
otro

\setcounter{subsss}{\value{subsubsection}} dfgr
\subsection{\ene  尢 (丌, 尣)} \label{cuarentaytress} 
\setcounter{subsubsection}{\arabic{subsss}} %\input{1trazo/dos} 
\subsubsection{\ene 就}\label{就} %%% yfiku 501
\index[esquinas]{$0391_{2}$!{\ene 就}} \index[fon]{jiu!{\ene 就}}
{\Large 12 [43, 9]  \textbf{\begin{tabular}{ | c|} \hline 
jiù \\ \hline \end{tabular} } } \\% \par
\begin{enumerate} [noitemsep,label=\Roman{enumi}.,ref=\Roman{enumi}, leftmargin=*]
\item   \textit{verbo y verbo-preposición} {\begin{enumerate*}[label=\textbf{\arabic*})]
\item tiene varios significados продвигаться (устремляться) к,
\item занимать, принимать ocuparse, tomar.
\end{enumerate*}}
\item \textit{adverbio} {\begin{enumerate*}[label=\textbf{\arabic*})]
\item entonces, así pues; \item (\textit{también} {是就}) solamente, sólo si.
\end{enumerate*}}
\item \textit{conjunción}
\end{enumerate}
\setcounter{subsss}{\value{subsubsection}} dfghtere
\subsection{\ene  尸} \label{cuarentaycuatros} 
\setcounter{subsubsection}{\arabic{subsss}} %\input{1trazo/dos} 

\subsubsection{a}sdf
\subsubsection{d}sdf
\subsubsection{f}sdf
\subsubsection{ff}sdf
\setcounter{subsss}{\value{subsubsection}} dfghyee
\subsection{\ene  屮} \label{cuarentaycincos} \setcounter{subsubsection}{\arabic{subsss}} %\input{1trazo/dos} 

\subsubsection{a}sdf
\subsubsection{d}sdf
\subsubsection{f}sdf
\subsubsection{ff}sdf
\setcounter{subsss}{\value{subsubsection}} dfg
hyety
\subsection{\ene  山} \label{cuarentayseiss}aqer
\setcounter{subsubsection}{\arabic{subsss}} %\input{1trazo/dos} 

\subsubsection{a}sdf
\subsubsection{d}sdf
\subsubsection{f}sdf
\subsubsection{ff}sdf
\setcounter{subsss}{\value{subsubsection}} dfg
\subsection{\ene  巛} \label{cuarentaysietes} 
\setcounter{subsubsection}{\arabic{subsss}} %\input{1trazo/dos} 

\subsubsection{a}sdf
\subsubsection{d}sdf
\subsubsection{f}sdf
\subsubsection{ff}sdf
\setcounter{subsss}{\value{subsubsection}} dfghter
\subsection{\ene 工} \label{cuarentayochos} ef
\setcounter{subsubsection}{\arabic{subsss}} %\input{1trazo/dos} 

\subsubsection{a}sdf
\subsubsection{d}sdf
\subsubsection{f}sdf
\subsubsection{ff}sdf
\setcounter{subsss}{\value{subsubsection}} dfggh ocho bits
\subsection{\ene  己 (已, 巳)} \label{cuarentaynueves} 
\setcounter{subsubsection}{\arabic{subsss}} %\input{1trazo/dos} 
%\subsection{\ene 己 }\label{cientotreintaydoss}\setcounter{subsubsection}{\arabic{subsss}} %
\subsubsection{\ene 己}\label{己} %%% su 335
\index[esquinas]{$1771_{7}$!{\ene 己}} \index[fon]{ji!{\ene 己}} \index[fon]{qi!{\ene 己}}
{\Large 3 [49, 0]  \textbf{\begin{tabular}{|c | c|} \hline 
j\v{\i} & q\v{\i}\\ \hline \end{tabular} } } %\\ \par
\begin{enumerate} [noitemsep,label=\Roman{enumi}.,ref=\Roman{enumi}, leftmargin=*]
\item   \textit{pronombre (reflexivo y posesivo)} me, mí; te, tí; se, sí; conmigo, consigo.  
\item \textit{verbo} organizar, poner en orden.
\end{enumerate} %%% su 己
\entry{己見}{j\v{\i}jiàn}{}{punto de vista, perspectiva personal.}
\setcounter{subsss}{\value{subsubsection}} dfgsdg
\subsection{\ene  巾} \label{cincuentas}df cua
\setcounter{subsubsection}{\arabic{subsss}} %\input{1trazo/dos} 

\subsubsection{a}sdf
\subsubsection{d}sdf
\subsubsection{f}sdf
\subsubsection{ff}sdf
\setcounter{subsss}{\value{subsubsection}} dfgrto bye
\subsection{\ene  干} \label{cincuentayunos} a
\setcounter{subsubsection}{\arabic{subsss}} %\input{1trazo/dos} 

\subsubsection{a}sdf
\subsubsection{d}sdf
\subsubsection{f}sdf
\subsubsection{ff}sdf
\setcounter{subsss}{\value{subsubsection}} dfgsdf 
\subsection{\ene  幺} \label{cincuentaydoss} t
\setcounter{subsubsection}{\arabic{subsss}} %\input{1trazo/dos} 

\subsubsection{a}sdf
\subsubsection{\ene 幾}\label{幾} %\IN{er!{\enese 冬}} %% hey
\index[esquinas]{$2225_{3}$!{\ene 幾}}\index[fon]{ji!{\ene 幾}}
{\Large 3 [15, 3]  \textbf{\begin{tabular}{|c | c | c | c | } \hline %\large
j\={\i} &j\v{\i} &q\v{\i}  & jì \\ \hline \end{tabular} } }% \\ \par
\begin{enumerate}
\item \textit{numeral}  {\begin{enumerate*}[label=\textbf{\arabic*})]
\item  \textbf{jì} algo, alguno; \item  \textbf{j\v{\i}} ¿cuánto?
\end{enumerate*}}
\item \textit{palabra interrogativa en el lenguaje literario} {\begin{enumerate*}[label=\textbf{\arabic*})]
\item  ¿cuál? \item par.
\end{enumerate*}}
\end{enumerate}
\subsubsection{ff}sdf
\setcounter{subsss}{\value{subsubsection}} dfgreqd
\subsection{\ene  广} \label{cincuentaytress} 
\setcounter{subsubsection}{\arabic{subsss}} %\input{1trazo/dos} 
\subsubsection{\ene 应}\label{应} %%% pmf 600\index[esquinas]{$00??_{0}$!{\ene 应}}
 \index[fon]{ying!{\ene 应}} 
{\Large 7 [53, 4]} \abreviacion\ \enlugarde\ {\ene  應} \vease\ \textnumero\ \ref{應}.%pywv 

\setcounter{subsss}{\value{subsubsection}} dfggdsfg 
\subsection{\ene  廴} \label{cincuentaycuatros}d
\setcounter{subsubsection}{\arabic{subsss}} %\input{1trazo/dos} 

\subsubsection{a}sdf
\subsubsection{d}sdf
\subsubsection{f}sdf
\subsubsection{ff}sdf
\setcounter{subsss}{\value{subsubsection}} dfgfh 
\subsection{\ene  廾} \label{cincuentaycincos}
\setcounter{subsubsection}{\arabic{subsss}} %\input{1trazo/dos} 

\subsubsection{a}sdf
\subsubsection{d}sdf
\subsubsection{f}sdf
\subsubsection{ff}sdf
\setcounter{subsss}{\value{subsubsection}} dfghyte
\subsection{\ene  弋} \label{cincuentayseiss}f
\setcounter{subsubsection}{\arabic{subsss}} %\input{1trazo/dos} 

\subsubsection{a}sdf
\subsubsection{d}sdf
\subsubsection{f}sdf
\subsubsection{ff}sdf
\setcounter{subsss}{\value{subsubsection}} dfgfdfgh
\subsection{\ene  弓} \label{cincuentaysietes} 
\setcounter{subsubsection}{\arabic{subsss}} %\input{1trazo/dos} 

\subsubsection{a}sdf
\subsubsection{d}sdf
\subsubsection{f}sdf
\subsubsection{ff}sdf
\setcounter{subsss}{\value{subsubsection}} dfggfds fgff
\subsection{\ene  彐 (⺕, 彑)} \label{cincuentasyochos}  
\setcounter{subsubsection}{\arabic{subsss}} %\input{1trazo/dos} 

\subsubsection{a}sdf
\subsubsection{d}sdf
\subsubsection{f}sdf
\subsubsection{ff}sdf
\setcounter{subsss}{\value{subsubsection}} dfg
fdsrt 
\subsection{\ene  彡} \label{cincuentaynueves} sadg
\setcounter{subsubsection}{\arabic{subsss}} 

\subsubsection{a}sdf
\subsubsection{d}sdf
\subsubsection{f}sdf
\subsubsection{ff}sdf
\setcounter{subsss}{\value{subsubsection}} dfg
\subsection{\ene  彳} \label{sesentas}sdf
\setcounter{subsubsection}{\arabic{subsss}} %\input{1trazo/dos} 
\subsubsection{  \ene 從}\label{從} %%% hoo 940
\index[esquinas]{$2828_{1}$!{\ene 從}} \index[fon]{cong!{\ene 從}} 
\index[fon]{zong!{\ene 從}} 
{\Large 11 [60, 8] \textbf{\begin{tabular}{|c | c | c|} \hline 
cóng & zòng & z\={o}ng\\ \hline \end{tabular} } }\\
\textit{en compuestos también:}{\Large \textbf{\begin{tabular}{| c | c|} \hline 
c\={o}ng& z\v{o}ng \\ \hline \end{tabular} } }
 %\\% \par} \\% \par
\begin{enumerate} [noitemsep,label=\Roman{enumi}.,ref=\Roman{enumi}, leftmargin=*]
\item  \textbf{cóng} \textit{verbo} {\begin{enumerate*}[label=\textbf{\arabic*})]
\item seguir; \item no sé \end{enumerate*}}
\item \textbf{cóng}  \textit{verbo-preposición} {\begin{enumerate*}[label=\textbf{\arabic*})]
\item \textit{ante todo no sé}
\end{enumerate*}}
\item \textit{adjetivo/adverbio} {\begin{enumerate*}[label=\textbf{\arabic*})]
\item \textbf{zòng}   \textit{ante todo no sé}
\end{enumerate*}}
\end{enumerate}
\setcounter{subsss}{\value{subsubsection}} dfg
%\end{multicols}