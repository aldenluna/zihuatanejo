\subsection{\ene 一 }\label{unos} %\input{1trazo/uno}
\subsubsection{\ene 一}\label{一}
\index[esquinas]{$1000$!{\ene 一}} \index[fon]{yi!{\ene 一}}
 {\Large 1 [1,0] \textbf{ \begin{tabular}{|c | } \hline %
y\={\i}  \\ \hline
\end{tabular} } }%%
\begin{enumerate} [noitemsep,label=\Roman{enumi}.,ref=\Roman{enumi}, leftmargin=*]
\item \textit{numeral/adjetivo/adverbio}  {\begin{enumerate*}[label=\textbf{\arabic*})]
\item  uno, 1; \item primero; \item único, unitario.
\end{enumerate*}}
\item \textit{conjunción} {\begin{enumerate*}[label=\textbf{\arabic*})]
\item  ocasión; cada vez; \item como cuando.
\end{enumerate*}}
\end{enumerate}
\entry{一切}{y\={\i}qiè'}{}{todo, entero, global}

\subsubsection{\ene 丄}\label{丄}
\index[esquinas]{$2010_{0}$!{\ene 丄}} \index[fon]{shang!{\ene 丄}}
{\Large 2 [1,1]} \** % \begin{tabular}{|c | c| } \%пень & нем \\ \hline\end{tabular}
\enlugarde\ {\ene 上} \vease\ \textnumero\ \ref{上}% {\ene\ 上 例} omnn N. 


\subsubsection{\ene 丅} \label{丅} 
\index[esquinas]{$1020_{0}$!{\ene 丅}}\index[fon]{xie!{\ene 丅}}
{\Large 2 [1,1]} \enlugarde\ {\ene 下}
\vease\ \textnumero\ \ref{下}% {\ene\ 上 例} omnn N. 

\subsubsection{\ene 丁}\label{丁} 
\index[esquinas]{$1020_{0}$!{\ene 丁}}
\index[fon]{ding!{\ene 丁}}\index[fon]{zheng!{\ene 丁}}
{\Large 2 [1,1] \textbf{ \begin{tabular}{|c | } \hline %\large
 d\v{\i}ng\\ \hline\end{tabular} }}
\textit{en repetición} {\Large \textbf{ \begin{tabular}{|c | } \hline
 zh\={e}ng\\ \hline\end{tabular} } }%\par
trabajador, mayor de edad. \par
Exponer, empujar. \par


% \subsubsection{\ene  ㄘ} %%%% yyyxt  ㄘ
 %{\Large 2 [1, 1]}  \enlugarde\ {\ene 七  } \vease\ \textnumero\ ref.; \textit{ no confundir con el símbolo del alfabeto zhuyin}
 
\subsubsection{\ene 七}\label{七} %%% ju 七
\index[esquinas]{$4070_{1}$!{\ene 七}}\index[fon]{qi!{\ene 七}}
{\Large 2 [1,1] \textbf{ \begin{tabular}{|c| } \hline
q\={\i} \\ \hline \end{tabular} } } \textit{también}\ 
{\Large \begin{tabular}{|c| } \hline
qí \\ \hline\end{tabular}} \par
Siete; 7.\\
\entry{七一}{q\={\i}y\={\i}}{} {Primer número del séptimo mes lunar; 
1 de Julio (Día de la fundación del Partido Comunista Chino, 1921)}	

\subsubsection{\ene 三} \label{三}
\index[esquinasl]{$1010_{1}$!{\ene 三}}\index[fon]{san!{\ene 三}}
{\Large 3 [1, 2] \textbf{\begin{tabular}{|c| c | c | } \hline %\large
s\={a}n & s\`{a}n & s\={a}  \\ \hline
\end{tabular} } } \\ \par% \marginalnote{xxx}

\textbf{s\={a}n} tres; 3; {\ene 三人合作} tres personas trabajaron en colaboración;
{\ene 三十三} 33.\par 

\subsubsection{\ene 上}\label{上}
\index[esquinas]{$1010_{9}$!{\ene 上}}\index[fon]{shang!{\ene 上}}
{\Large 3 [1, 2] \textbf{\begin{tabular}{|c | c | c | } \hline 
sh\`{a}ng & sh\v{a}ng & -sh\r{a}ng \\ \hline
\end{tabular}} }\\ \par
I. \textbf{sh\`{a}ng} Superior, arriba, encima.
\entry{上晝}{shàng-zhòu}{}{al mediodía.}
\entry{上上}{shàng'shàng}{}{óptimo, el mejor; la primera (más favorable) de diez posibilidades; lo más ventajoso.}
%\entry{上工}{shàng g\ouno ng}{}{ir al trabajo; comenzar el trabajo.}

\subsubsection{\ene 下} \label{下}
\index[esquinas]{$1023_{0}$!{\ene 下}}\index[fon]{xia!{\ene 下}}
{\Large 3 [1, 2] \textbf{ \begin{tabular}{|c | c| } \hline %\large
xi\`{a}& -xi\r{a}\\ \hline \end{tabular} } }\\ \par
I \textbf{xi\`{a}} \emph{adj./conv.} \textbf{1}) Inferior, abajo; {\ene 門牙} incisivos inferiores; {\ene 下視} mirar [de arriba] abajo.
\entry{下晝}{xiàzhòu}{}{segunda mitad del día, después del mediodía, después de comer.}
\entry{下江}{xiàji\={a}ng'}{}{bajo Yang-Tse.}
\entry{下雪}{xiàxiè}{}{nevada}
\entry{下世}{xià-shì}{}{ ir al otro mundo, morir; dos\ el siguiente siglo, vida futura; tres\ }
\entry{下日}{xià'rì}{}{uno\ [en el] siguiente día; por la mañana; dos\ en breve, }

\subsubsection{\ene 兀}%\label{兀} 
\index[esquinas]{$1022_{0}$!{\ene 兀}}\index[fon]{ji!{\ene 兀}}
{\Large 3 [1, 2] \textbf{ \begin{tabular}{| c | } \hline
j\={\i}  \\ \hline
\end{tabular}}} \\ \par
Soporte.\par

\subsubsection{\ene 丈}\label{丈}
\index[esquinas]{$5000_{0}$!{\ene 丈}}\index[fon]{zhang!{\ene 丈}}
{\Large  3 [1, 2] \textbf{ \begin{tabular}{| c | } \hline %
zhàng \\ \hline \end{tabular} }}\\ \par

\subsubsection{\ene 车}\label{车} %%% 车 jq
\index[esquinas]{$4050_{0}$!{\ene 车}}
\index[fon]{che!{\ene 车}}\index[fon]{ju!{\ene 车}}
 {\Large 4 [1, 3], [159, 0]} \abreviacion\ \enlugarde\ {\ene 車} \vease\ \textnumero\ ref. 

\subsubsection{\ene 开}%\label{开}
\index[esquinas]{$1044_{0}$!{\ene 开}}\index[fon]{kai!{\ene 开}}
4 [1, 3], [55, 1] \abreviacion\ \enlugarde\ {\ene 開} \vease\  \textnumero\ ref.

\subsubsection{\ene 冇}\label{冇}%xkb
\index[esquinas]{$4022_{7}$!{\ene 冇}}\index[fon]{mou!{\ene 冇}}
{\Large 4 [1, 3]  \textbf{\begin{tabular}{|c | } \hline 
m\v{o} u  \\ \hline\end{tabular} } } \\ \par
negación

%\subsubsection{  \ene 韦}\label{韦} 
%\index[esquinas]{$5002_{7}$!{\ene 韦}}\index[fon]{wei!{\ene 韦}} 
%4 [1, 3] \abreviacion\ \enlugarde\ {\ene 韋} \vease\ \textnumero\ ref. %\vease\ 

\subsubsection{\ene 乌}\label{乌}% pvsm
\index[esquinas]{$2712_{7}$!{\ene 乌}}\index[fon]{wu!{\ene 乌}}
4 [1, 3] \abreviacion\ \enlugarde\ {\ene 鳥} %hayf
 \vease\ \textnumero\ ref.  

\subsubsection{\ene 不}\label{不}
\index[esquinas]{$1090_{0}$!{\ene 不}}\index[fon]{bu!{\ene 不}} 
{\Large 4 [1, 3] \textbf{ \begin{tabular}{| c | } \hline %\large
bù \\ \hline \end{tabular}} }
\begin{tabular}{c}
\textit{se lee en fuentes antiguas con}\\
\textit{frecuencia reemplazado por:}\\
\end{tabular}    \par {\Large
\textbf{ \begin{tabular}{| c c | c | c | c |} \hline
f\otres u & f\ouno u & f\etres i & p\euno i & b\iuno \\ \hline
\end{tabular}} } \\ \par
\begin{tabular}{c}
\textit{en nombres propios}\\
\textit{ se lee también:}\\
\end{tabular} {\Large \textbf{ \begin{tabular}{| c | } \hline
biáo \\ \hline
\end{tabular} } }\\ \par
I. \textit{negación universal}

\subsubsection{\ene 丕}\label{丕}  
\index[esquinas]{$1010_{9}$!{\ene 丕}}\index[fon]{pi!{\ene 丕}}
{\Large 5 [1, 4], 6 [7, 4] \textbf{ \begin{tabular}{|c | } \hline
p\iuno  \\ \hline\end{tabular}} }\\ \par
inmenso, grande, gigante; mucho, fuerte; muy, demasiado; del todo; excesivo; desmesurado.\par
II tomar, recibir educación superior (completar); {丕命} recibir órdenes; aceptar mandato (Cielo).\par 
 peseta.\par 
\entry{丕塞地}{p\={\i} sài't\={a} }{}{peseta.}

\subsubsection{\ene 丘} \label{丘} 
\index[esquinas]{$7210_{2}$!{\ene 丘}}\index[fon]{qi!{\ene 丘}} 
{\Large 5 [1, 4] \textbf{ \begin{tabular}{| c | } \hline
qi\={u} \\ \hline \end{tabular} } }\\ \par


\subsubsection{\ene 业 } \label{业} 
\index[esquinas]{$3210_{0}$!{\ene 业}}\index[fon]{ye!{\ene 业}} 
{\Large 5 [1, 4]} \abreviacion\ \enlugarde\ {\ene 業 } \vease\ \textnumero\ ref.

\subsubsection{\ene 且} \label{且}
\index[esquinas]{$7710_{2}$!{\ene 且}}
\index[fon]{qi!{\ene 且}} \index[fon]{ju!{\ene 且}} 
{\Large 5 [1, 4] \textbf{ \begin{tabular}{| c | c | c | c | } \hline
qi\etres & j\uuno & qù & z\utres \\ \hline
\end{tabular}} }\\ \par
I. \textbf{qi\v{e}} 

\subsubsection{\ene 世} \label{世}% pt 世
\index[esquinas]{$4471_{7}$!{\ene 世}}\index[fon]{shi!{\ene 世}} 
 {\Large 5 [1, 4] \textbf{ \begin{tabular}{| c | } \hline
shì \\ \hline\end{tabular}} }\\ \par
generación, género; generaciones (\textit{pasada, presente y futura});  de generación en generación; tradición.
\entry{世紀}{shì'jì}{}{siglo, centenario.}
\subsubsection{\ene 丽}\label{丽}%%%% mmbb 丽
\index[esquinas]{$1022_{7}$!{\ene 丽}} \index[fon]{liang!{\ene 丽}}  
{\Large 7 [1, 6]} %\abreviacion\ \enlugarde\ {\ene 麗} \vease\ \textnumero\ ref.  %%%%麗 mmbbp
 
\subsubsection{\ene 两}\label{两} %%% moob
\index[esquinas]{$1022_{7}$!{\ene 两}} \index[fon]{liang!{\ene 两}} 
{\Large 7 [1, 6]} \abreviacion\ \enlugarde\ {\ene 兩 } \vease\ \textnumero\ ref. 

\subsubsection{\ene 严}\label{严} %%% mtch 严
\index[esquinas]{$1020_{1}$!{\ene 严}} \index[fon]{yan!{\ene 严}} 
{\Large 7 [1, 6]} \abreviacion\ \enlugarde\ {\ene 嚴 } \vease\ \textnumero\ ref. 
\textnumero\
\subsubsection{\ene 並} \label{並} %%% ttc
 \index[esquinas]{$8010_{2}$!{\ene 並}}\index[fon]{pang!{\ene 並}} 
\index[fon]{bing!{\ene 並}} 
 {\Large 8 [1, 7] \textbf{ \begin{tabular}{| c | c | } \hline %\large
bìng & pàng \\ \hline
\end{tabular}} }\\ \par
%I. \textbf{bìng} \verbo\ \uno\ estar en fila.
\setcounter{subsss}{\value{subsubsection}} 

\subsection{\ene 丨}\label{doss} 
\setcounter{subsubsection}{\arabic{subsss}} %\input{1trazo/dos} 
%\subsubsection{a}sdf
%\guillemotleft % \guillemotright
\subsubsection { \ene 丨}\label{丨} 
\index[esquinas]{$2000_{0}$!{\ene 丨}}\index[fon]{gun!{\ene 丨}}
{\Large 1 [2,0]\textbf{ \begin{tabular}{|c | } \hline
g\v{u}n  \\ \hline
\end{tabular} } }\\ \par
I. sustantivo\ uno\ palo, vara, bastón, varilla (\textit{trazo vertical axial en los ideogramas}).
II. CONVENCION: \textit{símbolo, que representa el jeroglífico inicial en las entradas de los artículos de los diccionarios chinos. } \vease\ ref. Paladio.

\subsubsection { \ene 丩}\label{丩} %%xxvl 
\index[esquinas]{$2270_{0}$!{\ene 丩}}\index[fon]{jiu!{\ene 丩}}
 {\Large 2 [2,1] \textbf{ \begin{tabular}{|c | } \hline
ji\={u}  \\ \hline
\end{tabular} } }\\ \par
\textit{observación: no confundir con el símbolo del alfabeto zhuyin} \textbf{j(i)} \par
verbo\*\  enredar, enrollar, envolver, encuadernar, entrelazar, trenzar, tejer.

\subsubsection { \ene 丫}\label{丫} %%%%%% cl
\index[esquinas]{$8020_{7}$!{\ene 丫}}\index[fon]{ya!{\ene 丫}}
 {\Large 3 [2,2]\textbf{ \begin{tabular}{|c | } \hline
y\auno  \\ \hline\end{tabular} } }\\ \par
\textit{observación: no confundir con el 25\textdegree\ símbolo del alfabeto zhuyin, correspondiente a la vocal} \textbf{\guillemotleft A\guillemotright} \textit{en la transcripción alfabética china y latina. }\par
sustantivo\ uno\ cuernito; ramificación, horcadura (POREJEMPLO\ \textit{árbol}); {\ene 丫桂} mesa con superficie bifurcada; {\ene 丫戾} sobresalir por un lado (cuernito); %%%%% 戾 hsik
dos\ manojo, haz, trenzas (\textit{peinado de niños, niñas}).

\subsubsection{\ene  个}\label{个}
\index[esquinas]{$8020_{0}$!{\ene 个}}
\index[fon]{ge!{\ene 个}}
{\Large 3 [2,2]\textbf{ \begin{tabular}{|c c | } \hline
gè & g\={e}  \\ \hline
\end{tabular} } }\\ \par
sustantivo\ uno\ \abreviacion\ \enlugarde\ {\ene 個} \vease\ \textnumero\ ref{個}.; dos\ \*\ ala de reposo, pabellón; {\ene 青陽右个} pabellón derecho de la sala Qinian.
\entry{个字}{gè'zì}{}{pintura\ símbolo ge (\textit{trazo para representación de hojas, POREJEMPLO\ de bambú}).}
 

\subsubsection{\ene 㐄}\label{㐄}% 㐄 jv2
\index[esquinas]{$5070_{0}$!{\ene 㐄}}\index[fon]{kua!{\ene 㐄}}
{\Large 3 [2, 2] \textbf{ \begin{tabular}{|c c | } \hline
ku\atres & kuà  \\ \hline\end{tabular} } }\\ \par
verbo\ dar pasos cortos; pasos cortos.

\subsubsection{\ene 丯}\label{丯} %%%% 丯 qj2
\index[esquinas]{$5000_{0}$!{\ene 丯}}\index[fon]{jie!{\ene 丯}}
{\Large 4 [2, 3] \textbf{ \begin{tabular}{|c | } \hline
jiè  \\ \hline\end{tabular} } }\\ \par
adjetivo\ embrollado, enmarañado, desordenado (\textit{sobre el pasto crecido}).


\subsubsection{\ene 丰}\label{丰} 
\index[esquinas]{$5000_{0}$!{\ene 丰}}\index[fon]{feng!{\ene 丰}}
{\Large 4 [2, 3] \begin{tabular}{|c | } \hline
f\euno ng  \\ \hline\end{tabular} }\\ \par
I. adjetivo\ uno\ bello, impecable, bien visto, atractivo (\textit{de exteriores}).

\subsubsection{\ene 中}\label{中} %4 [2, 3] 
\index[esquinas]{$5000_{6}$!{\ene 中}}
\index[fon]{zhong!{\ene 中}}
{\Large 4 [2, 3] \textbf{ \begin{tabular}{|c c | c |} \hline
zh\={o}ng & zhòng  & -zh\r{o}ng \\ \hline
\end{tabular} } }\\ \par
I. \textbf{zh\={o}ng} sustantivo\ uno\ centro, mitad;

%\subsubsection{\ene 韦} 
\subsubsection{  \ene 韦}\label{韦} 
\index[esquinas]{$5002_{7}$!{\ene 韦}}\index[fon]{wei!{\ene 韦}} 
4 [2, 3] \abreviacion\ \enlugarde\ {\ene 韋} \vease\ \textnumero\ ref.
%\vease\ 

\subsubsection{\ene 书}\label{书} %%% 书 ids
\index[esquinas]{$5302_{7}$!{\ene 书}}\index[fon]{shu!{\ene 书}} 
 {\Large 4 [2, 3]} \abreviacion\ \enlugarde\ {\ene 書} %% lga
 \vease\ \textnumero\ ref.

\subsubsection{\ene 丱}\label{丱}%%%%% vlllm 丱
\index[esquinas]{$2277_{0}$!{\ene 丱}}
\index[fon]{guan!{\ene 丱}} \index[fon]{kuang!{\ene 丱}} 
 {\Large 5 [2, 4] \textbf{ \begin{tabular}{|c | } \hline
guàn  \\ \hline
\end{tabular} } } {\Large \textbf{ \begin{tabular}{|c | } \hline
kuàng,  ku\v{a}ng  \\ \hline\end{tabular} } }\\ \par
sustantivo\ dos moños (en forma de cuerno) en la cabeza de los infantes; infantil; 
{\ene 丱齒} desarrollo infantil, infancia; %%%% ymuoo
\entry{丱童}{guàn-tóng}{}{niño.}%%%%% ytwg
\entry{丱女}{guàn-n\v{ü}}{}{niña.}

\subsubsection{\ene 串}\label{串} 
\index[esquinas]{$5000_{6}$!{\ene 串}}
\index[fon]{chuan!{\ene 串}} \index[fon]{guan!{\ene 串}} 
 {\Large 7 [2, 6] \textbf{ \begin{tabular}{| c | c |} \hline
 chuàn  & guàn \\ \hline\end{tabular} } }\\ \par
I. \textbf{chuàn} verbo\ uno enhebrar, ensartar; atravesar (pasar) de lado a lado;
{\ene 串錢} coser monedas en el estambre; {\ene 貫串} a) pasar a través; b) figurado\ superar, aprender de cabo a rabo; dos\ visitar, hacer una visita (a alguien).\\
\entry{串星}{liánx\={\i}ng'}{}{astronomia\ estrella doble.}
\entry{串聯}{chuànlián'}{}{uno\  electronica\ circuito en serie; dos\ acuerdo inmediato.} %%%% wjbuc 貫

\subsubsection{\ene 丳}\label{丳} %%% llww 丳
\index[esquinas]{$5500_{6}$!{\ene 丳}}\index[fon]{chan!{\ene 丳}} 
{\Large 8 [2, 7] \textbf{ \begin{tabular}{|c | } \hline
chàn  \\ \hline\end{tabular} } }\\ \par
sustantivo\ asador; {\ene 以⺼貫丳} poner la carne en el asador (brocheta).

\subsubsection{\ene 临}\label{临}%%%% lloa 临
\index[esquinas]{$2806_{3}$!{\ene 临}}\index[fon]{lin!{\ene 临}} 
  {\Large 9 [2, 8]} \abreviacion\ \enlugarde\ {\ene 臨} %% slorr 臨
 \vease\ \textnumero\ ref.

\subsubsection{ \ene 丵}\label{丵} %%%%%% tctj 丵
\index[esquinas]{$3240_{1}$!{\ene 丵}}\index[fon]{zhuo!{\ene 丵}}  
 {\Large 3 [2,2] \textbf{ \begin{tabular}{|c | } \hline
zhuó  \\ \hline\end{tabular} } }\\ \par
 sustantivo\ pasto espeso; maleza.
%\subsubsection{ff}sdf
\setcounter{subsss}{\value{subsubsection}} 

\subsection{\ene 丶}\label{tress}

\setcounter{subsubsection}{\arabic{subsss}} %\input{1trazo/dos} 
\subsubsection{\ene   丶} \label{丶} %% i2
\index[esquinas]{$3000_{0}$!{\ene 丶}} \index[fon]{zhu!{\ene 丶}} 
{\Large 1 [3, 0] \textbf{ \begin{tabular}{|c | } \hline
zh\v{u}  \\ \hline \end{tabular} } }\\ \par 
%\sustantivo\ \uno\ \*\ 
punto; %\dos\ \*\ llama luminosa.

\subsubsection{\ene   丸}\label{丸} %%% 丸 kni
\index[esquinas]{$5001_{7}$!{\ene 丸}} \index[fon]{wan!{\ene 丸}} 
{\Large 3 [3, 2] \textbf{ \begin{tabular}{|c | } \hline
wán  \\ \hline \end{tabular} } }\\ \par 
%I. \sustantivo/\CLASIFICADOR\ \uno\ bola; pelota; bala, núcleo;

\subsubsection{\ene    丹} \label{丹}%%% 丹 by
\index[esquinas]{$7744_{0}$!{\ene 丹}} \index[fon]{dan!{\ene 丹}} 
 {\Large 4 [3, 3] \textbf{ \begin{tabular}{|c | } \hline
d\={a}n  \\ \hline \end{tabular} } }\\ \par 
I.% \sustantivo\ \uno\
  cinabrio; bermellón; tinte rojo; rojo {\ene 丹色} color rojo; %\dos\ píldora; panacea; \daoismo\ piedra filosofal, píldora de la inmortalidad.

\subsubsection{\ene   为}\label{为} %%%% iksi 为 bhnf 爲 iknf 為
\index[esquinas]{$3402_{7}$!{\ene 为}}
\index[fon]{wei!{\ene 为}} 
 {\Large 4 [3, 3]} \abreviacion\ \enlugarde\ {\ene 爲} \vease\ \textnumero\ ref. 

\subsubsection{\ene   主} \label{主}% 主 yg
\index[esquinas]{$0010_{4}$!{\ene 主}} \index[fon]{zhu!{\ene 主}} 
{\Large 5 [3, 4] \textbf{ \begin{tabular}{|c | } \hline
zh\v{u}  \\ \hline \end{tabular} } }\\ \par
I.% \sustantivo\ \uno\
 dueño; patrón; propietario;

\subsubsection{\ene     乓}  \label{乓} %%% omi 乓
\index[esquinas]{$7203_{1}$!{\ene 乓}} \index[fon]{pang!{\ene 乓}} 
{\Large 6 [3, 4] \textbf{ \begin{tabular}{|c | } \hline
p\={a}ng  \\ \hline \end{tabular} } }\\ \par
\textit{solamente en compuestos;} \vease\ {\ene 乒乓} %%% omh

\subsubsection{\ene    举}{\label{举}}%fcq2 举 9 [3, 8] %%%% 舉 hcq
\index[esquinas]{$9050_{8}$!{\ene 挙}} \index[fon]{ju!{\ene 挙}} 
 {\Large 9 [3, 8]} \abreviacion\ \enlugarde\ {\ene 舉} \vease\ \textnumero\ ref. 
\setcounter{subsss}{\value{subsubsection}} 

\subsection{\ene 丿 乁  乀} \label{cuatros} 
\setcounter{subsubsection}{\arabic{subsss}} %\input{1trazo/dos} 

\subsubsection{\ene 丿}\label{丿}%  1 [4, 0] %%%% qfbk 撇
\index[esquinas]{$2000_{0}$!{\ene 丿}}\index[fon]{pie!{\ene 丿}} 
 {\Large 4 [3, 3]\textbf{ \begin{tabular}{|c | } \hline
pi\v{e}  \\ \hline\end{tabular} } }\\ \par 
\enlugarde\ {\ene 撇} \textit{trazo caligráfico plegado de derecha a izquierda}

\subsubsection{\ene ㇏}\label{㇏} %%%%1 [4, 0]
\index[esquinas]{$3000_{0}$!{\ene ㇏}}\index[fon]{yi!{\ene ㇏}} 
 {\Large 1 [4, 0] \textbf{ \begin{tabular}{|c | } \hline
yí \\ \hline\end{tabular} } }\\ \par 
\textit{observación: no confundir con el símbolo} {\ene 乀} \textit{del alfabeto zhuyin.}\\
verbo\ \*\ correr. 

%\subsubsection{\esetc  𠂆} % xh 𠂆
%\index[esquinas]{$7220_{0}$!{\esetc 𠂆}}\index[fon]{yi!{\esetc 𠂆}} 
%{\Large 2 [4, 1] \textbf{ \begin{tabular}{|c | } \hline yì \\ \hline\end{tabular} } }\\ \par 
%I\ verbo\ tirar; arrastrar.\par
%II\ adjetivo\ diáfano.\par
%III\ sustantivo\ \textit{trazo} {\esetc 𠂆} (\guillemotleft yì\guillemotright en caligrafía)

\subsubsection{\ene  𠂇}\label{𠂇} %% 𠂇 xk3
\index[esquinas]{$4000_{0}$!{\ene 𠂇}}\index[fon]{zuo!{\ene 𠂇}} 
 {\Large 2 [4, 1]} \enlugarde\ {\ene 左} \vease\ \textnumero\ ref.
 
\subsubsection{\ene 乃}\label{乃} %%% 乃 2[4, 1] nhs 乃
\index[esquinas]{$1722_{7}$!{\ene 乃}}
\index[fon]{nai!{\ene 乃}} \index[fon]{ai!{\ene 乃}} 
 {\Large 2 [4, 1] \textbf{ \begin{tabular}{|c | } \hline
n\v{a}i \\ \hline\end{tabular} } }tablaencompuestostambien\ {\Large \textbf{\begin{tabular}{|c | } \hline
\atres i \\ \hline\end{tabular} } }\\ \par 
 \textit{cópula en oraciones con predicado nominal, frecuentemente con énfasis (insistencia):} precisamente, exactamente;
dos\ \textit{conjunción adverbial frente al predicado en oraciones subordinadas en el lenguaje literario, a menudo con énfasis y significado principal} entonces. \textit{el siguiente contexto merece especial atención:}
a) \textit{después de una oración temporal suplementaria:} entonces, y entonces, sólo cuando, siempre y cuando. b) \textit{después de oraciones suplementarias condicionales:} entonces, en tal caso. c) \
\entry{乃組}{n\v{a} i-z'\v{u}}{}{uno\ tu tío; tu padre; dos\ tu ancestro, tu clan (género). }
\entry{乃是}{n\v{a}i' shì}{}{precisamente; siendo; sea.}
\entry{乃父}{n\v{a}i-fù}{}{tu padre; yo (el padre de uno).}
\entry{乃公}{n\v{a}i-g\={o}ng}{}{uno\ tu padre; dos\ yo, mío.}


\subsubsection{\ene 㐅}\label{㐅} %%%% k5 㐅
\index[esquinas]{$4000_{0}$!{\ene 㐅}}\index[fon]{wu!{\ene 㐅}} 
{\Large 2 [4, 1]}  \enlugarde\ {\ene 五} \vease\ \textnumero\ ref.

\subsubsection{\ene  乂}\label{乂} %%%% 2 [4, 1]
\index[esquinas]{$4000_{0}$!{\ene 乂}}\index[fon]{yi!{\ene 乂}}
\index[fon]{si!{\ene 乂}} \index[fon]{ai!{\ene 乂}} 
{\Large 2 [4, 1] \textbf{ \begin{tabular}{| c | c | c |} \hline
yì & ài  & sì \\ \hline\end{tabular} } }\\ \par
verbo\ uno\ \textbf{yì} regularizar; dirigir, enderezar, poner en orden; 
dos\ \textbf{yì} \enlugarde\ {\ene 刈} (\textit{segar, cortar});
I\ \textbf{sì} numeral\ cifra 4 (\textit{trazo comercial abreviado}).
\entry{乂粟}{yì-sù}{}{dialectal\ maíz, mazorca} %%%粟 mwfd
%\Large{  \def\stackalignment{r}%
%\topinset{ \begin{turn}{-4}\ene 乀\end{turn}}{\ene   乂}{}{2.9pt}} 2 [4, 1]
\subsubsection{\ene  乡}\label{乡} %%%% 乡 vvh %%%%鄉 vhi3
\index[esquinas]{$2020_{2}$!{\ene 乡}}\index[fon]{xiang!{\ene 乡}} 
 {\Large 3 [4, 2]} \abreviacion\ \enlugarde\ {\ene 鄉} \vease\ \textnumero\ ref.
 
 \subsubsection{\ene  久}\label{久} %%%% no 久 %%%%鄉 vhi3
 \index[esquinas]{$2780_{0}$!{\ene 久}}\index[fon]{jiu!{\ene 久}}
 {\Large 3 [4, 2]  \textbf{ \begin{tabular}{|c | } \hline
ji\utres \\ \hline\end{tabular} } }\\ \par 
 I. djetivo/adverbio\ uno\ extenso, largo; continuación, prolongación 
 
 \subsubsection{\ene  乇}\label{乇} %%%%  %%%%hp 乇
 \index[esquinas]{$2071_{4}$!{\ene 乇}}\index[fon]{zhe!{\ene 乇}}
{\Large 3 [4, 2] \textbf{ \begin{tabular}{|c | } \hline
zhè \\ \hline\end{tabular} } }\\ \par
sustantivo\ hojas [de pasto] vegetales; crecimiento [de pasto].
 
  \subsubsection{\ene  么}\label{么} %%%% 么 hi2
 \index[esquinas]{$2073_{2}$!{\ene 么 }}
\index[fon]{yao!{\ene 么}}\index[fon]{ma!{\ene 么}}
{\Large 3 [4, 2]} 
\begin{enumerate}[label=\Roman*]
  \item
 \abreviacion\ \enlugarde\ {\ene 幺} \vease\ \textnumero\ ref.
\item
\abreviacion\ \enlugarde\ {\ene 麽} \vease\ \textnumero\ ref.
 \end{enumerate}
 
 \subsubsection{\ene 币}\label{币} %%%% hlb 币 %%%% slb 匝
 \index[esquinas]{$2022_{7}$!{\ene 币}}\index[fon]{za!{\ene 币}}
 {\Large 4 [4, 3], [50, 1]} \abreviacion\ \enlugarde\ {\ene 匝} \vease\ \textnumero\ ref.
 
\subsubsection{\ene 长}\label{长} %%%% lhmo 长 %%%% smv 長
\index[esquinas]{$4273_{0}$!{\ene 长}}
  \index[fon]{chang!{\ene 长}}\index[fon]{zhang!{\ene 长}}
 {\Large 4 [4, 3], [168, 0]} \abreviacion\ \enlugarde\ {\ene 長} \vease\ \textnumero\ ref.
 
 \subsubsection{\ene 之}\label{之} %%%% lhmo 长 %%%% smv 長
 \index[esquinas]{$3030_{2}$!{\ene 之}}
\index[fon]{zhi!{\ene 之}}
{\Large 4 [4, 3] \textbf{ \begin{tabular}{| c  | c |} \hline
zh\iuno & -zh\icero   \\ \hline\end{tabular} } }\\ \par
I. \textbf{-zh\r{\i}} pauxiliar\\
uno\ \textit{palabra auxiliar del lenguaje literario, usada para separar un determinante previo del siguiente; forma palabras compuestas atributivas;}
{\ene 魯迅之文章} obra de Lu Xìn. %%%% 魯 nwfa

\subsubsection{\ene 玍}\label{玍} %
 \index[esquinas]{$2110_{4}$!{\ene 玍}}\index[fon]{ga!{\ene 玍}}   
{\Large 5 [4, 4] \textbf{ \begin{tabular}{| c |} \hline
g\v{a} \\ \hline\end{tabular} } }\\ \par
I. adjetivo\ uno\ dialectal\ reacio; pendenciero; intransigente; {\ene 他脾氣很玍} el es todo un terco; dos\ ávido; tacaño, avaro.\par
II. verbo\ dialectal\ uno\ cortar, rebanar.\par
\entry{玍古}{g\v{a}'g\={u}}{}{uno\ extraño, anormal; raramente; terco, obstinado.}
 

\subsubsection{\ene 乍}\label{乍} %%%% hs2 乍 %%%% sm
  \index[esquinas]{$8021_{1}$!{\ene 乍}}
\index[fon]{zha!{\ene 乍}}\index[fon]{zuo!{\ene 乍}}
{\Large 5 [4, 4] \textbf{ \begin{tabular}{| c | c |} \hline
zhà & zuò  \\ \hline\end{tabular} } }\\ \par
I. \textbf{zhà} adverbio\ uno\ súbito, de pronto; casualmente, fortuitamente;
\textit{en duplicación} en caso\ldots, en caso \ldots; {\ene 富乍} enriquecimiento fortuito; {\ene 天氣乍冷熱} el clima de pronto se enfría, de pronto se sofoca; dos\ primero, sólo si; {\ene 乍到} sólo si llega [al lugar]; {\ene 看很好} a primera vista parece muy bueno; tres\ preferentemente, mejor. \par
II. verbo\ uno\ \textbf{zhà} poner (colocar) en puntas; poner en guardia, poner en alerta, {\ene 毛乍} poner los pelos de punta (POREJEMPLO\ \textit{de miedo}). \par
III. nombrepropio\ Zhà (familia)


\subsubsection{\ene 乎}\label{乎} %%%% hfd 乎 %%%% sm
 \index[esquinas]{$2040_{9}$!{\ene 乎}}\index[fon]{hu!{\ene 乎}}
{\Large 5 [4, 4] \textbf{ \begin{tabular}{| c |} \hline
h\={u}  \\ \hline\end{tabular} } }\\ \par
I. pmodal: uno\ \textit{partícula interrogativa del lenguaje literario, colocada al final de la oración, sin pertenecer a la palabra interrogativa} comparar\ si \textit{de la} ; a) \textit{en oraciones interrogativas comunes}; 
{\ene 子見夫子乎}? ¿si has visto al maestro (\textit{Confucio})?; b) en preguntas retóricas;  {\ene 可以人而不如乎鳥} ¿siendo hombre se puede estar peor que un ave?

 \subsubsection{\ene 鸟} %%%% 鸟 hvsm
 \index[esquinas]{$2712_{7}$!{\ene 鸟}}
\index[fon]{niao!{\ene 鸟}}\index[fon]{diao!{\ene 鸟}}
 {\Large 5 [4, 4], [196, 0]} \abreviacion\ \enlugarde\ {\ene 鳥} \vease\ \textnumero\ ref.
 
\subsubsection{\ene 乏}\label{乏} %%%% 乏 hino
 \index[esquinas]{$2030_{2}$!{\ene 乏}}\index[fon]{fa!{\ene 乏}}
{\Large 5 [4, 4] \textbf{ \begin{tabular}{| c |} \hline
fá  \\ \hline\end{tabular} } }\\ \par 
I. verbo\ uno\ no bastar, hacer falta.\par
II. sustantivo\ uno\ insuficiencia, escasez, carencia, deficit; necesidad, pobre; vacante.\par
III. adjetivo\ uno\ escaso, insuficiente; dos\ impropio; incompleto.

 \subsubsection{\ene 后}\label{后} %%%% hmr 后
\index[esquinas]{$7226_{1}$!{\ene 后}}\index[fon]{hou!{\ene 后}}
 {\Large 6 [4, 5], [30, 3]} \abreviacion\ \enlugarde\ {\ene 從} \vease\ \textnumero\ ref.
 \subsubsection*{\ene 后}%\label{后} %%%% hmr 后
\index[esquinas]{$7226_{1}$!{\ene 后}}\index[fon]{hou!{\ene 后}}
 {\Large 6 [4, 5], [30, 3]} \abreviacion\ \enlugarde\ {\ene 從} \vease\ \textnumero\ ref.

 \subsubsection{\ene 𠂤}\label{𠂤} %%%% hrlr 𨸏
  \index[esquinas]{$2777_{7}$!{\ene 𠂤}}
\index[fon]{dui!{\ene 𠂤}}\index[fon]{zui!{\ene 𠂤}}
 {\Large 6 [4, 5]} \abreviacion\ \enlugarde\ {\ene 堆} \vease\ \textnumero\ ref.
 
 \subsubsection{\ene 乒}\label{乒} %%%% omh 乒 %%%% sm
 \index[esquinas]{$7220_{1}$!{\ene 乒}}\index[fon]{ping!{\ene 乒}}
{\Large 6 [4, 5] \textbf{ \begin{tabular}{| c |} \hline
p\={\i}ng  \\ \hline\end{tabular} } }\\ \par 
onomatopeyico\ ¡pam! ¡bang! \textit{onomatopeya de golpe, crujido.}
\entry{乒乓}{p\={\i} ngpang'}{}{uno\ \textit{partícula onomatopéyica del choque de guijarros,} POREJEMPLO\ \textit{granizo en el tejado}; dos\ \abreviacion\ ping-pong, tenis de mesa.}
\entry{乒乓球}{p\={\i} ng nangqiú'}{}{uno\ pelota de ping-pong; dos\ tenis de mesa.}

 \subsubsection{\ene 㐆}\label{㐆} %%%% hsms 㐆 %%%% sm
\index[esquinas]{$2022_{7}$!{\ene 㐆}}\index[fon]{yin!{\ene 㐆}}
{\Large 6 [4, 5] \textbf{ \begin{tabular}{| c |} \hline
y\={\i} n  \\ \hline\end{tabular} } }\\ \par 
verbo\*\ volverse, voltearse.

 \subsubsection{\ene 丟}\label{丟} %%%% omh 乒 %%%% sm
   \index[esquinas]{$2073_{2}$!{\ene 丟}}\index[fon]{diu!{\ene 丟}}
{\Large 6 [4, 5] \textbf{ \begin{tabular}{| c |} \hline
di\={u} \\ \hline\end{tabular} } }\\ \par 
verbo\ uno\  perder, extraviar; dos\ echar, arrojar, lanzar; tres\ abandonar, dejar.

\subsubsection{\ene 乖}\label{乖} %%%% hjlp 乖 %%%% sm
\index[esquinas]{$2011_{2}$!{\ene 乖}}\index[fon]{guai!{\ene 乖}}
{\Large 8 [4, 7] \textbf{ \begin{tabular}{| c |} \hline
gu\={a}i  \\ \hline
\end{tabular} } }\\ \par 
I. adjetivo\ uno\ obediente; {\ene 㧡} %%%% qyvo 㧡
razonable; dos\ ingenioso, hábil, ágil.\par
II. verbo\ uno\ negar; ir en contra (ser contrario a); perturbar; retroceder, retirarse (de algo); renunciar; apartarse.

 \subsubsection{\ene 乘}\label{乘} %%%% hdlp 乘 %%%% sm
  \index[esquinas]{$2090_{1}$!{\ene 乘}}
\index[fon]{cheng!{\ene 乘}}\index[fon]{sheng!{\ene 乘}}
{\Large 9 [4, 8] \textbf{ \begin{tabular}{| c | c |} \hline
chéng & shèng  \\ \hline\end{tabular} } }\\ \par 
I. \textbf{chéng} verbo\ uno\ sentarse adentro de\ldots, en\ldots (POREJEMPLO\ \textit{en el vagón, en el bote}); cargar en\ldots (adentro de); ir (navegar) en\ldots; dos\ servirse (POREJEMPLO\ \textit{ de una oportunidad}), utilizar, usar.
%4
\setcounter{subsss}{\value{subsubsection}} 

\subsection{\ene 乙}\label{cincos} 
\setcounter{subsubsection}{\arabic{subsss}} %\input{1trazo/dos} 
\subsubsection{\ene ㄥ}\label{ㄥ} %%% 
 \index[esquinas]{$2071_{0}$!{\ene ㄥ}}\index[fon]{gong!{\ene ㄥ}}
{\Large 1 [5, 0] \*} \enlugarde {\ene 肱} %%%% bki2 %% yyyyk ㄥ  
\vease\ \textnumero\ ref.

\subsubsection{\ene 乚}\label{乚} %% xu
 \index[esquinas]{$2071_{0}$!{\ene 乚}}\index[fon]{yin!{\ene 乚}}
{\Large 1 [5, 0] \*} \enlugarde {\ene 隱} %%%% nlbmp 隱 %% yyyyk ㄥ  
\vease\ \textnumero\ ref.

\subsubsection{\ene 乙}\label{乙} %nu
 \index[esquinas]{$1771_{0}$!{\ene 乙}}\index[fon]{yi!{\ene 乙}}
 {\Large 1 [5, 0]  \textbf{ \begin{tabular}{|c | } \hline %\large
y\={\i}  \\ \hline\end{tabular} } }\\ \par
I. CONVENCION\ uno\ y (\textit{segundo símbolo del ciclo decimal; asociado con el sector SEE de la cúpula celeste, elemento {\ene 木} árbol en la medicina china con el hígado}); dos\ \textit{segundo punto de conteo:}
II, 2), B, b), $\beta$; {\ene 乙細胞} casilla B. \par
II.sustantivo\*\  golondrina. \par %ласточка
 III. nombrepropio\ antiguo\ Yi (familia).
 
\subsubsection{\ene 乜}\label{乜} %pn
 \index[esquinas]{$4771_{0}$!{\ene 乜}}\index[fon]{nie!{\ene 乜}}
 {\Large 2 [5, 1] \textbf{ \begin{tabular}{|c | } \hline
niè \\ \hline
\end{tabular} } } {\Large \textbf{
\begin{tabular}{|c | } \hline
mi\euno \\ \hline\end{tabular} } }\\ \par 
nombrepropio\ Ne (familia).

\subsubsection{\ene 九} \label{九}%kn
 \index[esquinas]{$4001_{6}$!{\ene 九}}\index[fon]{jiu!{\ene 九}}
{\Large 2 [5, 1] \textbf{ \begin{tabular}{| c | c |} \hline
ji\utres & ji\uuno   \\ \hline\end{tabular} } }\\ \par
I. \textbf{ji\v{u}} numeral/adjetivo/adverbio\ uno\ nueve, 9; {\ene 九天} noveno día; {\ene 九十九} 99.

\subsubsection{\ene 也}\label{也} 
\index[esquinas]{$4471_{2}$!{\ene 也}}
\index[fon]{ye!{\ene 也}}
 {\Large 3 [5, 2] \textbf{  \begin{tabular}{|c | } \hline %\large
y\etres  \\ \hline\end{tabular} } }\\ \par
I. \textit{conjunción adverbial}\par
uno\ también; tal; y (\textit{con negación:} y además); {\ene 他去, 我也去}
[si] el va, yo también; {\ene 他去, 我也不去} [incluso si] el va, yo además no voy.\par
dos\ y$\ldots$; y$\ldots$, y$\ldots$; (\textit{relaciona algunos predicados con la misma partícula verbal, habitualmente repetida ante cada uno de ellos}); {\ene 他也看書, 也看報} %%%% gjsle
el lee libros y revistas.\par
II. nombrepropio\par
uno\ (\abreviacion\ \enlugarde\ {\ene 也門}) Yemen; yemenita.\par
dos\ Ye (familia).

\subsubsection{\ene 飞}\label{飞} %no3 % 飛 nohto
 \index[esquinas]{$1201_{3}$!{\ene 飞}}\index[fon]{fei!{\ene 飞}}
{\Large 3 [5, 2]} \abreviacion\ \enlugarde\  {\ene 飛} \vease\ \textnumero\ ref{}


\subsubsection{\ene 乞}\label{乞} %on 
 \index[esquinas]{$8071_{7}$!{\ene 乞}}\index[fon]{qi!{\ene 乞}}
{\Large 3 [5, 2] \textbf{ \begin{tabular}{| c | c |} \hline
q\itres & qì   \\ \hline\end{tabular} } }\\ \par
I. verbo\ uno\ \textbf{q\v{\i}} pedir, solicitar; implorar, suplicar.\par
dos\ \textbf{q\v{\i}} pedir limosna, limosnear; {\ene 行乞} vagar y limosnear. \par
II. sustantivo\ uno\ \textbf{q\v{\i}} petición, ruego.

\subsubsection{\ene 乣}\label{乣} % %viu 
 \index[esquinas]{$2271_{0}$!{\ene 乣}}\index[fon]{diu!{\ene 乣}}
 {\Large 4 [5, 3] \textbf{ \begin{tabular}{|c | } \hline %\large
di\={u}  \\ \hline\end{tabular}} }\\ \par
soloencompuestos.\par %%% bjwj 軍

\subsubsection{\ene 电}\label{电} %lwu
 \index[esquinas]{$5071_{6}$!{\ene 电}}\index[fon]{dian!{\ene 电}}
{\Large 5 [5, 4]} \abreviacion\ \enlugarde\  {\ene 電} \vease\ \textnumero\ ref{} %nxu

\subsubsection{\ene 乩}\label{乩} %xyru 
 \index[esquinas]{$2261_{0}$!{\ene 乩}}\index[fon]{ji!{\ene 乩}}
 {\Large 4 [5, 3] \textbf{  \begin{tabular}{|c | } \hline %\large
j\={\i}  \\ \hline\end{tabular} } }\\ \par
verbo\ antiguo\  adivinar en el plato con arena;
{\ene 扶乩} adivinar mediante preguntas directas a los espíritus (\textit{en el animismo chino}).

\subsubsection{\ene 乱}\label{乱} % hru2
 \index[esquinas]{$2261_{0}$!{\ene 乱}}
\index[fon]{luan!{\ene 乱}}\index[fon]{lan!{\ene 乱}}
{\Large 7 [5, 6]} \abreviacion\ \enlugarde\  {\ene 亂} \vease\ \textnumero\ \ref{亂}

\subsubsection{\ene 龟}\label{龟} % nwu 亀 
 \index[esquinas]{$2771_{6}$!{\ene 龟}}\index[fon]{gui!{\ene 龟}}
\index[fon]{qiu!{\ene 龟}}\index[fon]{jun!{\ene 龟}}
{\Large 7 [5, 6]} \abreviacion\ \enlugarde\  {\ene 龜} \vease\ \textnumero\ ref{} %nxu

\subsubsection{\ene 乹}\label{乹} %jju 乹 xjju
 \index[esquinas]{$4241_{0}$!{\ene 乹}}
\index[fon]{gan!{\ene 乹}}\index[fon]{qian!{\ene 乹}}
{\Large 9 [5, 8]} \abreviacion\ \enlugarde\  {\ene 乾} \vease\ \textnumero\ \ref{乾} %nxu

\subsubsection{\ene 乾}\label{乾} %乾 jjon 
 \index[esquinas]{$4841_{7}$!{\ene 乾}}
\index[fon]{gan!{\ene 乾}}\index[fon]{qian!{\ene 乾}}
{\Large 11 [5, 10] \textbf{  \begin{tabular}{| c | c |} \hline
g\auno n & qián   \\ \hline\end{tabular} } }\\ \par
I. \textbf{g\={a}n} adjetivo/adverbio/ uno\ seco, secado, desecado; 
dos\ sin agua, desértico.

\subsubsection{\ene 亁}\label{亁} %on 亁 jjon2
 \index[esquinas]{$4841_{7}$!{\ene 亁}}
\index[fon]{gan!{\ene 亁}}\index[fon]{qian!{\ene 亁}}
{\Large 12 [5, 11]} \abreviacion\ \enlugarde\  {\ene 乾} \vease\ \textnumero\ \ref{乾} %nxu

\subsubsection{\ene 亂}\label{亂} %bbu 
 \index[esquinas]{$2221_{0}$!{\ene 亂}}
\index[fon]{luan!{\ene 亂}}\index[fon]{lan!{\ene 亂}}
{\Large 11 [5, 10] \textbf{ \begin{tabular}{| c c |} \hline
luàn & làn   \\ \hline\end{tabular} }} \  
{ \begin{tabular}{ c  } %\hline
\textit{en dialecto}   \\ %\hline
\textit{pekinés también}  \\% \hline
\end{tabular} }\ 
 {\Large \textbf{ \begin{tabular}{|c | } \hline %\large
luán \\ \hline\end{tabular} } }\\ \par
I. adjetivo/adverbio\ uno\ enmarañado, enredado, confuso; mezclado; desordenado; dos\ irreflexivo, desconsiderado; absurdo; excesivo, desorbitante, desmedido; tres\ rebelde, traicionero.\par
II. verbo\ enredar, enmarañar; confundirse.\par
III. sustantivo\ uno\ desorden, caos; {\ene 大亂} caos total; dos\ disturbio, discordia.
\entry{亂亡}{luàng-wáng}{}{uno\ perderse en el revuelo; dos\ devastar, arruinar (POREJEMPLO\ \textit{un país}) como resultado de la rebelión.}
\entry{亂忙}{luàngmáng'}{}{agitarse, agitación, alboroto.}
\entry{亂臣}{luàn-chén}{}{uno\* dignatario activo, funcionario estatal capaz; dos\ súbdito sedicioso; funcionario rebelde.}
\entry{亂世}{luànshì'}{}{siglo agitado; tiempos revueltos; tiempos de guerra, época de revueltas}
%\\ \par 
\setcounter{subsss}{\value{subsubsection}} 

\subsection{\ene 亅}\label{seiss} 
\setcounter{subsubsection}{\arabic{subsss}} %\input{1trazo/dos} 
\subsubsection{\ene 亅}\label{亅}%xn n2
\index[esquinas]{$2000_{2}$!{\ene 亅}}
\index[fon]{jue!{\ene 亅}}
{\Large 2 [6, 1] \textbf{ \begin{tabular}{|c | } \hline
jué  \\ \hline\end{tabular} } }\\ \par
CONVENCION\ gancho (\textit{en caligrafía}).

\subsubsection{\ene 了}\label{了}
\index[esquinas]{$1720_{7}$!{\ene 了}}\index[fon]{liao!{\ene 了}} 
\index[fon]{le!{\ene 了}} 
 {\Large 2 [6, 1] \textbf{ \begin{tabular}{|c | c | c | } \hline
li\v{a}o & liào & -l\={e}  \\ \hline\end{tabular} } }\\ \par
\textit{observación: no confundir con el símbolo} {\ene ^^^^31e1 }\textit{ del alfabeto zhuyin}\\
I. verbo\ uno\ \textbf{li\v{a}o} dividir, partir.


\subsubsection{\ene 予}\label{予} %%%%% ninn 予
\index[esquinas]{$1720_{2}$!{\ene 予}}\index[fon]{yuo!{\ene 予}} 
\index[fon]{yu!{\ene 予}} \index[fon]{zhu!{\ene 予}} 
 {\Large 4 [6, 3] \textbf{ \begin{tabular}{|c | c | c | } \hline
y\v{u} o & yú & zh\v{u}  \\ \hline\end{tabular} } }\\ \par
I. \textbf{y\v{u}} verbo\ uno\ dar; anhelar.
\subsubsection{\ene 事}\label{事} %\IN{er!{\enese 二}}
\index[esquinas]{$5000_{7}$!{\ene 事}}\index[fon]{shi!{\ene 事}}
{\Large 8 [6, 7]  \textbf{\begin{tabular}{|c | } \hline %\large
shì  \\ \hline \end{tabular} } } \\ \par
\begin{enumerate} [noitemsep,label=\Roman{enumi}.,ref=\Roman{enumi}, leftmargin=*]
\item \textit{sustantivo/clasificador}  {\begin{enumerate*}[label=\textbf{\arabic*})]
\item hecho, acto; \item incidente, suceso.
\end{enumerate*}}
\item \textit{verbo} {\begin{enumerate*}[label=\textbf{\arabic*})]
\item  hacer, ocuparse (con); servir.
\end{enumerate*}}
\end{enumerate}
\setcounter{subsss}{\value{subsubsection}} 
